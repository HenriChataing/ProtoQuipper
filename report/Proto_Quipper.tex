\documentclass[twoside]{article}

\usepackage{times}
\usepackage{fancyhdr}

% ----------------------------------------------------------------------
% Packages:

\usepackage{amssymb}          % Usual AMS packages.
\usepackage{amsmath}
\usepackage{amsthm}
\usepackage{proof}            % For proof trees.
\usepackage{mdframed}         % To box figures.
\usepackage{verbatim}
\input{Qcircuit}              % To draw quantum circuits.
\usepackage{geometry}
\usepackage{graphicx}
\usepackage{bussproofs}
\usepackage[utf8]{inputenc}
\usepackage[]{hyperref}
\usepackage{cite}


% ----------------------------------------------------------------------
% Margins: An environment to locally change the margins.

\newenvironment{changemargin}[2]{%
\begin{list}{}{%
\setlength{\topsep}{0pt}%
\setlength{\leftmargin}{#1}%
\setlength{\rightmargin}{#2}%
\setlength{\listparindent}{\parindent}%
\setlength{\itemindent}{\parindent}%
\setlength{\parsep}{\parskip}%
}%
\item[]}{\end{list}}


% ----------------------------------------------------------------------
% Theorem definitions:
%
% theorem, lemma, proposition, corollary, definition, example, 
% examples, openproblem, principle, remark, remarks, convention, 
% notation
%
% All are numbered by default. Get unnumbered versions by un-lemma etc.

\newtheorem{n-lemma}{Lemma}[section]
\newtheorem{n-proposition}[n-lemma]{Proposition}
\newtheorem{n-theorem}[n-lemma]{Theorem}
\newtheorem{n-principle}[n-lemma]{Principle}
\newtheorem{n-corollary}[n-lemma]{Corollary} 
\newtheorem*{un-lemma}{Lemma}
\newtheorem*{un-theorem}{Theorem}
\newtheorem*{un-proposition}{Proposition}
\newtheorem*{un-corollary}{Corollary} 
\newtheorem*{un-principle}{Principle}

\theoremstyle{definition}
\newtheorem{n-definition}[n-lemma]{Definition}
\newtheorem{n-assumption}[n-lemma]{Assumption}
\newtheorem{n-algorithm}[n-lemma]{Algorithm}
\newtheorem{n-openproblem}[n-lemma]{Open Problem}
\newtheorem*{un-definition}{Definition}
\newtheorem*{un-assumption}{Assumption}
\newtheorem*{un-openproblem}{Open Problem}

\theoremstyle{remark}
\newtheorem{n-example}[n-lemma]{Example}
\newtheorem{n-examples}[n-lemma]{Example}
\newtheorem{n-remark}[n-lemma]{Remark}
\newtheorem{n-notation}[n-lemma]{Notation}
\newtheorem{n-remarks}[n-lemma]{Remarks}
\newtheorem{n-convention}[n-lemma]{Convention}
\newtheorem*{un-example}{Example}
\newtheorem*{un-examples}{Examples}
\newtheorem*{un-remark}{Remark}
\newtheorem*{un-remarks}{Remarks}
\newtheorem*{un-convention}{Convention}

\newenvironment{theorem}{\begin{n-theorem}}{\end{n-theorem}}
\newenvironment{lemma}{\begin{n-lemma}}{\end{n-lemma}}
\newenvironment{proposition}{\begin{n-proposition}}{\end{n-proposition}}
\newenvironment{corollary}{\begin{n-corollary}}{\end{n-corollary}}
\newenvironment{definition}{\begin{n-definition}}{\end{n-definition}}
\newenvironment{example}{\begin{n-example}}{\end{n-example}}
\newenvironment{examples}{\begin{n-examples}}{\end{n-examples}}
\newenvironment{openproblem}{\begin{n-openproblem}}{\end{n-openproblem}}
\newenvironment{principle}{\begin{n-principle}}{\end{n-principle}}
\newenvironment{remark}{\begin{n-remark}}{\end{n-remark}}
\newenvironment{remarks}{\begin{n-remarks}}{\end{n-remarks}}
\newenvironment{convention}{\begin{n-convention}}{\end{n-convention}}
\newenvironment{notation}{\begin{n-notation}}{\end{n-notation}}
\newenvironment{algorithm}{\begin{n-algorithm}}{\end{n-algorithm}}

\theoremstyle{plain}
\newtheorem{thm}[n-lemma]{Theorem}     % reset theorem numbering for 
                                       % each chapter

\theoremstyle{definition}
\newtheorem{defn}[n-lemma]{Definition} % definition numbers are dependent 
                                       % on theorem numbers
\newtheorem{exmp}[n-lemma]{Example}    % same for example numbers
\newtheorem{prop}[n-lemma]{Proposition}
\newtheorem{coro}[n-lemma]{Corollary}

% ----------------------------------------------------------------------
% Proofs.
\newenvironment{proofof}[1]                     % loose proofs
        {\begin{trivlist}\item[]{\it Proof of #1:\hspace{.5em}}\rm}
        {\end{trivlist}}
\newenvironment{proofsketch}                    % proof sketch
        {\begin{trivlist}\item[]{\it Proof sketch:\hspace{.5em}}\rm}
        {\end{trivlist}}

% ----------------------------------------------------------------------
% Set and category notation.
\newcommand{\Cc}{{\bf C}}
\newcommand{\Dd}{{\bf D}}
\newcommand{\abs}[1]{|{#1}|}                  % |A|: underlying set.
\newcommand{\homof}[2]{\textrm{hom}_{#1}(#2)} % hom set 
                                              % \homof{C}{A,B} = C(A,B).
\renewcommand{\hom}[1]{\homof{}{#1}}          % hom set \hom{A,B} = (A,B).
\newcommand{\obj}[1]{|#1|}                    % |C|: objects of a cat.
\newcommand{\seq}{\subseteq}    
\newcommand{\such}{\,\,|\,\,}                 % in sets {x|...}.
\newcommand{\cp}{\circ}                       % composition.
\newcommand{\id}{{\textrm{\rm id}}}           % identity.
\newcommand{\from}{\colon}                    % F\from A\ii B = F:A->B.
\newcommand{\ii}{\rightarrow}                 % functions f:a->b 
                                              % (alternative: \to).
\newcommand{\iii}{\longrightarrow}
\newcommand{\pii}{\rightharpoonup}            % partial function f:a-`b.
\newcommand{\adjoint}{\dashv}                 % F\adjoint G: F is left 
                                              % adjoint.
\newcommand{\iso}{\cong}                      % isomorph ~=.
\newcommand{\family}[2]{(#1)_{#2}}            % (X_n)_J or {X_n|J}..
\newcommand{\rest}[1]{|_{#1}}                 % restriction S|A.
\newcommand{\atuple}[1]{\langle#1\rangle}     % <a,b,c...>.
\newcommand{\rtuple}[1]{(#1)}                 % round tuple (a,b,c...).
\newcommand{\tuple}{\rtuple}
\newcommand{\p}{\atuple}                      % <a,b>
\newcommand{\copair}[1]{[#1]}                 % [a,b]
\newcommand{\s}[1]{\{#1\}}                    % {a,b}
\newcommand{\ms}[1]{\{| #1 |\}}               % {|a,b|}
\newcommand{\es}{\emptyset}                   % empty set
\newcommand{\defeq}{:=}                       % := or =_{def}
\renewcommand{\iff}{\Leftarrow\!\!\!\Rightarrow}
\newcommand{\defiff}{\mathrel{{:}{\iff}}}     % :<=> or <=>_{def}
\newcommand{\pow}{{\mathscr{P}}}
\newcommand{\catarrow}{\xrightarrow}          % extensible arrow for 
                                              % cat notation
\newcommand{\da}{^\dagger}
\newcommand{\dada}{^{\dagger\dagger}}
\newcommand{\x}{\tensor}
\newcommand{\+}{\oplus}

% ----------------------------------------------------------------------
% particular categories and sets
\usepackage{mathrsfs}                      % special script style 
                                           % \mathscr{S} for \Set
\newcommand{\Set}{\mathscr{S}}             % Cat of Sets
\newcommand{\N}{\mathbb{N}}                % Nat'l numbers AMS fonts
\newcommand{\Z}{\mathbb{Z}}                % Integers AMS fonts
\newcommand{\R}{{\mathbb{R}}}              % Reals
\newcommand{\C}{{\mathbb{C}}}              % Complex
{\makeatletter                             % Cat of finite-dim Hilbert 
                                           % spaces
\gdef\alphalabels{\def\theenumi{\@alph\c@enumi}\def\labelenumi{(\theenumi)}}}
\newcommand{\FinHilb}{\textrm{\bf FinHilb}}

% ----------------------------------------------------------------------
% Spacing and notes.
\newcommand{\sep}{\hspace{2em}} 
\newcommand{\ssep}{\hspace{1em}} 
\newcommand{\todo}[1]{[#1]\marginpar{\mbox{\Huge !}}}
\newcommand{\toask}[1]{[#1]\marginpar{\mbox{\Huge ?}}}

% ----------------------------------------------------------------------
% Special symbols.
\newcommand{\entails}{\vdash}                     % turnstile
\newcommand{\semm}[1]{[\![#1]\!]}                 % semantic brackets
\newcommand{\sem}[2]{\semm{#1}_{#2}}              % semantic brackets 
                                                  % with subscript
\newcommand{\sems}[2]{\semm{#1}^{#2}}             % semantic brackets 
                                                  % w/ superscript
\newcommand{\semp}{\sem{\,\,\,}{}}                % \sem prototype
\newcommand{\eqbyrule}[1]{\stackrel{#1}{=}}       % annotated =
\newcommand{\downdeal}{{\downarrow}}              % downdeal
\newcommand{\updeal}{{\uparrow}}                  % updeal
\newcommand{\chk}{\raisebox{.2em}{\tiny$\surd$}}  % check in case 
                                                  % distinctions
\newcommand{\IH}{\textit{IH}}                     % ind. hyp.
\newcommand{\tensor}{\otimes}                     % tensor 
                                                  % (alternative: \x).
\newcommand{\bk}{\discretionary{}{}{}}            % optional break
\newcommand{\loli}{\multimap}                     % linear implication

% ----------------------------------------------------------------------
% BNF.
\newcommand{\bnf}{::=}                                   % ::= in bnf
\newcommand{\bor}{\ \ \rule[-.75ex]{.01in}{2.75ex}\ \ }  % | in bnf

% ----------------------------------------------------------------------
% Prettier symbols.
\renewcommand{\leq}{\leqslant}          
\renewcommand{\geq}{\geqslant}  


% ----------------------------------------------------------------------
% Quipper specifics. 
\def\true{\texttt{True}}                % Boolean: True
\def\false{\texttt{False}}              % Boolean: False
\def\type{\texttt{Type}}                % Type
\def\term{\texttt{Term}}                % Term
\def\val{\texttt{Val}}                  % Value
\def\qdatatype{\texttt{QDataType}}      % QDataType
\def\qdataterm{\texttt{QDataTerm}}      % QDataTerm
\def\qdata{\texttt{QData}}              % QData
\def\spec{\texttt{Spec}}                % Specimen of a type


\def\lset{\mathcal{L}}
\def\qset{\mathcal{Q}}
\def\vset{\mathcal{V}}
\def\cset{\mathcal{C}}
\def\fset{\mathcal{F}}
\def\pset{\mathcal{P}}


% ----------------------------------------------------------------------
% More of Henri's notation. 
\def\bang{{!}}
\newcommand{\pair}[2]{\langle #1, #2 \rangle}
\newcommand{\letin}[3]{\textrm{\rm let}~ #1 = #2 ~\textrm{\rm in}~ #3}
\newcommand{\letrecin}[3]{\textrm{\rm let rec}~ #1 = #2 ~\textrm{\rm in}~ #3}
\newcommand{\ifthenelse}[3]{\textrm{\rm if}~ #1 ~\textrm{\rm then}~ #2 ~\textrm{\rm else}~ #3}
\newcommand{\nbang}[1]{\bang^{#1}}
\newcommand{\binding}{\mathfrak{b}}              % Arbitrary binding

\newcommand{\QP}{{\rm QP}}
\newcommand{\FQ}{\mathop{\textrm{FQ}}\nolimits}    % free quantum variables
\newcommand{\FV}{\mathop{\textrm{FV}}\nolimits}    % free variables
\newcommand{\Circ}{\mathop{\textrm{Circ}}\nolimits} 
\newcommand{\qubit}{\textrm{\bf qubit}}
\newcommand{\bool}{\textrm{\bf bool}}
\newcommand{\bit}{\textrm{\bf bit}}
\newcommand{\intx}{\textrm{\bf int}}
\newcommand{\boxx}{\mathop{\textit{box}}\nolimits}
\newcommand{\dom}{\mathop{\textrm{\rm dom}}\nolimits}
\newcommand{\cod}{\mathop{\textrm{\rm cod}}\nolimits}
\newcommand{\unbox}{\mathop{\textit{unbox}}\nolimits}
\newcommand{\rev}{\mathop{\textit{rev}}\nolimits}
\newcommand{\In}{\mathop{\textrm{In}}\nolimits}
\newcommand{\Out}{\mathop{\textrm{Out}}\nolimits}
\newenvironment{compactitemize}{\begin{itemize}\itemsep -.8ex}{\end{itemize}}
\newcommand{\imp}{\Rightarrow}
\newcommand{\cupdot}{\mathbin{\mathaccent\cdot\cup}}
\newcommand{\rul}[1]{\textit{#1}}
\newcommand{\qutt}{\texttt{qubit}}
\newcommand{\ttt}{\mathbin{\texttt{*}}}
\newcommand{\leftx}{\textit{left}}
\newcommand{\rightx}{\textit{right}}
\newcommand{\finbij}{\textrm{\rm Bij}_f}
\newcommand{\synarg}[1]{\langle{\it #1}\rangle}
\newcommand{\bind}{\mathop{\textit{bind}}\nolimits}

\newenvironment{code}{\par\vspace{2ex}\noindent\begin{minipage}{\textwidth}\footnotesize}{\end{minipage}\vspace{2ex}\par}
\newenvironment{splitcode}{\par\vspace{2ex}\noindent\begin{minipage}{0.5\textwidth}\footnotesize}{\end{minipage}\vspace{2ex}\par}
\renewcommand{\split}{\end{minipage}\begin{minipage}{0.5\textwidth}\center}

%----------------------------------------------------------------------         
% a hack: \m is similar to \vcenter, but works better.                          
% \mp{0.2}{x}: raise the box x so that 20% of it is below the                   
% centerline. Unlike \vcenter, don't change the horizontal spacing              

\newlength{\localh}
\newlength{\locald}
\newbox\mybox
\def\mp#1#2{\setbox\mybox\hbox{#2}\localh\ht\mybox\locald\dp\mybox\addtolength{\localh}{-\locald}\raisebox{-#1\localh}{\box\mybox}}
\def\m#1{\mp{0.5}{#1}}


\oddsidemargin 1in
\evensidemargin 1in
\topmargin 1in
\textwidth 4.5in \textheight 6.6in
\headheight 0in
\headsep .1in
\voffset -.35in


% ----------------------------------------------------------------------
% General Information.
\title{Report on Proto-Quipper 0.1}
\author{Report authors: \\
  Henri Chataing (Dalhousie University)\\
  Neil J. Ross (Dalhousie University)\\
  Peter Selinger (Dalhousie University)
}
\date{Program PI:\\
  D. Scott Alexander (Applied Communication Sciences)}

\pagestyle{fancy}
\def\notice{\footnotesize FOR GOVERNMENT AND APPLIED COMMUNICATION SCIENCES
  (ACS) TEAM USE ONLY}
\fancyhead{}
\fancyhead[LO,RE]{\thepage}
\fancyfoot{}
\fancyfoot[LE]{\notice}

\renewcommand{\headrulewidth}{0pt}
\renewcommand{\footrulewidth}{0pt}

% ----------------------------------------------------------------------
\begin{document}
\maketitle
\thispagestyle{fancy}

This work was performed under U.S. Government contract D12PC00527.
\vspace{-1ex}

\tableofcontents

% ----------------------------------------------------------------------
\clearpage
\section{Introduction}

\subsection{Overview}

The PLATO team's primary quantum programming language, Quipper, is
currently implemented as an embedded language, within the host
language Haskell. This has many advantages, for example, the ability
to implement a large-scale system very rapidly. However, there are
also a few disadvantages. One of these is that the Haskell type
system, while providing many type-safety properties, is not in general
strong enough to ensure full type-safety of the quantum programs. In
the current Quipper implementation, it is therefore the programmer's
responsibility to ensure that quantum components are plugged together
in physically meaningful ways. This means that certain types of
programming errors will not be prevented by the compiler; in the worst
case, this may lead to ill-formed output or run-time errors in other
parts of the tool chain.

In this report, we describe a type-safe version of a small fragment of
Quipper, which we call {\em Proto-Quipper}. Proto-Quipper has been
implemented as a stand-alone (non-embedded) programming language.  The
Proto-Quipper language is designed to ``enforce the physics'', in the
sense that it will detect, at compile-time, programming errors that
could lead to ill-formed or undefined circuits.  

We note that type-safety in Proto-Quipper, as in all programming
languages, only guarantees the {\em well-formedness} of programs, and
not their {\em correctness}. Correctness, of course, remains the
programmer's responsibility. In particular, type-safety does not
guarantee anything about the correctness of {\em programmer
  assertions}, including the correct use of assertive termination
gates. 

We distinguish between the \emph{core} of Proto-Quipper, which 
represents the proved type-safe fragment of the language, and 
the \emph{extensions} of the language, containing newly 
implemented features of Proto-Quipper that are currently under 
development. 

\subsection{Outline}

This report is organized as follows. In Section~\ref{sec-design}, we
briefly explain the rationale behind Proto-Quipper's design, including
the differences between Quipper, Proto-Quipper, Core Proto-Quipper,
and the quantum lambda calculus. In Section~\ref{sec-core}, we
formally define the Core Proto-Quipper language, its type system, and
its operational semantics. In Section~\ref{sec-type-safety}, we prove
the type-safety of the language. In Section~\ref{sec-inference}, we
describe a type inference algorithm for Core Proto-Quipper. In
Section~\ref{sec-extensions}, we discuss some extensions of
Proto-Quipper that are already implemented in our language
implementation, but not yet formalized. In Section~\ref{sec-future},
we discuss possible future work, including how Proto-Quipper could be
extended with additional Quipper features.
Section~\ref{sec-user-guide} provides a user-level tutorial guide to
the Proto-Quipper implementation.

\subsection{Additional materials}

This document is accompanied by the Proto-Quipper software version
0.1:
\begin{compactitemize}
\item the Proto-Quipper language implementation, including
\item the Proto-Quipper browsable documentation (in the doc/ directory).
\end{compactitemize}

Please also note that a user-level tutorial introduction to the
Proto-Quipper implementation is given in
Section~\ref{sec-user-guide}. Readers who are looking for a software
guide, rather than the theory of Proto-Quipper, should skip ahead to
Section~\ref{sec-user-guide}.

% ----------------------------------------------------------------------
\clearpage
\section{Design rationale}\label{sec-design}

\subsection{Design goals}\label{ssec-goals}

In designing the Proto-Quipper language, our approach was to start
with a limited (but still expressive) fragment of the Quipper language
and make it completely type-safe.  This fragment serves as a robust
basis for future language extensions. The idea is to eventually close
the gap between Proto-Quipper and Quipper by extending Proto-Quipper
with one feature at a time while retaining type-safety. 

The current version of Proto-Quipper (0.1) was designed with two goals
in mind:
\begin{itemize}
  \item to incorporate Quipper's ability to generate and act on 
  quantum circuits, and 
\item to provide a linear type system to guarantee that the produced
  circuits are physically meaningful. In particular, properties like
  no-cloning are respected.
\end{itemize}

\subsection{Quantum lambda calculus}

An important precursor to Proto-Quipper is the quantum lambda calculus
{\cite{SeVa09}}. In the quantum lambda calculus, the execution of
programs is modelled by the so-called {\em reduction relation}. This
relation is defined on \emph{closures}, which are essentially pairs
$[C,t]$ consisting of a term of the language $t$ and a quantum state
$C$. The state is a unit vector in a complex Hilbert space and $t$ is
a term whose free variables are linked to qubits in $C$. The quantum
state is held in a quantum device capable of performing certain
operations (applying unitaries, measuring qubits,\ldots). The
reduction relation in the quantum lambda calculus is then defined as a
probabilistic rewrite procedure on these closures. Typically, the
reduction will be classical until a redex involving a quantum constant
is reached. At this point, the quantum device will be instructed to
perform the appropriate quantum operation. For example: ``Apply a
Hadamard gate to qubit number 3''.

\subsection{From quantum lambda calculus to Proto-Quipper}

One of the main differences between Quipper and the quantum lambda
calculus is that Quipper acts as a {\em circuit description language}.
Quipper provides the ability to treat circuits as data, and to
manipulate them as a whole.  For example, Quipper has operators for
reversing circuits, decomposing them into gate sets, optimization of
circuits, resource counting, and so on. By contrast, the quantum
lambda calculus only manipulates {\em qubits}. In other words, in
the quantum lambda calculus, all quantum operations are immediately
carried out on a quantum device, and not stored for symbolic
manipulation.

In the design of Proto-Quipper, we extended the quantum lambda
calculus with the minimal set of features that makes it Quipper-like.
Namely, we added the ability to create and manipulate circuits.  As
detailed later in this report, we equipped this language with a type
system and proved {\em type-safety}, i.e., the property that no
well-typed program will perform a non-physical operation.

The execution of Proto-Quipper programs has been formalized as a
reduction relation on closures, similarly to the quantum lambda
calculus.  However, a closure $[C,t]$ in Proto-Quipper consists of a
term $t$ and a \emph{circuit state} $C$. The state $C$ represents the
circuit currently being built. Instead of having a quantum device
capable of performing quantum operations, we assume that we have a
\emph{circuit constructor} capable or performing certain circuit
building operations (such as appending gates, reversing, etc.). The
reduction is then defined as a non-probabilistic rewrite procedure on
closures.  As in the quantum lambda calculus, some redexes will affect
the state by sending instructions to the circuit constructor. For
example: ``Append a Hadamard gate to wire number 3''.


% ----------------------------------------------------------------------
\clearpage
\section{The core language}\label{sec-core}

In this section, we define the core of Proto-Quipper. We 
present in details the syntax, type system and 
operational semantics of the language.

\subsection{Types and terms}
\label{ssec-types-and-terms}

\begin{definition} 
The \emph{types} of Proto-Quipper are defined by:
\begin{center}
\begin{tabular}{rrl}
$\texttt{Type}$&$~A,B~~\bnf$ & $ \qubit \bor 1 \bor \bool \bor A\tensor B \bor 
A\loli B \bor {!}A \bor \Circ (T,U).$\\
\end{tabular}
\end{center}
Among the types, we single out the subset of \emph{quantum data types}:
\begin{center}
\begin{tabular}{rrl}
$\qdatatype$&$~T,U~~\bnf$ & $\qubit \bor 1 \bor T \tensor U.$
\end{tabular}
\end{center}
\end{definition}

The type system of Proto-Quipper, like the type system of the quantum
lambda calculus, is based on intuitionistic linear logic
{\cite{Gir87}}.  It can be helpful to think of types as sets of
values. Under this interpretation, the types can be understood as
follows. The type $\bool$ represents the set of booleans, $1$ is the
unit type, $A\x B$ is the set of pairs composed of a left member of
type $A$ and of right member of type $B$, and $A\loli B$ is the set of
functions from $A$ to $B$. The type ${!}A$ can be understood as the
subset of $A$ consisting of values that have the additional property
of being \emph{duplicable} or \emph{reusable}. We will sometimes write
${!}^nA$, with $n\in\N$, to mean:
\[
\underbrace{{!}\ldots {!}}_{n} A.
\]
The remaining types are specific to Proto-Quipper. The elements of
$\qubit$ are (quantum) wire identifiers, i.e., essentially references
to a logical qubit within a computation. They can be thought of as
references to quantum bits on some physical device, or simply as
references to quantum wires within the circuit currently being
constructed. Elements of quantum data types describe sets of circuit
endpoints, and consist of tuples of wire identifiers. We can think of
these as describing circuit interfaces. Finally, the type $\Circ(T,U)$
is the set of all circuits having an input interface of type $T$ and 
an output interface of type $U$. 

\begin{definition}
  Assume three countable sets: a set $\vset$ of \emph{term variables},
  denoted $x,y,z,\ldots$; a set $\qset$ of \emph{quantum variables},
  denoted $q,q_1,q_2,\ldots$; and a set $\cset$ of \emph{circuit
    constants}, denoted $C,D,\ldots$. The \emph{terms} of
  Proto-Quipper are defined by:
\begin{center}
\begin{tabular}{rrl}
$\term$&$~a,b,c~~ \bnf$ & $x \bor q \bor (t,C,a) \bor \true 
  \bor \false \bor \p{a,b} \bor * \bor$ \\[0.05in]
& & $ab \bor \lambda x.a \bor \boxx^T \bor \unbox \bor \rev 
    \bor $\\[0.05in]
& & $\ifthenelse{a}{b}{c} \bor \letin{\p{x,y}}{a}{b}.$
\end{tabular}
\end{center}
Among the terms, we single out the subset of \emph{quantum data terms}:
\begin{center}
\begin{tabular}{rrl}
$\qdataterm$&$~t,u~~\bnf$ & $q \bor * \bor \p{t,u}.$
\end{tabular}
\end{center}
Moreover, we assume that $\cset$ is equipped with two functions 
$\In,\Out\from \cset\to\pset_f(\qset)$ and that $\qset$ is 
well-ordered. Here, $\pset_f(\qset)$ denotes the set of finite subsets
of $\qset$.
\end{definition}

The meaning of most terms is intended to be the standard one. For
example $\p{a,b}$ is the pair of $a$ and $b$, $\true$ and $\false$ are
the booleans and $\lambda x.a$ is the function which maps $x$ to
$a$. For the more unusual terms we have:
\begin{itemize}
\item A circuit constant $C$ represents a low-level quantum circuit.
  Because it would be complicated, and somewhat besides the point, to
  define a formal language for describing low-level quantum circuits,
  Proto-Quipper assumes that there exists a constant symbol for {\em
    every} possible quantum circuit. Each circuit $C$ is equipped with
  a finite set of {\em inputs} and a finite set of {\em outputs},
  which are subsets of the set of quantum variables $\qset$.

  Proto-Quipper's abstract treatment of quantum circuits is further
  explained in Section~\ref{ssec-circuit-cons}.

\item The term $(t,C,a)$ represents a quantum circuit, regarded as
  Proto-Quipper data. The purpose of the terms $t$ and $a$ is to
  provide structure on the (otherwise unordered) sets of inputs and
  outputs of $C$, so that these inputs and outputs can take the shape
  of Proto-Quipper quantum data.  For example, suppose that $C$ is a
  circuit with inputs $\s{q_1,q_2,q_3}$ and outputs
  $\s{q_4,q_5,q_6}$. Then the term
  \[ (\p{q_2,\p{q_3,q_1}}, C, \p{\p{q_4,q_6},q_5})
  \] 
  represents the circuit $C$, but also specifies what it means to
  apply this circuit to a quantum data term $\p{p,\p{r,s}}$. Namely,
  in this case, the circuit inputs $q_2$, $q_3$, and $q_1$ will be
  applied to qubits $p$, $r$, and $s$, respectively.  Moreover, if the
  output of this circuit is to be matched against the pattern
  $\p{\p{x,y},z}$, then the variables $x$, $y$, and $z$ will be bound,
  respectively, to the quantum bits at endpoints $q_4$, $q_6$, and
  $q_5$.

  Terms of the form $(t,C,a)$ are not intended to be written by the
  user of the programming language; in fact, the Proto-Quipper
  implementation does not provide a concrete syntax for such
  terms. Rather, these terms are internally generated during the {\em
    evaluation} of Proto-Quipper programs. However, the circuits for
  certain basic gates may be made available to the user as pre-defined
  symbols.
\item $\boxx^T$ is a built-in function to turn a circuit-producing
  function (for example, a function of type $T\loli U$) into a
  circuit regarded as data (for example, of type $\Circ(T,U)$). It is
  the equivalent of the {\tt encapsulate\_generic} operator in Quipper.
\item $\unbox$ is a built-in function for turning a circuit regarded
  as data into a circuit-producing function. It is an inverse of
  $\boxx^T$, and corresponds to the {\tt unencapsulate\_generic}
  operator in Quipper.
\item $\rev$ is a built-in function for reversing a low-level
  circuit.
\end{itemize}
Note that the term $\boxx^T$ is parameterized by a type $T$. This
\emph{Church-style} typing of the language is the reason why types
were introduced before terms. Also note that in a term like $(t,C,a)$,
$t$ is assumed to be a quantum data term, but $a$ is not. The type
system to be introduced later will guarantee that even though $a$ is
not yet a quantum data term it will eventually reduce to one.

The \emph{values} are the terms whose execution is finished.

\begin{definition}
The \emph{values} of Proto-Quipper are defined by:
\begin{center}
\begin{tabular}{rrl}
$\val$&$~v,w~~\bnf$ & $x \bor q \bor (t,C,u) \bor \true \bor 
  \false \bor \p{v,w} \bor$ \\
& & $* \bor \lambda x.a  \bor \boxx^T \bor \unbox \bor \rev.$
\end{tabular}
\end{center}
\end{definition}

\noindent
For reference, we summarize some notational conventions that will be
used throughout these notes. Additional such conventions will be
adapted throughout. We write:
\begin{itemize}
  \item $x,y,z,\ldots$ for variables,
  \item $q_1,q_2,q_3,\ldots$ for quantum variables,
  \item $C,D,\ldots$ for circuit constants,
  \item $A,B,\ldots$ for types,
  \item $a,b,c,\ldots$ for terms,
  \item $S,T,U,\ldots$ for quantum data types,
  \item $s,t,u,\ldots$ for quantum data terms and
  \item $v,w,\ldots$ for values.
\end{itemize}

\subsection{Operations on types and terms}

We now introduce some useful syntactic operations on types and terms. 

\begin{definition}
The set of \emph{free (term) variables} of a term $a$, written $\FV(a)$, 
is defined recursively as follows:
\begin{itemize}
  \item $\FV(x)=\s{x}$,
  \item $\FV(\p{a,b})=\FV(a)\cup \FV(b)$,
  \item $\FV(ab)=\FV(a)\cup \FV(b)$,
  \item $\FV(\lambda x.a)=\FV(a)\setminus\s{x}$,
  \item $\FV(\ifthenelse{a}{b}{c}) = \FV(a) \cup \FV(b) \cup \FV(c)$,
  \item $\FV(\letin{\p{x,y}}{a}{b}) = \FV(a)\cup (\FV(b)\setminus \s{x,y})$,
  \item $\FV((t,C,a))= \FV(a)$ and
  \item $\FV(a)=\emptyset$ in all remaining cases.
\end{itemize}
\end{definition}

The above definition of free variables extends the usual one from
lambda calculus. Note that the free variables of a term of the form
$(t,C,a)$ are the free variables of $a$. This is justified since no
variables ever appear in the quantum data term $t$. Variables that are
not free are {\em bound}. The notions of $\alpha$-equivalence,
capture-avoiding substitution, etc., are defined in a straightforward
manner. To wit, two terms are {\em $\alpha$-equivalent} if they differ
only in the naming of bound variables, and we generally identify terms
up to $\alpha$-equivalence.

By analogy with the free term variables of a term, we introduce a
notion of \emph{quantum variable of term}.

\begin{definition}
  The set of \emph{free quantum variables} of a term $a$, written
  $\FQ(a)$, is recursively defined as follows:
\begin{itemize}
  \item $\FQ(q)=\s{q}$,
  \item $\FQ(\p{a,b})=\FQ(a)\cup \FQ(b)$,
  \item $\FQ(ab)=\FQ(a)\cup \FQ(b)$,
  \item $\FQ(\lambda x.a)=\FQ(a)$,
  \item $\FQ(\ifthenelse{a}{b}{c}) = \FQ(a) \cup \FQ(b) \cup \FQ(c)$,
  \item $\FQ(\letin{\p{x,y}}{a}{b})= \FQ(a)\cup \FQ(b)$ and
  \item $\FQ(a)=\emptyset$ in all remaining cases.
\end{itemize}
\end{definition}

Note that $\FQ((t,C,a))=\emptyset$. This reflects the idea that the 
quantum variables appearing in $t$ and $a$ are ``bound'' in $(t,C,a)$. 

To append circuits, we will need to be able to express the way in 
which wires should be connected. For this, we use the notion of a 
\emph{binding}.

\begin{definition}
A \emph{finite bijection} on a set $X$ is a bijection between two 
finite subsets of $X$. We write $\finbij(X)$ for the set of finite 
bijections on $X$.
\end{definition}

\begin{definition}
A \emph{binding} is a finite bijection on $\qset$ and if $\binding$ 
is a binding we write $\dom(\binding)$ for the domain of $\binding$.
\end{definition}

Adding to the list of notational conventions above, we will write 
$\binding$ for an arbitrary binding. 

\begin{definition}
If $a$ is a term, $\binding$ is a binding and $\dom(\binding)\cap
\FQ(a)=\s{q_1,\ldots,q_n}$, then $\binding(a)$ is the following 
term:
\[
\binding(a)= a[\binding(q_1)/q_1,\ldots ,\binding(q_n)/q_n].
\]
\end{definition}

\begin{definition}
The partial function $\bind: \qdataterm^2\to \finbij(\qset)$ is 
defined recursively as follows:
\[
\bind (t,u)= \left\{
  \begin{array}{ll}
    \emptyset & \mbox{if}~~ t=u=*, \\
    \s{(q_1,q_2)} & \mbox{if}~~ t=q_1 \mbox{ and } u=q_2, \\        
    \bind (t_1,u_1) \cup \bind (t_2,u_2) & 
      \mbox{if}~~ t=\p{t_1,t_2} \mbox{ and } u=\p{u_1,u_2}, \\
    \mbox{undefined} & \mbox{in all remaining cases.}
  \end{array}
\right.
\]
\end{definition}

\begin{definition}
Let $T$ be a quantum data type and $X$ a finite subset of $\qset$. 
An \emph{$X$-specimen} for $T$ is quantum data term written $\spec_X(T)$ 
defined by recursion as follows:
\begin{itemize}
  \item $\spec_X(1)=*$,
  \item $\spec_X(\qubit)=q$ where $q$ is the smallest quantum 
  index of $\qset\setminus X$,
  \item $\spec_X(T\tensor U)=\p{t,u}$ where $t=\spec_X(T)$ 
  and $u=\spec_{X\cup \FQ(t)}(U)$.  
\end{itemize}
\end{definition}

% The definition of specimen uses the fact that $\mathcal{Q}$ is 
% well-ordered.

Informally, an $X$-specimen for $T$ is a quantum data term $t$ that is 
``fresh" with respect to the quantum variables appearing in $X$.
If $X$ is clear from the context, we simply write $\spec (T)$.


\subsection{The type system}

The fact that a term is reusable should not prevent us from using it
exactly once. Intuitively, this should imply that if ${!}A$ is a valid
type for a given term, then $A$ should also be a valid type for it.
To capture this idea, we use a sub-typing relation on types.

\begin{definition}
The \emph{sub-typing relation} $<:$ is the smallest relation on 
types satisfying the rules given in 
Figure~\hyperref[subtyping_congruences]{\ref*{subtyping_congruences}}.
\end{definition}

\begin{figure}[!ht]
\begin{mdframed}
\[
  \infer[]{\qubit <: \qubit}{}
~~~~
  \infer[]{1 <: 1}{}
~~~~
  \infer[]{\bool <: \bool}{}
\]
\[
  \infer[]{ (A_1\x A_2) <:  (B_1 \x B_2)}{A_1<:B_1~~A_2<:B_2}
~~~~
  \infer[]{ (A_1\loli B_1) <: (A_2\loli B_2)}{A_2<:A_1~~B_1<:B_2}
\]
\[
  \infer[]{ \Circ(A_1, B_1) <:  \Circ(A_2, B_2)}{A_2<:A_1~~B_1<:B_2}
\]
\[
  \infer[]{ {!}^nA <: {!}^mB}{
    A<:B
    &
    (n=0 \imp m=0)
  }
\]
% A not as nice version of the subtyping rules:
%
%\[
%  \infer[]{{!}^n\qubit <: {!}^m\qubit}{}
%~~~~
%  \infer[]{{!}^n1 <: {!}^m1}{}
%~~~~
%  \infer[]{{!}^n\bool <: {!}^m\bool}{}
%\]
%\[
%  \infer[]{{!}^n (A_1\x A_2) <: {!}^m (B_1 \x B_2)}{A_1<:B_1~~A_2<:B_2}
%~~~~
%  \infer[]{{!}^n (A'\loli B) <: {!}^m (A\loli B')}{A<:A'~~B<:B'}
%\]
%\[
%  \infer[.]{{!}^n \Circ(A', B) <: {!}^m \Circ(A, B')}{A<:A'~~B<:B'}
%\]
\end{mdframed}
\caption{The sub-typing rules of Proto-Quipper.}
\label{subtyping_congruences}
\end{figure}

Note that the subtyping of $A\loli B$ and $\Circ(A,B)$ is {\em
  contravariant} in the left argument, i.e., $A<:A'$ implies $A'\loli
B<:A\loli B$.

\begin{remark}
\label{subtyping_shape}
If $A<:B$ then:
\begin{enumerate}
  \item if $A\in\s{\qubit, 1, \bool}$, then $A=B$;
  \item if $A=A_1\x A_2$, then $B=B_1\x B_2$, 
  $A_1<:B_1$ and $A_2<:B_2$;
  \item if $A=A_1\loli A_2$, then $B=B_1\loli B_2$, 
  $B_1<:A_1$ and $A_2<:B_2$;
  \item if $A=\Circ(A_1, A_2)$, then $B=\Circ(B_1,B_2)$, 
  $B_1<:A_1$ and $A_2<:B_2$;
  \item if $B={!}B'$, then $A={!}A'$ and $A'<:B'$;\label{subtype_bang}
  \item if $A$ is not of the form ${!}A'$, then $B$ is not 
  of the form ${!}B'$.
\end{enumerate}
\end{remark}

\begin{proposition}
The subtyping relation is reflexive and transitive.
\end{proposition}

Note that by taking $n=0$ and $m=1$ in the last subtyping rule, the
following rule is derivable:
\[
  \infer[]{{!}A <: A}{}
  .
\]

\begin{definition}
A \emph{typing context} is a finite set 
$\s{x_1:A_1,\ldots,x_n:A_n}$ of pairs of a variable and 
a type, such that no variable occurs more than once. A 
\emph{quantum context} is a finite set of quantum variables. 
The expressions of the form $x:A$ in a typing context are 
called \emph{type declarations}.	
\end{definition}

We write $\Gamma$ or $\Delta$ for a typing context and $Q$ for 
a quantum context. If $\Gamma =\s{x_1:A_1,\ldots,x_n:A_n}$ is 
a typing context, then $|\Gamma|=\s{x_1,\ldots,x_n}$, 
$\Gamma (x_i)=A_i$ and we write ${!}\Gamma$ if $\Gamma(x_i)={!}A_i'$ 
for every $i$. The union of two contexts is defined when
$|\Gamma|\cap  |\Gamma'|=\emptyset$ and is denoted by $\Gamma,\Gamma'$, 
and similarly for quantum contexts. Finally, we extend the subtyping 
relation to typing contexts as follows: $\Gamma <: \Gamma'$ if and 
only if $|\Gamma | = |\Gamma'|$ and $\Gamma (x_i)<: \Gamma' (x_i)$ 
for every $i$.

\begin{definition}
Let $T,U$ be quantum data types and $m,n\in\N$. For each of the constants 
$\boxx^T$, $\unbox$, and $\rev$, we introduce a type as follows:
\begin{itemize}
  \item $A_{\boxx^T}(T,U)={!}(T\loli U)\loli {!}\Circ(T,U)$,
  \item $A_{\unbox}(T,U)=\Circ(T,U)\loli {!}(T\loli U)$, and
  \item $A_{\rev}(T,U)=\Circ(T,U) \loli {!}\Circ(U,T)$.
\end{itemize}
\end{definition}

\begin{definition}
A \emph{typing judgment} is an expression of the form:
\[
\Gamma ; Q \entails a:A
\] 
where $\Gamma$ is a typing context, $Q$ is a quantum context, 
$a$ is a term and $A$ is a type. A typing judgment is \emph{valid} if 
it can be inferred from the rules given in Figure~\hyperref[trules]{\ref*{trules}}. In the rule $(\rul{cst}_c)$, 
$c$ ranges over the set $\s{\boxx^T, \unbox, \rev}$. Each typing rule
carries an implicit side condition that the judgements appearing in it
are well-formed. In particular, a rule containing a context of the form
$\Gamma_1,\Gamma_2$ may not be applied unless
$|\Gamma_1|\cap|\Gamma_2|=\emptyset$. 
\end{definition}

Note that in the typing judgements of Proto-Quipper, quantum variables
and variables are kept separate. As a result, we do not have to specify
that $q:\qubit$ for every quantum variable $q$ since the typing rules
implicitly enforce this. However, when a future version of
Proto-Quipper will be equipped with the ability to manipulate quantum
\emph{and} classical wires, the type of a wire might have to be
explicitly stated.

% Another reason for this is because in the reduction rule for
% (t,c,u) we need to be able to construct, e.g., t in a context
% that is empty but for the quantum variables appearing in t.

\begin{figure}[!ht]
\begin{mdframed}
\[
\infer[(\rul{ax}_c)]{{!}\Delta, x:A;\emptyset\entails x:B}{
  A<:B
}
~~~~
\infer[(\rul{ax}_q)]{{!}\Delta;\s{q}\entails q:\qubit}{
} 
\]
\[
\infer[(\rul{cst}_c)]{{!}\Delta;\emptyset \entails c:B}{
  {!}A_{c}(T,U)<:B
} 
~~~~
\infer[(\rul{unit})]{{!}\Delta;\emptyset\entails *:{!}^n 1}{
}
\]
\[
\infer[(\lambda_1)]{\Gamma;Q\entails \lambda x.b:A\loli B}{
  \Gamma,x:A;Q \entails b:B
}
~~~~
\infer[(\lambda_2)]{{!}\Delta;\emptyset \entails \lambda x.b:~{!}^{n+1}(A\loli B)}{
  {!}\Delta, x:A;\emptyset \entails b:B
}
\]
\[
\infer[(\rul{app})]{\Gamma_1,\Gamma_2, {!}\Delta;Q_1,Q_2\entails ca:B}{
  \Gamma_1, {!}\Delta;Q_1\entails c:A\loli B 
  &
  \Gamma_2, {!}\Delta ;Q_2\entails a:A 
}
\]
\[
\infer[(\x\mbox{-i})]{\Gamma_1,\Gamma_2, {!}\Delta;Q_1,Q_2\entails \p{a,b}:{!}^n(A\x B)}{
  \Gamma_1, {!}\Delta;Q_1\entails a:{!}^nA 
  &
  \Gamma_2, {!}\Delta ;Q_2\entails b:{!}^nB
}
\]
% Note that in fact the \x intro rule could have equivalently been written with
% a:{!}^nA b:{!}^mB and (a,b):{!}^o(A\x B) where o=min {n,m}
% Indeed the rules are equivalent via the type isomorphism {!!}A~{!}A and the 
% subtyping relation {!}A<:A. The moral here is that a pair is as 
% reusable as its least reusable component.
\[
\infer[(\x\mbox{-e})]{\Gamma_1,\Gamma_2, {!}\Delta;Q_1,Q_2\entails \letin{\p{x,y}}{b}{a}:A}{
  \Gamma_1, {!}\Delta;Q_1\entails b:{!}^n(B_1\x B_2) 
  &
  \Gamma_2, {!}\Delta, x:{!}^nB_1, y:{!}^nB_2 ;Q_2\entails a:A
}
\]
\[
\infer[(\top)]{{!}\Delta;\emptyset\entails \true:{!}^n \bool}{
} 
~~~~
\infer[(\bot)]{{!}\Delta;\emptyset\entails \false:{!}^n \bool}{
}
\]
\[
\infer[(\rul{if})]{\Gamma_1,\Gamma_2, {!}\Delta;Q_1,Q_2\entails \ifthenelse{b}{a_1}{a_2}:A}{
  \Gamma_1, {!}\Delta;Q_1\entails b:\bool 
  &
  \Gamma_2, {!}\Delta;Q_2 \entails a_1:A ~~~ \Gamma_2, {!}\Delta;Q_2 \entails a_2:A
}
\]
\[
\infer[(\rul{circ})]{{!}\Delta;\emptyset \entails (t,C,a):{!}^n\Circ(T,U)}{
  Q_1\entails t:T 
  &
  {!}\Delta ; Q_2\entails a:U 
  &
  \In(C)=Q_1 
  &
  \Out(C)=Q_2
}
\]
\end{mdframed}
\caption{The typing rules of Proto-Quipper.}
\label{trules}
\end{figure}

As a first illustration of the safety properties of the type system,
note that the $(\x\mbox{-i})$ rule ensures that $\lambda x. \p{x,x}$
cannot be given the type $\qubit\loli \qubit \x \qubit$.


\subsection{Circuit constructors}\label{ssec-circuit-cons}

As mentioned in Section 2, the reduction relation for Proto-Quipper is
defined in the presence of a \emph{circuit constructor}. This is a
device capable of performing certain basic circuit building
operations. It is not necessary to have a detailed description of the
inner workings of this device. In fact, all that is required for the
definition of Proto-Quipper's operational semantics is the existence
of some primitive operations. We now axiomatize these
operations. Their intuitive meaning will be explained following
Definition~\ref{circuit_constructor}.

\begin{definition}
\label{circuit_constructor}
A \emph{circuit constructor} consists of a pair of countable sets $\atuple{Q,S}$ 
together with the following maps:
\begin{itemize}
  \item $\mathtt{New}\from \pow_f(Q) \to S$,
  \item $\mathtt{In}\from S\to \pow_f(Q)$,
  \item $\mathtt{Out}\from S\to \pow_f(Q)$,
  \item $\mathtt{Rev}\from S \to S$,
  \item $\mathtt{Append}\from S\times S\times \finbij(Q) \to S \times \finbij(Q)$
\end{itemize}
satisfying the following conditions:
\begin{enumerate}
  \item $\mathtt{Rev}\circ\mathtt{Rev}=1_S$,
  \item $\mathtt{In}\circ\mathtt{Rev}= \mathtt{Out}$ and 
        $\mathtt{Out}\circ\mathtt{Rev}= \mathtt{In}$\label{in_out_rev},
  \item $\mathtt{In}\circ\mathtt{New} = \mathtt{Out}\circ\mathtt{New} =
    1_{\mathcal{P}_f(\mathcal{Q})}$,
  \item if $\mathtt{Append} (C,D,b)=(C',b')$ and 
    $b\from X \to \mathtt{In}(D)$ for some $X\subseteq \mathtt{out}(C)$, 
    then \label{Append_cond_x}
    \begin{enumerate}
      \item $\mathtt{In}(C') = \mathtt{In}(C)$,\label{Append_cond_2b}    
      \item $b'\from \mathtt{Out}(D)\to Y$ for some 
      $Y\subseteq \mathtt{Out}(C')$ and\label{Append_cond_2}
      \item $\mathtt{Out}(C')=Y\cupdot
      (\mathtt{Out}(C)\setminus X)$.\label{Append_cond_3}
    \end{enumerate}
\end{enumerate}
\end{definition}

If $\p{Q,S}$ is a circuit constructor, we call the elements of $S$
\emph{circuit states} and the elements of $Q$ \emph{wire identifiers}.
We now explain the intended meaning of these operations. An element
$C\in S$ is a quantum circuit, such as 
\[ C \quad=\quad \mp{.8}{\Qcircuit @C=1em @R=.7em {
& \lstick{q_1} & \gate{H} & \targ & \rstick{q_3} \qw \\
& \lstick{q_2} & \qw & \ctrl{-1} & \rstick{q_4.}\qw
}}
\]
Each circuit has a finite set of inputs and a finite set of outputs,
given by the functions $\mathtt{In}$ and $\mathtt{Out}$. For example,
$\mathtt{In}(C) = \s{q_1,q_2}$ and $\mathtt{Out}(C)=\s{q_3,q_4}$.
For $X\seq Q$, the circuit $\mathtt{New}(X)$ is the identity circuit
with inputs and outputs $X$; for example,
\[
\mathtt{New}(q_1,q_2,q_3)\quad=\quad~~
\mp{0.5}{\Qcircuit @C=1em @R=1.3em {
&\lstick{q_1}& \qw & \qw & \qw & \rstick{q_1}\qw \\
&\lstick{q_2}& \qw & \qw & \qw & \rstick{q_2}\qw \\
&\lstick{q_3}& \qw & \qw & \qw & \rstick{q_3.}\qw 
}}
\]
The operator $\mathtt{Rev}$ reverses a circuit, swapping its inputs
and outputs in the process. When $(C',b') = \mathtt{Append}(C,D,b)$,
the circuit $C'$ is obtained by appending the circuit $D$ to the end
of the circuit $C$. The function $b$ is used to specify along which
wires to compose $C$ and $D$ while the function $b'$ updates the wire
names post composition.  An illustration of this is given in
Figure~\hyperref[rep_unencap]{\ref*{rep_unencap}}. 

\begin{figure}[!ht]
\[
\mbox{
\Qcircuit @C=.5em @R=1.7em {
& & & & & 
     & & & & & & \ustick{b} & & &
     & & & 
     & & & & & & \ustick{b'} & & & \\
&\qw &\ustick{q_1}\qw &\qw &\qw &\multigate{4}{~~~C~~~} 
     &\qw &\qw &\ustick{q_1}\qw &\qw &\qw &\qw &\qw &\qw &\ustick{q_1'}\qw
     &\qw &\qw &\multigate{2}{~~~D~~~} 
     &\qw &\qw &\ustick{p_1}\qw &\qw &\qw &\qw &\qw &\qw &\ustick{p_1'}\qw \\
&\qw &\ustick{q_2}\qw &\qw &\qw &\ghost{~~~C~~~}       
     &\qw &\qw &\ustick{q_2}\qw &\qw &\qw &\qw &\qw &\qw &\ustick{q_2'}\qw
     &\qw &\qw &\ghost{~~~D~~~} 
     &\qw &\qw &\ustick{p_2}\qw &\qw &\qw &\qw &\qw &\qw &\ustick{p_2'}\qw \\     
&\qw &\ustick{q_3}\qw &\qw &\qw &\ghost{~~~C~~~}        
     &\qw &\qw &\ustick{q_3}\qw &\qw &\qw &\qw &\qw &\qw &\ustick{q_3'}\qw
     &\qw &\qw &\ghost{~~~D~~~} 
     &\qw &\qw &\ustick{p_3}\qw &\qw &\qw &\qw &\qw &\qw &\ustick{p_3'}\qw \\     
&\qw &\ustick{q_4}\qw &\qw &\qw &\ghost{~~~C~~~}       
     &\qw &\qw &\ustick{q_4}\qw &\qw &\qw &\qw &\qw &\qw &\qw
     &\qw &\qw &\qw 
     &\qw &\qw &\qw &\qw &\qw &\qw &\qw &\qw &\ustick{q_4}\qw \\     
&\qw &\ustick{q_5}\qw &\qw &\qw &\ghost{~~~C~~~}     
     &\qw &\qw &\ustick{q_5}\qw &\qw &\qw &\qw &\qw &\qw &\qw
     &\qw &\qw &\qw 
     &\qw &\qw &\qw &\qw &\qw &\qw &\qw &\qw &\ustick{q_5}\qw     
     \gategroup{2}{3}{6}{8}{3.3em}{--}
     \gategroup{2}{16}{4}{20}{3.3em}{--}     
     \gategroup{2}{23}{5}{24}{4.5em}{^\}}     
     \gategroup{2}{11}{5}{13}{4.5em}{^\}}     
}}
\]
\caption{A representation of $\mathtt{Append} (C,D,b)$.}
\label{rep_unencap}
\end{figure}

We note that the axiomatization of
Definition~\ref{circuit_constructor} does not mention the concept of a
{\em gate}. Indeed, any gate is a circuit, and thus a member of the
set $S$; conversely, any circuit can be used as a gate. In
Proto-Quipper, we simply assume that certain members of $S$ are
available as pre-defined constants, serving as ``elementary'' gates.
The operation of appending a gate to a circuit is subsumed 
by the more general operation of composing circuits. 

Proto-Quipper's quantum variables and circuit constants are supposed 
to be the syntactic representatives of a circuit constructor's wire 
identifiers. This idea is formalized in the following definition.

\begin{definition}
A circuit constructor $\atuple{Q,S}$ is \emph{adequate} if it can 
be equipped with bijections $\mathtt{Wire}\from \mathcal{Q} \to Q$ 
and $\mathtt{Name}\from \mathcal{C} \to S$ such that:
\[\In=\mathtt{Wire}' \circ\mathtt{In}\circ \mathtt{Name} 
~\mbox{ and }~ 
\Out=\mathtt{Wire}'\circ\mathtt{Out}\circ \mathtt{Name}
\]
where $\mathtt{Wire}'$ denotes the lifting of $\mathtt{Wire}^{-1}$ 
from $\mathcal{Q}$ to $\mathcal{P}(\mathcal{Q})$.
\end{definition}

\begin{remark}
\label{structure-transfer}
The existence of the bijections $\mathtt{Wire}$ and $\mathtt{Name}$ has 
the following consequences:
\begin{itemize}
  \item $\mathcal{C}$ can be equipped with an involution:
\[
(.)^{-1} = \mathtt{Name}\circ \mathtt{rev} \circ \mathtt{Name}^{-1}
           \from \mathcal{C}\to\mathcal{C}
\]
such that $\In(C^{-1})=\Out(C)$ and 
$\Out(C^{-1})=\In(C)$.
  \item If $t$ and $u$ are quantum data terms such that 
  $\bind(t,u)=\binding$, then we can define 
  $b = \mathtt{Wire}\circ \binding\circ \mathtt{Wire}^{-1}
  \in \finbij(Q)$.
\end{itemize}

From now on, we always assume an adequate circuit constructor. Moreover, we work 
under the simplifying assumptions that $\mathcal{Q}=Q$,
$\mathcal{C}=S$, $\mathtt{Wire}=1_Q$, and $\mathtt{Name}=1_S$.
\end{remark}


\subsection{Operational semantics}

We are now in a position to define Proto-Quipper's operational 
semantics.

\begin{definition}
Let $\atuple{Q,S}$ be an adequate circuit constructor. A 
\emph{closure} is a pair $[C,a]$ where $C\in S$, $a$ is a 
term and $\FQ(t)\subseteq\mathtt{Out}(C)$. 
\end{definition}

% We require $\FQ(t)\subseteq\mathtt{Out}(C)$ and not $\FQ(t)=\mathtt{Out}(C)$
% because the reduction is defined by recursion on the terms. For
% example, to reduce a pair, one must first reduce the left component
% in the context of the current circuit. If we required an equality 
% we could not simply do this.

\begin{definition}
The \emph{one-step reduction relation}, written $\to$, is defined 
on closures by the rules given in Tables~\hyperref[cong_rules]{\ref*{cong_rules}}, 
\hyperref[classical_rules]{\ref*{classical_rules}} 
and \hyperref[circ_gen_rules]{\ref*{circ_gen_rules}}. 
The \emph{reduction relation}, 
written $\to^*$, is defined to be the reflexive and transitive closure of $\to$.
\end{definition}

\begin{figure}[p]
\begin{mdframed}
\[
  \infer[(\rul{fun})]{[C,ab]\to[C',a'b]}{
    [C,a]\to[C',a']
  }
~~~~~~
  \infer[(\rul{arg})]{[C,vb]\to [C',vb']}{
    [C,b]\to [C',b']
  }
\]
\[
  \infer[(\rul{right})]{[C,\langle a,b\rangle]\to [C',\langle a,b'\rangle]}{
    [C,b]\to [C',b']
  }
~~~~~~
  \infer[(\rul{left})]{[C,\langle a,v\rangle]\to [C',\langle a',v\rangle]}{
    [C,a]\to [C',a']
  }
\]
\[
  \infer[(\rul{let})]{[C,\letin{\p{x,y}}{a}{b}]\to 
                [C', \letin{\p{x,y}}{a'}{b}]}{
    [C,a]\to [C',a']
  }
\]
\[
  \infer[(\rul{cond})]{[C, \ifthenelse{a}{b}{c}]\to [C', \ifthenelse{a'}{b}{c}]}{
    [C,a]\to [C',a']
  }
\]
\[
  \infer[(\rul{circ})]{[C, (t,D,a)]\to [C, (t,D',a')]}{
    [D,a]\to [D',a']
  }
\]
\end{mdframed}
\caption{The congruence rules.}
\label{cong_rules}
\end{figure}

\begin{figure}[p]
\begin{mdframed}
\[
  \infer[(\beta)]{[C,(\lambda x.a)v]\to [C, a[v/x]]}{}
\]
\[
  \infer[(\rul{pair})]{[C,\letin{\p{x,y}}{\p{v,w}}{a}]\to[C,a[v/x,w/y]]}{}
\]
\[
  \infer[(\rul{if}\mbox{-}\mathtt{F})]{[C,\ifthenelse{\mathtt{False}}{a}{b}] \to [C, b]}{}
\]
\[
  \infer[(\rul{if}\mbox{-}\mathtt{T})]{[C,\ifthenelse{\mathtt{True}}{a}{b}] \to [C, a]}{}
\]
\end{mdframed}
\caption{The classical rules.}
\label{classical_rules}
\end{figure}

\begin{figure}[p]
\begin{mdframed}
\[
  \infer[(\boxx)]{[C,\boxx^T(v)]\to [C,(t,D,vt)]}{
    \spec_{\FQ(v)}(T)=t
    &
    \mathtt{new}(\FQ(t))=D
  }
\]
\[
  \infer[(\unbox)]{[C,(\unbox\,(u,D,u'))v]\to [C',\binding'(u')]}{
    \bind(v,u)=\binding 
    &
    \mathtt{Append}(C,D,\binding) = (C',\binding') 
  }
\]
\[
  \infer[(\rev)]{[C,\rev\,(t,C,t')]\to [C,(t',C^{-1},t)]}{}
\]
\end{mdframed}
\caption{The circuit generating rules.}
\label{circ_gen_rules}
\end{figure}

The classical reduction rules and the congruence rules (except $(\rul{circ})$) 
are standard. They describe the usual call-by-value reduction strategy. 
The $(\rev)$ rule is straightforward. We briefly discuss the remaining 
rules. 

The $(\boxx)$ rule is to be understood as follows. To reduce a closure
of the form $[C, \boxx^T(v)]$, start by generating a specimen of type
$T$. Then apply the function $v$ on the input $t$ in the context of an
empty circuit of the appropriate arity.  By the $(\rul{circ})$
congruence rule, this computation will continue until a value is
reached, i.e., a term of the form $(t,D,t')$. Note that while this
computation is taking place, the state $C$ is not accessible. When a
value of the form $(t,D,t')$ is reached, the construction of $C$ can
resume. Note that it was necessary to know the type $T$ in order to
generate the appropriate specimen. This explains the choice of a
Church-style typing of the $\boxx$ operator. 

The $(\unbox)$ rule will first generate a binding from $v$ and the
input $u$ of $D$. Then, it will compose $C$ and $D$ along that binding
and update the names of the wire identifiers appearing in $u'$
according to $\binding'$. 

The recursive nature of the reduction rules explains why closures are 
not required to satisfy $\FQ(a)=\mathtt{Out}(C)$. The requirement that 
$\FQ(a)\subseteq \mathtt{Out}(C)$ is justified by the idea that a term 
should not affect a wire outside of $C$. But if we also asked for the 
opposite inclusion, it would not be possible to define a recursive
reduction in a straightforward way. For example, the reduction of a 
pair is done component-wise: to reduce $\p{a,b}$ one first reduces $b$. 
The simplest way to express this in terms of closures is to carry the 
whole circuit state along. This implies that if both $a$ and $b$ 
contain wire identifiers, then the equality $\FQ(a)=\mathtt{Out}(C)$ 
cannot be satisfied.

Proto-Quipper's reduction is non-probabilistic, in the sense that the 
right member of any reduction rule is a unique closure. The following 
proposition establishes that Proto-Quipper's reduction is moreover  
\emph{deterministic}.

\begin{proposition}
\label{determinicity}
If $[C,a]$ is a closure, then at most one reduction rule applies
to it.
\end{proposition}

\begin{proof}
By case distinction on $a$.
\end{proof}


% ----------------------------------------------------------------------
\clearpage
\section{Type safety}\label{sec-type-safety}

In this section, we establish that the core of  Proto-Quipper 
is a \emph{type safe} language. We follow \cite{WrFe94} in 
considering that a language is type safe if it enjoys the 
following two properties: 
\begin{description}
  \item[Subject Reduction:] This property guarantees 
  that the type of a term is stable under reduction.
  As a corollary, it also shows that if a term is 
  well-typed, then it never reduces to an ill-typed 
  term.
  \item[Progress:] This property shows that a 
  well-typed term never reaches an ``error state", 
  which is defined to be a closure $[C,t]$ to which 
  no reduction rule applies but such that 
  $t\notin \mathtt{Val}$. 
\end{description}
Note that the reduction relation is defined on closures 
but the typing rules apply to terms. We therefore extend 
the notions of typing judgement and validity to closures.

\begin{definition}
A \emph{typed closure} is an expression of the form:
\[
\Gamma;Q\entails [C,a]:A,(Q'|Q'').
\]
It is \emph{valid} if $\mathtt{In}(C)=Q'$ and $\mathtt{Out}(C)=Q,Q''$,
and $\Gamma;Q\entails a:A$ is a valid typing judgement.
\end{definition}

\subsection{Properties of the type system}

Before proving the Subject Reduction and Progress properties, we
record some properties of the type system, including the technical but
important \emph{Substitution Lemma}. Note that the typing rules
enforce a \emph{strict} linearity on variables and quantum
variables. In particular, if a quantum variable appears in the quantum
context of a valid typing judgement for a term $a$, then it must
belong to the free quantum variables of $a$.

\begin{lemma}~
\label{prop_type_syst}
\begin{enumerate}
  \item If $\Gamma; Q \entails a:A$ is valid, 
  then $Q=\FQ(a)$.\label{q_context}
  \item If $\Gamma,x:B;Q \entails a:A$ is valid, 
  and $x\notin \FV(a)$, then $B={!}B'$ and 
  $\Gamma;Q \entails a:A$ is valid.\label{unused_var}
  \item If $\Gamma; Q \entails a:A$ is valid, 
  then $\Gamma, {!}\Delta ; Q \entails a:A$ is valid.\label{weakening}
  \item If $\Gamma ; Q \entails a:A$ is valid, $\Delta <: \Gamma$
  and $A<:B$, then $\Delta ; Q \entails a:B$ is valid.\label{subtype}
\end{enumerate}
\end{lemma}

\begin{proof}
By induction on the corresponding typing derivation.
\end{proof}

\begin{lemma}
\label{specimen}
If $T$ is a quantum data type and $X$ is a finite subset of
$\mathcal{Q}$, then $\FQ(t)\entails \spec_X(T):T$ is valid.
\end{lemma}

\begin{proof}
We prove the Lemma by induction on $T$.
  \begin{itemize}
    \item If $T=1$, then $\spec_X(T)=*$ and we can use the $(\rul{unit})$ rule.
    \item If $T=\qubit$, then $\spec_X(T)=q$ for some quantum variable $q$ and we can 
          use the $(\rul{ax}_q)$ rule.
    \item If $T=T_1\x T_2$, then $\spec_X(T)=\p{t_1,t_2}$ where $t_1=\spec_X(T_1)$ 
          and $u=\spec_{X\cup \FQ(t_1)}(T_2)$. By the induction hypothesis, both 
          $\FQ(t_1)\entails t_1:T_1$ and $\FQ(t_2)\entails t_2:T_2$ are valid typing 
          judgements. We can therefore conclude by applying the $(\tensor\mbox{-i})$ rule.
  \end{itemize}
\end{proof}

\begin{lemma}
\label{binding_judgement}
If $\Gamma;Q \entails a:A$ is valid and $\binding$ is a 
binding such that $FQ(a)\subseteq\dom(\binding)$ 
then $\Gamma;\binding (Q) \entails \binding(a):A$ is valid.
\end{lemma}

\begin{proof}
By induction on the typing derivation of $\Gamma;Q \entails a:A$.
\end{proof}

\begin{lemma}
\label{context_value}
If $v\in\mathtt{Val}$ and $\Gamma;Q \entails v:{!}A$ is valid, 
then $Q=\emptyset$ and $\Gamma={!}\Delta$ for some $\Delta$.
\end{lemma}

\begin{proof}
By induction on the typing derivation of $\Gamma;Q \entails v:{!}A$. 
In the case of $(\rul{ax}_c)$, use Lemma~\hyperref[subtype_bang]{\ref*{subtyping_shape}(\ref*{subtype_bang})}.
\end{proof}

\begin{lemma}
\label{non_values}
If a term $a$ is not a value then it is of one 
of the following forms: 
\begin{compactitemize}
\item $(t,C,a')$ with $a'\notin \mathtt{Val}$, 
\item $\p{a_1,a_2}$ with $a_1\notin \mathtt{Val}$ or $a_2\notin
  \mathtt{Val}$, 
\item $\ifthenelse{a_1}{a_2}{a_3}$,
\item $\letin{\p{x,y}}{a_1}{a_2}$, or
\item $a_1a_2$.
\end{compactitemize}
\end{lemma}

\begin{proof}
Trivial, by the definition of terms and values.
\end{proof}

\begin{lemma}
\label{form_values}
A well-typed value $v$ is either a variable, a quantum variable, a constant or one 
of the following case occurs: 
\begin{compactitemize}
\item if it is of type ${!}^n\Circ(T,U)$, it is of the 
form $(t,C,u)$ with $t$ and $u$ values; 
\item if it is of type ${!}^n\bool$ it is either 
$\true$ or $\false$; 
\item if it is of type ${!}^n (A\x B)$, it is of the
form $\p{w,w'}$, with $w$ and $w'$ values; 
\item if it is of type ${!}^n1$, it is 
precisely the term $*$; 
\item if it is of type ${!}^n (A\loli B)$, it is a 
lambda-abstraction or a constant.
\end{compactitemize}
\end{lemma}

\begin{proof}
By induction on the typing derivation of $v$.
\end{proof}

\begin{corollary}
\label{typed_qd_term}
If $T$ is a quantum data type and $v$ is a well-typed value of type $T$ then $v$ is 
a quantum data term.
\end{corollary}

\begin{lemma}
{\bf (Substitution).}
\label{substitution}
If $v\in\mathtt{Val}$ and both $\Gamma',{!}\Delta;Q' \entails v:B$ and 
$\Gamma,{!}\Delta,x:B;Q \entails a:A$ are valid typing judgements, 
then $\Gamma,\Gamma',{!}\Delta;Q,Q' \entails a[v/x]:A$ is also valid.
\end{lemma}

\begin{proof}
Let $\pi_1$ and $\pi_2$ be the typing derivations  of 
$\Gamma,{!}\Delta,x:B;Q \entails a:A$ and  $\Gamma',{!}\Delta;Q' \entails v:B$ 
respectively. We prove the Lemma by induction on $\pi_1$.
\begin{itemize}
 \item If the last rule of $\pi_1$ is $(\rul{ax}_c)$ and $a=x$, then $\pi_1$ is
 \[
   \infer[(\rul{ax}_c)]{{!}\Delta, x:B;\emptyset\entails x:A}{
     B<:A
   }
 \]
 with $\Gamma=Q=\emptyset$. Then $a[v/x]=v$ and can conclude by applying 
 Lemma~\hyperref[subtype]{\ref*{prop_type_syst}.\ref*{subtype}} to $\pi_2$.
 \item If the last rule of $\pi_1$ is $(\rul{ax}_c)$ and $a=y\neq x$, then 
 $\pi_1$ is
 \[
   \infer[(\rul{ax}_c)]{{!}\Delta, x:{!}B',y:A';\emptyset\entails y:A}{
     A'<:A
   }
 \]
 with $B={!}B'$, $Q=\emptyset$ and $\Gamma =\s{y:A'}$ or $\Gamma=\emptyset$ 
 depending on whether or not $A'$ is duplicable. Therefore $v$ is a value of 
 type ${!}B'$ and by Lemma~\hyperref[context_value]{\ref*{context_value}}, 
 we know that $\Gamma'=Q'=\emptyset$. Since $a[v/x]=y$ and $x\notin \FV(y)$ 
 we can conclude by applying
 Lemma~\hyperref[unused_var]{\ref*{prop_type_syst}.\ref*{unused_var}} to $\pi_1$.
 \item If the last rule of $\pi_1$ is one of $(\rul{ax}_q)$, $(\rul{cst}_c)$, 
 $(\rul{unit})$, $(\top)$ and $(\bot)$, and $a$ is the corresponding constant, 
 then $x\notin \FV(a)$ and  $x$ must be declared of some type ${!}B'$.
 We can therefore reason as in the previous case. 
 \item If the last rule of $\pi_1$ is $(\lambda_1)$ and $a=\lambda y.b$, then $\pi_1$ is
  \[
   \infer[(\lambda_1)]{\Gamma,{!}\Delta,x:B;Q\entails \lambda y.b:A_1\loli A_2}{
     \deduce[]{\Gamma, {!}\Delta,x:B,y:A_1;Q \entails b:A_2}{
       \vdots
     }
   }
 \]
 with $A=A_1\loli A_2$. By the induction hypothesis, 
 $\Gamma, \Gamma',{!}\Delta,y:A_1;Q,Q' \entails b[v/x]:A_2$ is valid and we can conclude
 by applying $(\lambda_1)$.
 \item If the last rule of $\pi_1$ is $(\lambda_2)$ and $a=\lambda y.b$, 
 then $\pi_1$ is
 \[
  \infer[(\lambda_2)]{{!}\Delta, x:{!}B';\emptyset \entails \lambda y.b:~{!}^{n+1}(A_1\loli A_2)}{
    \deduce[]{{!}\Delta, x:{!}B',y:A_1;\emptyset \entails b:A_2}{
      \vdots
    }
  }
 \]
 with $A={!}^{n+1}(A_1\loli A_2)$ and $B={!}B'$. Hence $v$ is a value of type ${!}B'$ and  
 by Lemma~\hyperref[context_value]{\ref*{context_value}}, we know that
 $\Gamma'=Q'=\emptyset$. The induction hypothesis therefore implies that
 ${!}\Delta,y:A_1;\emptyset \entails b[v/x]:A_2$ is valid and we can conclude
 by applying $(\lambda_2)$.
 \item If the last rule of $\pi_1$ is $(\rul{app})$, and $a=ca'$, then $\pi_1$ can 
 be of one of three forms depending on $B$. If $B$ is duplicable, then $\pi_1$ is
 \[
 \infer[(\rul{app})]{\Gamma_1,\Gamma_2,{!}\Delta,x:{!}B';Q_1,Q_2\entails ca':A}{
    \deduce[]{\Gamma_1, x:{!}B',{!}\Delta;Q_1\entails c:A'\loli A}{
      \vdots
    }
    &
    \deduce[]{\Gamma_2, x:{!}B',{!}\Delta ;Q_2\entails a':A' }{
      \vdots
    }
 }
 \]
 with $B={!}B'$. Using 
 Lemma~\hyperref[context_value]{\ref*{context_value}} again, we 
 know that $\Gamma'=Q'=\emptyset$. The induction hypothesis therefore 
 implies that $\Gamma_1,{!}\Delta;Q_1\entails c[v/x]:A'\loli A$ and 
 $\Gamma_2,{!}\Delta ;Q_2\entails a'[v/x]:A'$ are valid and we can conclude 
 by applying $(\rul{app})$. If, instead, $B$ is non-duplicable, then the declaration 
 $x:B$ can only appear in one branch of the derivation. This means that $\pi_1$ 
 is either 
  \[
 \infer[(\rul{app})]{\Gamma_1,\Gamma_2,{!}\Delta,x:B;Q_1,Q_2\entails ca':A}{
    \deduce[]{\Gamma_1, x:B,{!}\Delta;Q_1\entails c:A'\loli A}{
      \vdots
    }
    &
    \deduce[]{\Gamma_2, {!}\Delta ;Q_2\entails a':A'}{
      \vdots
    }     
 }
 \]
 or
  \[
 \infer[(\rul{app}).]{\Gamma_1,\Gamma_2,{!}\Delta,x:B;Q_1,Q_2\entails ca':A}{
    \deduce[]{\Gamma_1, {!}\Delta;Q_1\entails c:A'\loli A}{ 
      \vdots
    }
    &
    \deduce[]{\Gamma_2, x:B,{!}\Delta ;Q_2\entails a':A'}{
      \vdots
    }     
 }
 \]
 In the first case, the induction hypothesis implies that 
 $\Gamma_1,\Gamma' {!}\Delta;Q_1,Q'\entails c[v/x]:A'\loli A$ is valid and 
 we can conclude by $(\rul{app})$. The second case is treated analogously.
 \item If the last rule of $\pi_1$ is on of $(\x\mbox{-i})$, $(\x\mbox{-e})$ 
 and $(\rul{if})$, and $a$ is the corresponding term, then we can reason as above
 by considering in turn the case where $B$ is duplicable and the case where
 $B$ is non-duplicable.
 \item If the last rule of $\pi_1$ is $(\rul{circ})$, and $a=(t,C,a')$, then 
 $\pi_1$ is
 \[
 \infer[(\rul{circ})]{{!}\Delta,x:{!}B';\emptyset \entails (t,\mathbf{c},a'):{!}^n\Circ(T,U)}{
    \deduce[]{Q_1\entails t:T}{
      \vdots
    }    
    &
    \deduce[]{{!}\Delta,x:{!}B' ; Q_2\entails a':U}{
      \vdots
    }     
    &
    \In(C)=Q_1 
    &
    \Out(C)=Q_2
 }
 \]
 with $A={!}^n\Circ(T,U)$ and $B={!}B'$ for some types $T$, $U$ and $B'$. Using  
 Lemma~\hyperref[context_value]{\ref*{context_value}} again, we know 
 that $\Gamma'=Q'=\emptyset$. The induction hypothesis therefore implies that
 ${!}\Gamma; Q_2\entails a'[v/x]:U$ is valid and we can conclude
 by applying $(\rul{circ})$.
\end{itemize}
\end{proof}


\subsection{Subject reduction}

\begin{lemma}
\label{Inwires}
If $[C,a]\to[C',a']$ then $\mathtt{In}(C)=\mathtt{In}(C')$.
\end{lemma}

\begin{proof}
By induction on the derivation of $[C,a]\to[C',a']$. 
In all but the $(\unbox)$ case, the result follows either from the induction 
hypothesis or from the fact that $C=C'$. In the $(\unbox)$ case, use 
Definition~\hyperref[Append_cond_2b]{\ref*{circuit_constructor}(\ref*{Append_cond_2b})}.
\end{proof}

\begin{theorem}
{\bf (Subject Reduction).}
\label{thm-subject-red}
If $\Gamma;\FQ(a)\entails [C,a]:A,(Q'|Q'')$ is a valid typed closure 
and $[C,a]\to [C',a']$, then $\Gamma;\FQ(a')\entails [C',a']:A,(Q'|Q'')$ is 
a valid typed closure.
\end{theorem}

\begin{proof}
We prove the theorem by induction on the derivation of the reduction 
 $[C,a]\to[C',a']$. In each case, we start by reconstructing 
the unique typing derivation $\pi$ of $\Gamma;\FQ(a)\entails a:A$ and we use 
it to prove that $\Gamma;\FQ(a')\entails [C',a']:A,(Q'|Q'')$ is valid. 
By Lemma~\hyperref[Inwires]{\ref*{Inwires}} we never need to 
verify that $\mathtt{In}(C')=Q'$ so that we only need to show:
\begin{itemize}
  \item $\mathtt{Out}(C')=\FQ(a'),Q''$ and
  \item $\Gamma;\FQ(a')\entails a':A$ is valid.
\end{itemize}
Throughout the proof, we write $IH(\pi)$ to denote the proof obtained by applying 
the induction hypothesis to $\pi$.

\begin{description}
\item[Congruence rules:] These rules are treated uniformly. We illustrate the 
$(\rul{fun})$ and $(\rul{circ})$ cases.
\begin{itemize}
  \item $(\rul{fun})$: the reduction rule is
  \[
    \infer[]{[C,cb]\to[C',c'b]}{
      [C,c]\to[C',c']
    }
  \]
  with $a=cb$ and $a'=c'b$. The typing derivation $\pi$ is therefore
  \[
    \infer[]{\Gamma_1,\Gamma_2, {!}\Delta;\FQ(c),\FQ(b)\entails cb:A}{
      \deduce[]{\Gamma_1, {!}\Delta;\FQ(c)\entails c:B\loli A}{
        \vdots~\pi_1
      } 
      &
      \deduce[]{\Gamma_2, {!}\Delta ;\FQ(b)\entails b:B}{
        \vdots~\pi_2
      } 
    }
  \]
  and $\Gamma_1,\Gamma_2;\FQ(c),\FQ(b)\entails [C, cb],(Q'|Q'')$ is valid.
  It follows that $\Gamma_1, {!}\Delta;\FQ(c)\entails [C,c]:B\loli A,(Q'|\FQ(b),Q'')$ 
  is valid and, by the induction hypothesis, this implies that 
  $\Gamma_1, {!}\Delta;\FQ(c')\entails [C',c']:B\loli A,(Q'|\FQ(b),Q'')$ is also valid.
  In particular, it follows that $\mathtt{Out}(C')=\FQ(c'),\FQ(b),Q''$. This, 
  together with the following typing derivation,
  \[
    \infer[]{\Gamma_1,\Gamma_2, {!}\Delta;\FQ(c'),\FQ(b)\entails c'b:A}{
      \deduce[]{\Gamma_1, {!}\Delta;\FQ(c')\entails c':B\loli A}{
        \vdots~IH(\pi_1)
      } 
      &
      \deduce[]{\Gamma_2, {!}\Delta ;\FQ(b)\entails b:B}{
        \vdots~\pi_2
      } 
    }
  \]  
  shows that $\Gamma_1,\Gamma_2, {!}\Delta;\FQ(c'),\FQ(b)\entails [C',c'b]:A,(Q',Q'')$ 
  is valid.
  \item $(\rul{circ})$: the reduction rule is
  \[
    \infer[(\rul{circ})]{[C, (t,D,b)]\to [C, (t,D',b')]}{
      [D,b]\to [D',b']
    }
  \]  
  with $a=(t,D,b)$ and $a'=(t,D',b')$. The typing derivation $\pi$ is therefore
  \[
  \infer[]{{!}\Delta;\emptyset\entails (t,D,b):{!}^n\Circ(T,U)}{
    \deduce[]{\FQ(t)\entails t:T}{
      \vdots~\pi_1
    } 
    &
    \deduce[]{{!}\Delta ; \FQ(b)\entails b:U}{
      \vdots~\pi_2
    }
    &
    \deduce[]{\In(D)=\FQ(t)}{
      \Out(D)=\FQ(b)
    }
  }
  \]  
  and ${!}\Delta ; \emptyset \entails [C,(t,D,b)]:{!}^n\Circ(T,U),(Q'|Q'')$ is valid. 
  Disregarding $\pi_1$, it follows from the assumptions in the above rule 
  that ${!}\Delta ; \FQ(b)\entails [D,b]:U, (\FQ(t)|\emptyset)$ is valid and, 
  by the induction hypothesis, this implies that 
  ${!}\Delta,\FQ(b')\entails [D',b']:U,(\FQ(t)|\emptyset)$ is also valid.
  This, together with the following typing derivation,
  \[
  \infer[.]{{!}\Delta;\emptyset\entails (t,D,b'):{!}^n\Circ(T,U)}{
    \deduce[]{\FQ(t)\entails t:T}{
      \vdots~\pi_1
    } 
    &
    \deduce[]{{!}\Delta ; \FQ(b')\entails b':U}{
      \vdots~IH(\pi_2)
    }
    &
    \deduce[]{\In(D')=\FQ(t)}{
      \Out(D')=\FQ(b')
    }
  }  
  \]
  shows that ${!}\Delta;\emptyset\entails [C,(t,D',b')] :{!}^n\Circ(T,U),(Q'|Q'')$ 
  is valid.
\end{itemize}
\item[Classical rules:] These rules are also treated uniformly, we illustrate 
the $(\beta)$ case.
\begin{itemize}
  \item $(\beta)$: the reduction rule is
  \[
    \infer[(\beta)]{[C,(\lambda x.b)v]\to [C, b[v/x]]}{}
  \]  
  with $a=(\lambda x.b)v$ and $a'=b[v/x]$. The typing derivation $\pi$ is 
  therefore
  \[
    \infer[]{\Gamma_1,\Gamma_2,{!}\Delta;\FQ(b),\FQ(v)\entails (\lambda x.b)v:A}{
      \infer[]{\Gamma_1,{!}\Delta;\FQ(b)\entails \lambda x.b:B\loli A}{
        \deduce[]{\Gamma_1,{!}\Delta,x:B;\FQ(b) \entails b:A}{
          \vdots~\pi_1
        }
      }
      &
      \deduce[]{\Gamma_2,{!}\Delta;\FQ(v)\entails v:B}{
        \vdots~\pi_2
      }      
    }
  \]  
  and $\Gamma_1,\Gamma_2,{!}\Delta;\FQ(b),\FQ(v)\entails [C,(\lambda x.b)v]:A,(Q'|Q'')$ 
  is valid. By Lemma \hyperref[substitution]{\ref*{substitution}}, we know that
  $\Gamma_1,\Gamma_2,{!}\Delta;\FQ(b),\FQ(v)\entails b[v/x]:A$ is a valid typing 
  judgement which implies that 
  \[ \Gamma_1,\Gamma_2,{!}\Delta;\FQ(b),\FQ(v)\entails
  [C,b[v/x]]:A,(Q'|Q'')\]
  is a valid
  typed closure.
\end{itemize}
\item[Circuit generating rules:] These rules represent the most interesting cases. 
We treat them individually.
\begin{itemize}
  \item $(\boxx)$: the reduction rule is
  \[
  \infer[]{[C,\boxx^T(v)]\to [C,(t,D,vt)]}{
    \spec(T)=t
    &
    \mathtt{new}(\FQ(t))=D
  }
  \]
  with $a=\boxx^T(v)$ and $a'=(t,D,vt)$. Since $v$ is a value, we know by  
  Lemma~\hyperref[context_value]{\ref*{context_value}} that the typing 
  derivation $\pi$ is
  \[
  \infer[]{{!}\Delta;\emptyset\entails \boxx^T(v):{!}^n\Circ(T,U)}{
    \infer[]{{!}\Delta;\emptyset \entails \boxx^T:{!}(T\loli U)\loli {!}^n\Circ(T,U)}{
    }   
    &
    \deduce[]{{!}\Delta ;\emptyset\entails v:{!}(T\loli U)}{
     \vdots ~\pi_1
    }
  }
  \]
  and ${!}\Delta;\emptyset\entails [C,\boxx^T(v)]:{!}^n\Circ(T,U),(Q'|Q'')$ is valid.
  By Lemma \hyperref[specimen]{\ref*{specimen}}, there exists 
  a typing derivation $\pi_2$ of $\FQ(t)\entails t:T$ and applying 
  Lemma~\hyperref[subtype]{\ref*{prop_type_syst}.\ref*{subtype}}
  to $\pi_1$ we get a derivation $\pi_1'$ of ${!}\Delta ;\emptyset\entails v:T\loli U$.
  We can therefore construct the following derivation $\tau$:
  \[
  \infer[]{{!}\Delta;\FQ(t)\entails vt : U}{
    \deduce[]{{!}\Delta ;\emptyset\entails v:T\loli U}{
      \vdots ~\pi_1'
    }
    &
    \deduce[]{{!}\Delta;\FQ(t)\entails t:T}{
      \vdots ~\pi_2'
    }
  }
  \]
  where $\pi_2'$ is obtained from $\pi_2$ by Lemma 
  \hyperref[weakening]{\ref*{prop_type_syst}.\ref*{weakening}}.
  Moreover, since $\FQ(vt)=\FQ(t)=\mathtt{Out}(D)=\mathtt{In}(D)$, 
  we have:
  \[
  \infer[.]{{!}\Delta;\emptyset \entails (t,D,vt):{!}^n\Circ(T,U)}{
    \deduce[]{\FQ(t)\entails t:T}{
      \vdots~\pi_2
    }
    &
    \deduce[]{{!}\Delta ; \FQ(vt)\entails vt:U}{
      \vdots~\tau
    } 
    &
    \deduce[]{\In(D)=\FQ(t) }{
      \Out(D)=\FQ(vt)
    }
  }
  \]  
  Hence ${!}\Delta ;\emptyset \entails [C,(t,D,vt)]:{!}^n\Circ(T,U) (Q'|Q'')$ is 
  a valid typed closure.
  \item $(\unbox)$ the reduction rule is
  \[
    \infer[(\unbox)]{[C,(\unbox\,(t,D,t'))v]\to [C',\binding'(t')]}{
      \bind(v,t)=\binding 
      &
      \mathtt{Append}(C,D,\binding) = (C',\binding') 
    }
  \]
  with $a=(\unbox\,(t,D,t'))v$ and $a'=\binding'(t')$. 
  To reconstruct the typing derivation $\pi$, first note 
  that we have the following derivation $\pi_1$ of    
  ${!}\Delta;\emptyset\entails \unbox\,(t,D,t'):T\loli U$
  \begin{footnotesize}
  \[
    \infer[]{{!}\Delta;\emptyset\entails \unbox\,(t,D,t'):T\loli U}{
      \infer[]{{!}\Delta;\emptyset \entails \unbox:\Circ(T,U)\loli (T\loli U)}{
      }   
      &
      \infer[]{{!}\Delta ;\emptyset\entails (t,D,t'):\Circ(T,U)}{
        \deduce[]{\FQ(t)\entails t:T}{
          \vdots ~\pi_1^1
        }
        &
        \deduce[]{{!}\Delta;\FQ(t')\entails t':U}{
          \vdots ~\pi_1^2     
        }
      }
    }
    \]
    \end{footnotesize} 
  with $\In(D)=\FQ(t)$, $\Out(D)=\FQ(t')$. We can then use $\pi_1$ 
  to rebuild $\pi$ as follows:
  \begin{footnotesize}
  \[
  \infer[]{{!}\Delta; \FQ(v)\entails (\unbox\,(t,D,t'))v :U}{
    \infer[]{{!}\Delta;\emptyset\entails \unbox\,(t,D,t'):T\loli U}{
      \vdots~\pi_1            
    }
    &
    \infer[]{{!}\Delta ; \FQ(v)\entails v:T}{
      \vdots~\pi_2
    }
  }
  \]
  \end{footnotesize} 
  and the typed closure
  \[
  {!}\Delta; \FQ(v) \entails [C,(\unbox\,(t,D,t'))v] :U,(Q'|Q'')
  \] 
  is valid. In the conclusion of $\pi_2$, all the 
  term variables are declared of a duplicable type. This 
  follows from Corollary \ref{typed_qd_term} and 
  Lemma~\hyperref[unused_var]{\ref*{prop_type_syst}.\ref*{unused_var}}. 
  By Definition~\hyperref[Append_cond_2]{\ref*{circuit_constructor}
  (\ref*{Append_cond_2})}, we know that $\FQ(t')\subseteq\dom(\binding')$. 
  We can therefore apply Lemma~\hyperref[binding_judgement]
  {\ref*{binding_judgement}} to $\pi_1^2$ to get a typing 
  derivation $\tau$ of 
  \[
  {!}\Delta;\FQ(\binding(t'))\entails \binding(t'):U.
  \]
  Now by 
  Definition~\hyperref[Append_cond_3]{\ref*{circuit_constructor}.\ref*{Append_cond_3}} 
  we have:
  \[
  \begin{array}{rcl}
  \mathtt{Out}(C') & = & \binding(\mathtt{Out}(D)), (\mathtt{Out}(C)\setminus\binding^{-1}(\mathtt{In}(D))) \\
                   & = & \binding(\FQ(t')) , ((Q'',\FQ(v))\setminus \binding^{-1}(\FQ(t))) \\
                   & = & \FQ(\binding(t')) , ((Q'',\FQ(v))\setminus \FQ(v)) \\
                   & = & \FQ(\binding(t')),Q''.                   
  \end{array}
  \]
  Hence ${!}\Delta; \FQ(\binding(t'))\entails [C',\binding(t')] :U,(Q'|Q'')$  is valid.
  \item $(\rev)$ the reduction rule is
  \[
    \infer[(\rev)]{[C,\rev\,(t,D,t')]\to [C,(t',D^{-1},t)]}{}
  \]
  with $a=\rev\,(t,D,t')$ and $a'=(t',D^{-1},t)$. The typing derivation $\pi$ 
  is therefore
  \begin{footnotesize}
  \[
  \infer[]{{!}\Delta;\emptyset\entails \rev\,(t,D,t'):{!}^n\Circ(U,T)}{
    \infer[]{{!}\Delta;\emptyset \entails \rev:\Circ(T,U)\loli {!}^n\Circ(U,T)}{
    }   
    &
    \infer[]{{!}\Delta ;\emptyset\entails (t,D,t'):\Circ(T,U)}{
      \deduce[]{\FQ(t)\entails t:T}{
        \vdots ~\pi_1
      }
      &
      \deduce[]{{!}\Delta;\FQ(t')\entails t':U}{
        \vdots ~\pi_2     
      }
    }
  }
  \]
  \end{footnotesize}  
  with $\In(D)=\FQ(t)$, $\Out(D)=\FQ(t')$ and 
  ${!}\Delta;\emptyset\entails \rev\,(t,D,t'):{!}^n\Circ(T,U),(Q'|Q'')$ is 
  valid. Now note that since $t'$ is a quantum data term, it contains no variables. 
  Applying Lemma~\hyperref[unused_var]{\ref*{prop_type_syst}.\ref*{unused_var}} 
  to $\pi_2$ repeatedly we therefore get a derivation $\pi_2'$ of 
  $\FQ(t')\entails t':U$. Moreover, by applying
  Lemma~\hyperref[weakening]{\ref*{prop_type_syst}.\ref*{weakening}} to $\pi_1$ 
  we get a typing derivation $\pi_1'$ of ${!}\Delta ,\FQ(t)\entails t:T$.
  Since, by 
  Remark~\hyperref[structure-transfer]{\ref*{structure-transfer}}, 
  we have $\Out(D^{-1})=\In(D)=t$ and $\In(D^{-1})=\Out(D)=t'$, we can construct 
  the following typing derivation:
  \[
  \infer[]{{!}\Delta ;\emptyset\entails (t',D^{-1},t):\Circ(U,T)}{
    \deduce[]{\FQ(t')\entails t':U}{
      \vdots ~\pi_2'
    }
    &
    \deduce[]{{!}\Delta;\FQ(t)\entails t:T}{
      \vdots ~\pi_1'     
    }
    &
    \deduce[]{\In(D^{-1})=\FQ(t')}{
      \Out(D^{-1})=\FQ(t')
    }
  }  
  \]
  Hence ${!}\Delta;\emptyset\entails (t',D,t):{!}^n\Circ(T,U),(Q'|Q'')$ 
  is valid.    
\end{itemize}
\end{description}
\end{proof}

\begin{corollary}
If $\Gamma;\FQ(a)\entails [C,a]:A,(Q'|Q'')$ is a valid typed closure 
and $[C,a]\to^* [C',a']$, then $\Gamma;\FQ(a')\entails [C',a']:A,(Q'|Q'')$ 
is also a valid typed closure.
\end{corollary}

\begin{proof}
By induction on the length of the reduction sequence. The base case is 
provided by Theorem~\hyperref[thm-subject-red]{\ref*{thm-subject-red}}
\end{proof}

The above formulation of Subject Reduction explains why a typed closure contains 
information about the input and output wires of the circuit state. Indeed, 
Subject Reduction now guarantees (1) that the input wires of a circuit 
remain unchanged through reduction and (2) that a term can only affect 
wires whose identifiers are among its quantum variables.

\subsection{Progress}

\begin{theorem}
{\bf (Progress).}
If $\FQ(a)\entails [C,a]:A,(Q'|Q'')$ is a valid typed closure then either 
$a\in\mathtt{Val}$ or there exists a closure $[C',a']$ such that 
$[C,a]\to [C',a']$.
\end{theorem}

First, note that the Progress property is stated for a typed closure
whose typing context is empty. This is because the property is not
expected to hold if we allow for a non-empty typing context. Indeed,
it is easy to see that there are well-typed, non-closed closures such
as $[C,xy]$, which are neither values nor reduce. 
We now prove the theorem.

\begin{proof}
We prove the theorem by induction on the typing derivation $\pi$ of 
$\FQ(a)\entails a:A$. If $a$ is a value then there is nothing 
to prove. If $a$ is not a value, then by 
Lemma~\hyperref[non_values]{\ref*{non_values}} there are 5 cases to consider. 
In each case we show that $[C,a]$ is reducible in the sense that there exists a 
closure  $[C,b]$ such that $[C,a]\to[C,b]$
\begin{enumerate}
  \item If $a=(t,D,a')$ with $a'\notin \mathtt{Val}$, then the typing derivation 
  $\pi$ is:
  \[
  \infer[.]{\emptyset\entails (t,D,a'):\Circ(T,U)}{
    \deduce[]{\FQ(t)\entails t:T}{
      \vdots ~\pi_1
    }
    &
    \deduce[]{\FQ(a')\entails a':U}{
      \vdots ~\pi_2     
    }
    &
    \deduce[]{\In(D)=\FQ(t)}{
      \Out(D)=\FQ(a')
    }
  }   
  \]
  The typed closure 
  \[
  \begin{array}{rcl}
  \FQ(a') & \entails & [D,a']:U,(\FQ(t)|\emptyset)
  \end{array}
  \]
  is therefore valid. Since $a'$ 
  is not a value, the induction hypothesis implies that there exists $a''$ such 
  that $[D,a']\to [D',a'']$ and $[C,(t,D,a')]$ therefore reduces to 
  $[C,(t,D',a'')]$ by the $(\rul{circ})$ reduction rule.
  \item If $a=\p{a_1,a_2}$ with $a_1\notin \mathtt{Val}$ or $a_2\notin \mathtt{Val}$,
  then the typing derivation $\pi$ is:
  \[
  \infer[.]{\FQ(a_1),\FQ(a_2)\entails \p{a_1,a_2}:{!}^n(A_1\x A_2)}{
    \deduce[]{\FQ(a_1)\entails a_1:{!}^nA_1}{
      \vdots~\pi_1
    }
    & 
    \deduce[]{\FQ(a_2)\entails a_2:{!}^nA_2}{
      \vdots~\pi_2
    }
  }
  \] 
  The typed closures 
  \[
  \begin{array}{rcl}
  \FQ(a_1) & \entails & [C,a_1]:{!}^nA_1,(Q'|\FQ(a_2),Q'')\\
  \FQ(a_2) & \entails & [C,a_2]:{!}^nA_1, (Q'|\FQ(a_1),Q'')
  \end{array}
  \]
  are therefore both valid. 
  Now if $a_2\notin\mathtt{Val}$, then by the induction hypothesis 
  $[C,a_2]\to[C',a_2']$. Hence $[C,\p{a_1,a_2}]$ reduces to $[C',\p{a_1,a_2'}]$ 
  by the $(\rul{right})$ reduction rule.
  If on the other hand $a_2\in\mathtt{Val}$, then it must be the case that $a_1\notin\mathtt{Val}$ 
  and we can conclude by reasoning analogously that $[C,\p{a_1,a_2}]$ reduces to 
  some $[C',\p{a_1',a_2}]$ by the $(\rul{left})$ reduction rule.. 
  \item If $a=\ifthenelse{a_1}{a_2}{a_3}$, then the typing derivation $\pi$ is:
  \[
  \infer[.]{\FQ(a_1),Q\entails \ifthenelse{a_1}{a_2}{a_3}:A}{
    \deduce[]{\FQ(a_1)\entails a_1:\bool }{
      \vdots~\pi_1
    }
    &
    \deduce[]{Q \entails a_2:A}{
      \vdots~\pi_2
    }    
    &
    \deduce[]{Q \entails a_3:A}{
      \vdots~\pi_3
    }    
  }
  \]  
  The typed closure
  \[
  \begin{array}{rcl}
  \FQ(a_1) & \entails & [C,a_1]:\bool,(Q'|\FQ(a_2),\FQ(a_3),Q'')
  \end{array}
  \]
  is therefore valid. Now if $a_1\notin\mathtt{Val}$, then by the induction hypothesis 
  $[C,a_1]\to[C',a_1']$ and thus $[C,\ifthenelse{a_1}{a_2}{a_3}]$ can
  be seen to reduce
  to $[C',\ifthenelse{a_1'}{a_2}{a_3}]$ by the $(\rul{cond})$ reduction rule. 
  If on the other hand $a_1\in\mathtt{Val}$, then by 
  Lemma~\hyperref[form_values]{\ref*{form_values}} either $a_1=\true$ or $a_1=\false$. 
  Thus $[C,\ifthenelse{a_1}{a_2}{a_3}]$ reduces either to $[C,a_2]$ by the 
  $(\rul{if}\mbox{-}\mathtt{T})$ reduction rule or 
  to $[C,a_3]$ by the $(\rul{if}\mbox{-}\mathtt{F})$ reduction rule.
  \item If $a=\letin{\p{x,y}}{a_1}{a_2}$, then we can reason as above to show that 
  if $a_1$ is not a value, then the $(\rul{let})$ congruence rule applies and that if 
  $a$ is a value then Lemma~\hyperref[form_values]{\ref*{form_values}} 
  guarantees that the $(\rul{pair})$ rule applies.
  \item If $a=a_1a_2$ then the typing derivation $\pi$ is: 
  \[
    \infer[.]{\FQ(a_1),\FQ(a_2)\entails a_1a_2:A}{
      \deduce[]{\FQ(a_1)\entails a_1:B\loli A}{
        \vdots~\pi_1
      }
      &
      \deduce[]{\FQ(a_2)\entails a_2:B}{
        \vdots~\pi_2
      }      
    }
  \]  
  The typed closures
  \[
  \begin{array}{rcl}
  \FQ(a_1) & \entails & [C,a_1]:B\loli A,(Q'|\FQ(a_2),Q'')\\
  \FQ(a_2) & \entails & [C,a_2]:B, (Q'|\FQ(a_1),Q'')
  \end{array}
  \]
  are therefore valid. There are three cases to treat.
  \begin{itemize}
    \item If $a_1\notin\val$, then $[C,a_1a_2]\to[C',a_1'a_2]$
    by the induction hypothesis and the $(\rul{fun})$ rule.
    \item If $a_1\in\val$ and $a_2\notin\mathtt{Val}$, then 
    $[C,a_1a_2]\to[C',a_1a_2']$ by the induction hypothesis and 
    the $(\rul{arg})$ rule.
    \item If $a_1,a_2\in\val$ then by 
    Lemma~\hyperref[form_values]{\ref*{form_values}}, $a_1$ is either 
    an abstraction or a constant. In each case $[C,a_1a_2]$ reduces 
    by the appropriate rule among $(\beta)$, $(\boxx)$, $(\unbox)$ 
    and $(\rev)$. 
  \end{itemize}
\end{enumerate}
\end{proof}

% ----------------------------------------------------------------------
\clearpage
\section{Type inference}\label{sec-inference}

In this section, we describe the type inference algorithm 
for Proto-Quipper. We detail the algorithm and prove its 
correctness for a fragment of the language and explain, 
at the end of the section, how to extend it.


\subsection{The $\QP'$ Language}

The grammar of core Proto-Quipper is rather extensive, so in the interest of readability, we formalize the type inference algorithm on a smaller fragment of the language. In Section~\ref{ssec-extending-to-QP}, we will explain how to extend type inference to all of core Proto-Quipper. To avoid confusing, we will denote core Proto-Quipper by $\QP$, and the smaller fragment by $\QP'$.

We will introduce the concepts of type inference that are specific to
linear type inference. However, we will presuppose a certain amount of
familiarity with type inference in general.  For a more comprehensive
introduction to type inference, including a discussion of the
significance of type variables and the different understandings of
type inference, please refer to Chapter 22 of Pierce {\cite{pierce02}}.

\begin{defn} The \emph{terms} of $\QP'$ are defined by:
	\begin{center}
	\begin{tabular}{rrl}
		$\texttt{Term}$&$ ~t,u~ ::=$ & $x ~|~ * ~|~ \lambda x.t ~|~ t \, u ~|~ \pair{t}{u} ~|~ \letin{\pair{x}{y}}{t}{u}. $
	\end{tabular}
	\end{center}
\end{defn}

For the purpose of type inference, it would seem natural to define the type system of $\QP'$ by extending the types of $\QP$ with type variables. This would give the following grammar for types:
	\begin{center}
	\begin{tabular}{rrl}
		$\texttt{Type}$&$ ~A,B~ ::=$ & $\alpha ~|~ 1 ~|~ A \multimap B ~|~ A \otimes B ~|~ \bang A$.
	\end{tabular}
	\end{center}
This would have been sufficient if the types were not linear. However, we need to be able to separately discuss the \textit{form} of a type (for example $\qubit$, $\bool \otimes \bool$) and its \textit{linearity} (whether the type is duplicable or not). We therefore introduce a new grammar which, though less intuitive, offers more control over the form of a type.

\begin{defn} The \emph{types} of $\QP'$ are defined by:
	\begin{center}
	\begin{tabular}{rrl}
		$\texttt{Flag}$&$~i~ ::=$ & $n ~|~ 0 ~|~ 1$ \\		
		$\texttt{LinearType}$&$ ~A~ ::=$ & $\alpha ~|~ 1 ~|~ T \multimap U ~|~ T \otimes U$ \\
		$\texttt{Type}$&$ ~T,U~ ::=$ & $!^i A$.
	\end{tabular}
	\end{center}
\end{defn}

Flags, linear types and types are to be interpreted as follows:
\begin{itemize}
	\item $\texttt{LinearType}$ corresponds to the subset of $\QP$ types that are always linear at the outermost level.
	\item $\texttt{Type}$ is set of all types. Depending of the value of its prefixing flag, a $\texttt{Type}$ is duplicable or not.
		In practice, the type of a term will always be represented by a \texttt{Type}, never by a \texttt{LinearType}.
	\item $\texttt{Flag}$ consists of \textit{flag variables} and the constants $0$ and $1$. They parameterize the linearity of a type.
		If the value of $n$ is $1$, then $!^n A$ is duplicable; if it is $0$, then $!^n A$ is linear. This idea is that $!^1A=!A$ and $!^0A=A$. For example, the type
		of any quantum bit will be $!^0 \qubit$.
\end{itemize}
Using a grammar that distinguishes linear types and types also allows us to unambiguously define substitutions as mappings from type variables to \textit{linear} types. This distinction was impossible to make in $\QP$. We will use the following notation:
\begin{itemize}
	\item $\alpha, \beta \dots$ for type variables, taken in the set $\vset$.
	\item $n, m \dots$ for flag variables, taken in the set $\fset$.
	\item $A, B \dots$ for linear types.
	\item $T, U \dots$ for types.
\end{itemize}

\begin{defn} The subtyping relation $<:$ is the smallest relation on types and linear types satisfying the rules given in Figure \ref{subtypingQP'}.
\end{defn}

\begin{figure}[!ht]
\begin{mdframed}
	$$ $$
	$$ \mbox{
  	\AxiomC{}
  	\RightLabel{$(var)$}
	 	\UnaryInfC{$\alpha <: ~ \alpha$}
	 	\DisplayProof
		~~
		~~
		\AxiomC{}
	 	\RightLabel{$(1)$}
	 	\UnaryInfC{$1 <: 1$}
		\DisplayProof
		~~
		~~
	 	\AxiomC{$A <: B$}
	 	\AxiomC{$m \le n$}
	 	\RightLabel{$(!)$}
	 	\BinaryInfC{$!^n A <: ~ !^m B$}
		\DisplayProof
	} $$
	$$ $$
	$$ \mbox{
	 	\AxiomC{$T' <: T$}
	 	\AxiomC{$U <: U'$}
	 	\RightLabel{$(\multimap)$}
	 	\BinaryInfC{$T \multimap U <: T' \multimap U'$}
		\DisplayProof
		~~
		~~
	 	\AxiomC{$T <: T'$}
	 	\AxiomC{$U <: U'$}
	 	\RightLabel{$(\otimes)$}
	 	\BinaryInfC{$T \otimes U <: T' \otimes U'$}
		\DisplayProof
	} $$
	$$ $$
\end{mdframed}
\caption{The sub-typing relation rules of $\QP'$.}
\label{subtypingQP'}
\end{figure}

\begin{proposition} \it The subtyping relation is reflexive and transitive.
\end{proposition}

The notion of a typing context, and the extension of the subtyping relation to typing contexts, are straightforwardly adapted from $\QP$. We introduce the following new notations for the typing contexts of $\QP'$. We write $\Gamma|_t$ to represent the context $\Gamma$ where only the bindings $x : T$ with $x \in FV(t)$ appear. Similarly, the notation $\Gamma \backslash_t$ designates the context $\Gamma$ where the bindings $x : T$ for $x \in FV(t)$ have been removed. Finally, we write $!^I \Gamma$ to represent contexts of the form $\Gamma = x_1 : \nbang{i_1} T_1 ~ \dots ~ x_n : \nbang{i_n} T_n$, with $I = \{i_1 \dots i_n\}$.

The typing rules have to be adapted to the new grammar of types.

\begin{defn}
A \emph{typing judgement} is an expression of the form:
\[
\Gamma \vdash t:T
\] 
where $\Gamma$ is a typing context, $t$ is a term and $T$ is a type. A typing judgment is \emph{valid} if it can be inferred from the rules given in Figure \ref{typingQP'}.
\end{defn}

As in $\QP$, a typing judgement should be interpreted as stating that under the assumptions $\Gamma$, the term $t$ has the type $T$. 

\begin{remark} For the most part, the typing rules of $\QP'$ are those of $\QP$ adapted to the different syntax of types. This is not quite the 
	case of the rules for $\otimes$. Recall that the typing rule for $\pair{a}{b}$ in $\QP$ was (omitting the quantum contexts):
		\begin{prooftree}
			\AxiomC{$\Gamma_1, !\Delta \vdash a : \,!^nA$}
			\AxiomC{$\Gamma_2, !\Delta \vdash b : \,!^nB$}
			\RightLabel{$(\otimes.I)$}
			\BinaryInfC{$\Gamma_1, \Gamma_2, !\Delta \vdash \pair{a}{b} : \,!^n(A \otimes B)$}
		\end{prooftree}
	This rule expresses the fact that the pair $\pair{a}{b}$ is duplicable if and only if both $a$ and $b$ are duplicable. However, the grammar of 
	$\QP'$ types doesn't allow us to move the $!$ annotation around. This is why the rule $(\otimes.I)$ of $\QP'$ uses the type 
	$!^p(!^nA \otimes \,!^mB)$ with the condition that $p$ can be $1$ if and only if both $n$ and $m$ are equal to one. This results in 
	the duplicable pair $\pair{a}{b}$ having the type $!^1(!^1 A \otimes \,!^1B)$ rather than $\bang (A \otimes B)$. A similar remark applies 
	to the $(\otimes.E)$ rule.
\end{remark}

\begin{exmp} The term $\lambda x.\pair{x}{x}$ can be typed in the empty context with the type 
	$!^1(!^1 A \multimap \,!^0(!^1A \otimes !^1A)$. The typing derivation is
		\begin{prooftree}
			\AxiomC{$!^1A <: \,!^1A$}
			\RightLabel{$(ax)$}
			\UnaryInfC{$x : \,!^1A \vdash x : \,!^1A$}
			\AxiomC{$!^1A <: \,!^1A$}
			\RightLabel{$(ax)$}
			\UnaryInfC{$x : \,!^1A \vdash x : \,!^1A$}
			\RightLabel{$(\otimes.I)$}
			\BinaryInfC{$x : \,!^1A \vdash \pair{x}{x} : \,!^0(!^1A \otimes !^1A)$}
			\RightLabel{$(\lambda_2)$}
			\UnaryInfC{$\varnothing \vdash \lambda x.\pair{x}{x} : \,!^1(!^1 A \multimap !^1(!^1A \otimes !^1A))$}
		\end{prooftree}
\end{exmp}

\begin{figure}[!ht]
\begin{mdframed}
	$$ $$
	$$ \mbox{
		\AxiomC{$T <: U$}
		\RightLabel{$(ax)$}
		\UnaryInfC{$!^1 \Delta, x : T \vdash x : U$}
		\DisplayProof
		~~
		~~
		~~
		\AxiomC{}
		\RightLabel{$(1)$}
		\UnaryInfC{$!^1 \Delta \vdash * : \,!^n 1$}
		\DisplayProof
	} $$
	$$ $$
	$$ \mbox{
		\AxiomC{$\Gamma, x : T \vdash t : U$}
		\RightLabel{$(\lambda_1)$}
		\UnaryInfC{$\Gamma \vdash \lambda x.t : \,!^0 (T \multimap U)$}
		\DisplayProof
		~~
		~~
		\AxiomC{$!^1 \Gamma, x : T \vdash t : U$}
		\RightLabel{$(\lambda_2)$}
		\UnaryInfC{$!^1 \Gamma \vdash \lambda x.t : \,!^1 (T \multimap U)$}
		\DisplayProof
	} $$
	$$ $$
	$$ \mbox{
		\AxiomC{$\Gamma_1, !^1\Delta \vdash t : \,!^0(T \multimap U)$}
		\AxiomC{$\Gamma_2, !^1\Delta \vdash u : T$}
		\RightLabel{$(app)$}
		\BinaryInfC{$\Gamma_1, \Gamma_2, !^1\Delta \vdash t u : U$}
		\DisplayProof
	} $$
	$$ $$
	$$ \mbox{
		\AxiomC{$\Gamma_1, !^1\Delta \vdash t : \, !^nA$}
		\noLine
		\UnaryInfC{$\Gamma_2, !^1\Delta \vdash u : \, !^mB$}
		\AxiomC{$p \le n, ~ p \le m$}
		\RightLabel{$(\otimes.I)$}
		\BinaryInfC{$\Gamma_1, \Gamma_2, !^1\Delta \vdash \, \pair{t}{u} \, : \,!^p(!^nA \otimes \,!^mB)$}
		\DisplayProof
	} $$
	$$ $$
	$$ \mbox{
		\AxiomC{$\Gamma_1, \,!^1\Delta \vdash t : \, !^p (!^nA \otimes \,!^mB)$}
		\noLine
		\UnaryInfC{$\Gamma_2, \,!^1\Delta, x : \,!^nA, y : \,!^nB \vdash u : T$}
		\AxiomC{$p \le n, ~ p \le m$}
		\RightLabel{$(\otimes.E)$}
		\BinaryInfC{$\Gamma_1, \Gamma_2, !^1\Delta \vdash \, \text{let} \pair{t}{u} = t ~ \text{in} ~ u \, : T$}
		\DisplayProof
	} $$
	$$ $$
\end{mdframed}
\caption{The typing rules of $\QP'$.}
\label{typingQP'}
\end{figure}

The correspondence between the type systems for $\QP$ and $\QP'$ is established in the following theorem:

\begin{thm}{\bf (Soundness and completeness).} Let $t$ be a $\QP'$ term. There exists two transformations $\tau$ and
	$\bar \tau$ defined on types as functions respectively from $\QP$ to $\QP'$ types, and $\QP'$ to $\QP$ types, satisfying the properties:
	\begin{enumerate}
		\item If there exist a context $\Gamma$ and a type $T$ such that the typing judgement $\Gamma \vdash t : T$ is valid in $\QP$, then
			$\tau\Gamma \vdash t : \tau T$ is valid in $\QP'$.
		\item If there exist a context $\Gamma$ and a type $T$ such that the typing judgement $\Gamma \vdash t : T$ is valid in $\QP'$, then
			$\bar \tau \Gamma \vdash t : \bar \tau T$ is valid in $\QP$.
	\end{enumerate}
	
	\begin{proofsketch}
		The transformations $\tau$ and $\bar \tau$ can be written as recursive functions on types. Their definition will not be explained
		here. The proof is a simple induction on the typing derivation of $\Gamma \vdash t : T$. The cases of $(\otimes.I)$ and
		$(\otimes.E)$ use the isomorphism $\bang\bang A \equiv \bang A$ of $\QP$-types.
	\end{proofsketch}
\end{thm}

\begin{remark}
	This theorem is necessary to ensure the validity of the type inference algorithm: by proving that
	terms can be typed in either $\QP$ or $\QP'$ with similar judgements, we ensure that:
	\begin{itemize}
		\item If the algorithm infers a typing judgement valid in $\QP'$, then it is possible to derive a typing judgement valid in $\QP$.
		\item Conversely, if the inference fails to produce such a judgement, then $t$ is surely not typeable in $\QP$.
	\end{itemize}
\end{remark}

\begin{lemma} The type system satisfies the following properties:
	\label{QP'prop}
	\begin{enumerate}
		\item Suppose $\Gamma \vdash t : T$ is a valid typing judgement in $\QP'$. Let $\Delta$ and $U$ be such that
			$\Delta <: \Gamma$ and $T <: U$. Then $\Delta \vdash t : U$ is valid.
	
		\item Suppose $\Gamma \vdash t : T$ is a valid typing judgement in $\QP'$. Then for any context $!^1 \Delta$ distinct from $\Gamma$;
			$\Gamma, \,!^1\Delta \vdash t : T$ is valid as well.
	
		\item Suppose $\Gamma, x : \,!^1 T \vdash t : T$ with $x \notin FV(t)$ is a valid typing judgement in $\QP'$.
			Then $\Gamma \vdash t : T$ is valid as well.
	\end{enumerate}
	
	\begin{proof}
		The properties are proved by induction on the typing derivation of $t$.
	\end{proof}
\end{lemma}

\begin{remark}
Note that the weaker statement
``\textit{If $\Gamma \vdash t : T$ and $T <: U$, then $\Gamma \vdash t : U$}'' cannot be proved directly by induction, because the case for
$\lambda x.t$ requires a relation of sub-typing on the context, and thus requires the stronger induction hypothesis.
\end{remark}

\begin{coro} The following typing rules are admissible in $\QP'$ :
	\begin{prooftree}
		\AxiomC{$\Gamma \vdash t : T$}
		\AxiomC{$T <: U$}
		\RightLabel{Sub-typing}
		\BinaryInfC{$\Gamma \vdash t : U$}
	\end{prooftree}
	
	\begin{prooftree}
		\AxiomC{$\Gamma \vdash t : T$}
		\AxiomC{$x \notin |\Gamma|$}
		\RightLabel{Weakening}
		\BinaryInfC{$\Gamma, x : \,!^1A \vdash t : T$}
	\end{prooftree}
	
	\begin{prooftree}
		\AxiomC{$\Gamma, x : \,!^1A \vdash t : T$}
		\AxiomC{$x \notin FV(t)$}
		\RightLabel{Refining}
		\BinaryInfC{$\Gamma \vdash t : T$}
	\end{prooftree}
\end{coro}

\subsection{Constraint Typing}

Unless otherwise stated, $\QP'$ is the default language for the rest of this section.

\begin{defn} A \textit{substitution} is a finite list of mappings $(\alpha \mapsto A)$ or $(n \mapsto \s{0, 1})$
	from type variables to \textit{linear} types and from flag variables to either $0$ or $1$.
	We write $dom(\sigma)$ for the set of variables appearing on the left-hand sides of the mappings, and $range(\sigma)$ for the set of variables
	appearing on the right-hand sides of pairs in $\sigma$. We also use $vDom$, $vRange$, $fDom$ and $fRange$ for the subsets of $dom$ and $range$
	that contain respectively the type variables and the flag variables.
	The application of a substitution $\sigma$ to types and linear types is defined in the obvious way:
	$$
	\begin{array}{lcl}
		\sigma (\alpha) &=& \left\{ \begin{array}{l}
		                              A \text{ if } (\alpha \mapsto A) \in \sigma \\
      		                        \alpha \text{ if not}
		                            \end{array} \right. \\
		\sigma(1) &=& 1 \\
		\sigma(T \otimes U) &=& \sigma T \otimes \sigma U \\
		\sigma(T \multimap U) &=& \sigma T \multimap \sigma U \\
		\sigma(!^n A) &=& \left\{ \begin{array}{l}
																	!^{0} \sigma A \text{ if } (n \mapsto 0) \in \sigma \\
																	!^{1} \sigma A \text{ if } (n \mapsto 1) \in \sigma \\
																	!^n \sigma A \text{ otherwise}
																\end{array} \right. \\
	\end{array}
	$$
\end{defn}	

Note that variables may appear in both $dom(\sigma)$ and $range(\sigma)$. The intention in such cases is for the mappings of $\sigma$ to
be applied simultaneously. The definition of $\sigma$ ensures that the kind of a type is left untouched: a linear type is mapped to a linear
type, and a type is mapped to a type. The substitution $\sigma$ is extended to contexts by defining
 		$$\sigma (x_1 : T_1 \dots x_n : T_n) = (x_1 : \sigma T_1 \dots x_n : \sigma T_n)$$
 		
\begin{defn} If $\sigma$ and $\tau$ are substitutions, we write $\sigma \circ \tau$ for the composition of $\sigma$ with $\tau$, defined as follows
		$$\tau \circ \sigma = \left[ \begin{array}{ll}
																			\alpha \mapsto \tau A & \text{for each } (\alpha \mapsto A) \in \sigma \\
																			\alpha \mapsto A & \text{for each } (\alpha \mapsto A) \in \tau \text{ with } \alpha \notin vDom(\sigma) \\
																			n \mapsto \sigma(n) & \text{for each } n \in fDom(\sigma) \\
																			n \mapsto \tau(n) & \text{for each } n \in fDom(\tau) \text{ with } n \notin fDom(\sigma) \\
																		\end{array} \right]$$
\end{defn}

The composition is associative: $\sigma \circ (\tau \circ \rho) = (\sigma \circ \tau) \circ \rho$. We say that a substitution
$\sigma$ is \textit{idempotent} if it satisfies $\sigma \circ \sigma = \sigma$. It can be proved in particular that $\sigma \neq Id$ is idempotent if $dom(\sigma) \cap range(\sigma) = \varnothing$. Because working with substitutions $\sigma$ such that
$\sigma \neq \sigma \circ \sigma$ can be really cumbersome, we consider in the following that all substitutions are chosen idempotent.

Substitutions preserve the validity of typing judgements. If a term $t$ is well-typed with the
type $T$ in the context $\Gamma$, then for any substitution $\sigma$, $\sigma \Gamma \vdash t : \sigma T$ is valid.

\begin{thm}
	\label{subs-judgement}
	Suppose a valid typing judgement $\Gamma \vdash t : T$. Then for every substitution $\sigma$, $\sigma \Gamma \vdash t : \sigma T$
	is valid.
\end{thm}

We now formalize the problem of type inference. 
Suppose $\Gamma$ a typing context, $t$ a term and $T$ a type, not necessarily satisfying $\Gamma \vdash t : T$. The purpose of type inference is to answer the question:
 \begin{center}
 	 ``Is \text{some} substitution instance of $t$ well-typed?"
 \end{center}
That is, can we find $\sigma$ such that $\sigma \Gamma \vdash t : \sigma T$ is well typed? Looking for valid instantiations of the
variables of $\Gamma, T$ leads to the problem of type inference. The set of all such substitutions is taken as the set of solutions
for the following problem:

\begin{defn}
	\label{decl_problem}
	Suppose $\Gamma$ a typing context, $t$ a term and $T$ a type. A substitution $\sigma$ is a solution for the problem $(\Gamma, t, T)$ if
	$\sigma \Gamma \vdash t : \sigma T$ is a valid typing judgement.
\end{defn}

\begin{exmp} Let $f$ be $\lambda x.x$ of type $!^n \alpha$, $a$ of type $!^m \beta$ and $t = f \, a$. Then the following 	are solutions for $(\varnothing, t, !^p \gamma)$:
  \[
  \begin{array}{l}\relax
    [{!}^n\alpha \mapsto {!}^1({!}^0\intx \multimap {!}^1\intx),~~ {!}^m\beta \mapsto {!}^1\intx,~~ {!}^p\gamma \mapsto {!}^1\intx], \\\relax
    [{!}^n\alpha \mapsto {!}^0({!}^0\qubit \multimap {!}^0\bit),~~ {!}^m\beta \mapsto {!}^0\qubit,~~ {!}^p\gamma \mapsto {!}^0\qubit], \\\relax
    [{!}^n\alpha \mapsto {!}^0({!}^1\intx \multimap {!}^0({!}^1\intx \otimes {!}^1\intx)),
    ~~ {!}^m\beta \mapsto {!}^1\intx,
    ~~ {!}^p\gamma \mapsto {!}^0({!}^1\intx \otimes {!}^1\intx)].
  \end{array}
  \]
\end{exmp}

\subsubsection{The Constraint Typing Relation}

\begin{defn} We introduce {\em sub-typing constraints} $X \prec: Y$ for $X, Y \in \texttt{Type}$ or $X, Y \in \texttt{LinearType}$.
We say that a constraint $X \prec: Y$ is:
		\begin{center}
		\begin{tabular}{l}
			\textit{atomic} if it is of the form $\alpha \prec: \beta$. \\
			\textit{semi-composite} if it is of the form $\alpha \prec: A$ or $A \prec: \alpha$ with $A \notin \vset$. \\
			\textit{composite} if it is of the form $X \prec: Y$ with $X, Y \notin \vset$.
		\end{tabular}
		\end{center}
  Similarly, flag constraints $n \le m$ are defined over flag variables and the constants $0$ and $1$. Flag constraints are atomic if of the form $n \le m$,
  for $m, n\in \fset$.
  A substitution $\sigma$ solves a constraint $T \prec: U$ ($n \le m$) if and only if $\sigma T <: \sigma U$ ($\sigma(n) \le \sigma(m)$).
\end{defn}

Though all flag constraints are written in the same form $(. \le .)$, they can hold different meanings, for example
	\begin{center}
	\begin{tabular}{ll}
		$n \le 0$ & can be viewed as $n = 0$, \\
 		$1 \le m$ & is the constraint $m = 1$, \\
 		$n \le m$ & can be interpreted as $n = 1 \Rightarrow m = 1$.
 	\end{tabular}
 	\end{center}

\begin{defn} A {\em constraint set} $\lset$ is a set of constraints $\{ A \prec: B \dots, T \prec: U \dots, n \le m \dots\}$
	over linear types, types and flags.
  A constraint set can have the following properties:
  \begin{itemize}
  	\item[]{\bf Atomicity} A set is atomic when all the constraints contained are atomic.
  	\item[]{\bf Satisfiability} A set is \textit{solved} when all the included constraints can be solved by application of one or
  		more of the sub-typing rules (type constraints) or by transitivity (flag constraints).
  		Thus, a set $\lset$ is \textit{solvable} if there exists a substitution $\sigma$ such that $\sigma \lset$ is solved.
	 	\item[]{\bf Derivability} When the validity of a set $\mathcal{L}$ can be deduced from another set $\mathcal{L'}$ by application of
		  the sub-typing rules and the transitivity, we say that $\lset$ is \textit{derivable} from $\lset'$,
		  written $\lset' \vdash \lset$.
  \end{itemize}
\end{defn}

\begin{defn} The \textit{constraint typing relation} $\Gamma \vdash_\mathcal{L} t : T ~|_\chi$ is defined by the constraint typing rules
	of Figure \ref{ctypeQP'}. It it to be interpreted as: \textit{Under the assumptions $\Gamma$, when the constraints $\lset$ are solved,
	the term $t$ has the type $T$}. 
\end{defn}

The set $\chi$ keeps track of the type and flag variables created during the construction of the derivation
of the typing judgement $\Gamma \vdash_\mathcal{L} t : T ~|_\chi$. 

\begin{algorithm}{\bf (ConstraintTyping).}
The constraint typing algorithm is obtained by reading the
constraint typing rules from bottom to top. It can be verified that the algorithm is indeed deterministic, whereas the typing
rules of $\QP'$ were not (see the rules $(\lambda_1)$ and $(\lambda_2)$). 
The function ConstraintTyping applies the constraint
typing algorithm to a typing judgement $\Gamma \vdash t : T$, and either fails, or returns a set $\mathcal{L}$ (and as a by-product the set of
variables $\chi$) such that the relation $\Gamma \vdash_\mathcal{L} t : T ~|_\chi$ holds.
\end{algorithm}	

\begin{figure}[!ht]
\begin{mdframed}
	$$ $$
	$$ \mbox{
		\AxiomC{$\mathcal{L} = \{1 \le I, ~ T \prec: U\}$}
		\RightLabel{$(ax)$}
		\UnaryInfC{$!^I \Delta, x : T \vdash_\mathcal{L} x : U ~|_\varnothing$}
		\DisplayProof
		~~
		~~
		\AxiomC{$\mathcal{L} = \{1 \le I, ~ \,!^1 1 \prec: T \}$}
		\RightLabel{$(\top)$}
		\UnaryInfC{$!^I \Delta \vdash_\mathcal{L} * : T ~|_\varnothing$}
		\DisplayProof
	} $$
	$$ $$
	$$ \mbox{
		\AxiomC{$!^I\Gamma, x : \,!^n\alpha \vdash_\mathcal{L'} t : \,!^m\beta ~|_\chi$} \noLine
		\UnaryInfC{$\alpha, \beta, n, m, p \notin \Gamma, T$} \noLine
		\UnaryInfC{$\mathcal{L} = \mathcal{L'} \cup \{ p \le I, ~ !^p(!^n\alpha \multimap \,!^m\beta) \prec: T \}$}
		\RightLabel{$(\lambda)$}
		\UnaryInfC{$\Gamma \vdash_\mathcal{L} \lambda x.t : T ~|_{\chi \cup \{\alpha, \beta, n, m, p \}}$}
		\DisplayProof
	} $$
	$$ $$
	$$ \mbox{
		\AxiomC{$\Gamma|_u \vdash_{\mathcal{L}_2} u : \,!^n\alpha ~|_{\chi_2}$} \noLine
		\UnaryInfC{$\Gamma|_t \vdash_{\mathcal{L}_1} t : \,!^0(!^n\alpha \multimap T) ~|_{\chi_1}$} \noLine
		\AxiomC{$\chi_1 \cap \chi_2 = \varnothing$} \noLine
    \UnaryInfC{$\alpha, n \notin \Gamma, T$} \noLine
    \UnaryInfC{$\Gamma \backslash_{t \oplus u} = \,!^I \Delta$} \noLine
    \UnaryInfC{$\mathcal{L} = \mathcal{L}_1 \cup \mathcal{L}_2 \cup \{ 1 \le I \}$}
		\RightLabel{$(app)$}
		\BinaryInfC{$\Gamma \vdash_\mathcal{L} t \, u : T ~|_{\chi_1 \cup \chi_2 \cup \{ \alpha, n\}}$}
		\DisplayProof
	} $$
	$$ $$	
	$$ \mbox{
		\AxiomC{$\Gamma|_t \vdash_{\mathcal{L}_1} t : \, !^n\alpha ~|_{\chi_1}$} \noLine
		\UnaryInfC{$\Gamma|_u \vdash_{\mathcal{L}_2} u : \, !^m\beta ~|_{\chi_2}$}
		\AxiomC{$\chi_1 \cap \chi_2 = \varnothing$} \noLine
		\AxiomC{$\Gamma \backslash_{t \oplus u} = \,!^I \Delta$} \noLine
		\BinaryInfC{$\alpha, \beta, n, m, p \notin \Gamma, T$} \noLine
		\BinaryInfC{$\mathcal{L} = \mathcal{L}_1 \cup \mathcal{L}_2 \cup \{ 1 \le I, p \le n, p \le m ~!^p(!^n\alpha \otimes \,!^m\beta) \prec: T \}$}
		\RightLabel{$(\otimes.I)$}
		\UnaryInfC{$\Gamma \vdash_\mathcal{L} \, \pair{t}{u} \, : T ~|_{\chi_1 \cup \chi_2 \cup \{\alpha, \beta, n, m, p \}}$}
		\DisplayProof
	} $$
	$$ $$
	$$ \mbox{
		\AxiomC{$\Gamma|_t \vdash_{\mathcal{L}_1} t : \, !^p (!^n\alpha \otimes \,!^m\beta) ~|_{\chi_1}$} \noLine
		\UnaryInfC{$\Gamma|_u, x : \,!^n\alpha, y : \,!^m\beta \vdash_{\mathcal{L}_2} u : T ~|_{\chi_2}$}
		\AxiomC{$\chi_1 \cap \chi_2 = \varnothing$} \noLine
		\UnaryInfC{$\alpha, \beta, n, m, p \notin \Gamma, T$} \noLine
		\UnaryInfC{$\Gamma \backslash_{t \oplus u} = \,!^I \Delta$} \noLine
		\BinaryInfC{$\mathcal{L} = \mathcal{L}_1 \cup \mathcal{L}_2 \cup \{ p \le n, p \le m, 1 \le I \}$}
		\RightLabel{$(\otimes.E)$}
		\UnaryInfC{$\Gamma \vdash_\mathcal{L} \, \text{let} \pair{x}{y} = t ~ \text{in} ~ u \, : T ~|_{\chi_1 \cup \chi_2 \cup \{\alpha,\beta,m,n,p \}}$}
		\DisplayProof
	} $$
	$$ $$
\end{mdframed}
\caption{Constraint typing rules.}
\label{ctypeQP'}
\end{figure}

\begin{exmp} The same example term $\lambda x.\pair{x}{x}$ can be typed using the constraint typing rules, with the derivation
	\begin{prooftree}
			\AxiomC{}
			\RightLabel{$(ax)$}
			\UnaryInfC{$x : \,!^{n_1}\beta \vdash_{\{!^{n_1}\beta \prec: \,!^{n_5}\epsilon\}} x : \,!^{n_5}\epsilon$}
			\AxiomC{}
			\RightLabel{$(ax)$}
			\UnaryInfC{$x : \,!^{n_1}\beta \vdash_{\{!^{n_1}\beta \prec: \,!^{n_4}\delta\}} x : \,!^{n_4}\delta$}
			\RightLabel{$(\otimes.I)$}
			\BinaryInfC{$x : \,!^{n_1}\beta \vdash_{\left\{ \scriptsize
			                                          \begin{array}{lll} 
			                                            !^{n_1}\beta \prec: \,!^{n_4}\delta, & n_6 \le n_4 &
			                                            	 !^{n_6}(!^{n_4}\delta \otimes \,!^{n_5}\epsilon) \prec: !^{n_2}\gamma \\
			                                            !^{n_1}\beta \prec: \,!^{n_5}\epsilon, & n_6 \le n_5 & 1 \le n_1
	  		                                        \end{array} \right\}}
			                                        \pair{x}{x} : \,!^{n_2}\gamma$}
			\RightLabel{$(\lambda)$}
			\UnaryInfC{$\varnothing \vdash_{\left\{ \scriptsize
			                                          \begin{array}{lll} 
			                                            !^{n_1}\beta \prec: \,!^{n_4}\delta, & n_6 \le n_4 &
			                                            	 !^{n_6}(!^{n_4}\delta \otimes \,!^{n_5}\epsilon) \prec: !^{n_2}\gamma \\
			                                            !^{n_1}\beta \prec: \,!^{n_5}\epsilon, & n_6 \le n_5 &
			                                              !^{n_3} (!^{n_1} \beta \multimap !^{n_2}\gamma) \prec: \,!^{n_0} \alpha \\
			                                            & 1 \le n_1 &
	  		                                        \end{array} \right\}} \lambda x.\pair{x}{x} : \,!^{n_0} \alpha$}
		\end{prooftree}
	The annotations $\chi$ are omitted to avoid cluttering the notation. Thus, we have
		$$ \text{ConstraintTyping}\,(\varnothing \vdash \lambda x.\pair{x}{x} : \,!^{n_0} \alpha) = \hspace{1in}$$ $$\left\{ \scriptsize
			                                          \begin{array}{lll} 
			                                            !^{n_1}\beta \prec: \,!^{n_4}\delta, & n_6 \le n_4 &
			                                            	 !^{n_6}(!^{n_4}\delta \otimes \,!^{n_5}\epsilon) \prec: !^{n_2}\gamma \\
			                                            !^{n_1}\beta \prec: \,!^{n_5}\epsilon, & n_6 \le n_5 &
			                                              !^{n_3} (!^{n_1} \beta \multimap !^{n_2}\gamma) \prec: \,!^{n_0} \alpha \\
			                                            & 1 \le n_1 &
	  		                                        \end{array} \right\} $$
\end{exmp}

The idea of the constraint typing algorithm is that we can check the existence of a typing derivation of $\Gamma \vdash t : T$ by first
collecting in $\lset$ all the constraints that must be met for it to exist. $\lset$ characterizes the possible valid instantiations of
$\Gamma \vdash t : T$. Then, to find solutions, we look for solutions of $\lset$. If no substitution can be found that solves $\lset$,
then we know that $t$ is not typeable with the type $T$ in the context $\Gamma$. The following definition formalizes this idea.

\begin{defn}
	\label{alg_problem}
	Suppose $\Gamma$ a typing context, $t$ a term, $T$ a type, and $\lset$ a constraint set, such that the constraint typing relation
	$\Gamma \vdash_\lset t : T ~|_\chi$ holds.
	We say that a substitution $\sigma$ is a {\em solution for the problem 	$(\Gamma, t, T, \lset)$} if it is a solution of $\lset$.
\end{defn}

We now have two different ways of characterizing the possible instantiations of $\Gamma \vdash t : T$ as
 	\begin{enumerate}
 		\item The solutions for $(\Gamma, t, T)$, according to Definition \ref{decl_problem}.
 		\item Supposing the relation $\Gamma \vdash_\lset t : T ~|_\chi$, the solutions for $(\Gamma, t, T, \lset)$ in the sense of Definition
 			\ref{alg_problem}.
 	\end{enumerate}

We now have to show that these definitions are equivalent. That is, we need to show that every solution $\sigma$ for $(\Gamma, t, T, \lset)$
is a solution for $(\Gamma, t, T)$. This implication is proved by the Soundness Theorem. Conversely, we also prove that from every solution
for $(\Gamma, t, T)$ we can derive a solution for $(\Gamma, t, T, \lset)$ by adding bindings for the freshly generated variables, as stated
by the Completeness Theorem.

\begin{prop}{\bf (Soundness).} \\
	Suppose that $\Gamma \vdash_\mathcal{L} t : T ~|_\chi$. If $\sigma$ is a solution
	of $\mathcal{L}$, then it is also a solution for $(\Gamma, t, T)$.

	\begin{proof}
 		By induction on the constraint typing derivation. If the term is
 		\begin{itemize}
 		\item $x$. Then, supposing that $\Gamma = \,!^I\Delta, x : U$, $\mathcal{L} = \{ 1 \le I, U \prec: T \}$. By hypothesis, $\sigma$ is a solution
 			of $\mathcal{L}$, which implies that $\sigma I = 1 \dots 1$ and $\sigma U <: \sigma T$.
 			By application of $(ax)$, we obtain $\sigma \Gamma \vdash x : \sigma T$.
 			
 		\item $\lambda x.t$. From the inversion of the rule $(\lambda)$, we have
		 		$$\mathcal{L} = \mathcal{L'} \cup \{p \le I, ~!^p(!^n\alpha \multimap \,!^m\beta) \prec: T\}$$
		 	supposing $\Gamma \equiv \,!^I \Gamma$ and $\Gamma, x : \,!^n\alpha \vdash_\mathcal{L'} t : \,!^m\beta$.
		 	$\sigma$ being a solution of $\mathcal{L}$, it is also a solution of $\mathcal{L'}$, and
		 	by the induction hypothesis, it is a solution for $(\Gamma, x : \,!^n\alpha, \,t, \,!^m\beta)$. Depending on the value
		 	of $\sigma (p)$,
		 	\begin{itemize}
		 		\item If $\sigma(p) = 1$, then $\sigma I = 1 \dots 1$. By the rule $(\lambda_1)$,
		 			$!^1\sigma\Gamma \vdash \lambda x.t : \,!^1(\sigma(!^n \alpha) \multimap \sigma(!^m \beta))$ is a valid typing judgement.
		 			
		 		\item If $\sigma(p) = 0$, then by the rule $(\lambda_2)$,
		 			$\sigma\Gamma \vdash \lambda x.t : \,!^0(\sigma(!^n \alpha) \multimap \sigma(!^m \beta))$ is a valid typing judgement.
		 	\end{itemize}
			In either case, due to the relation $!^{\sigma(p)} (\sigma(!^n\alpha) \multimap \sigma(!^m\beta)) <: \sigma T$, the admissible rule
			(sub-typing) can be applied, and $\sigma \Gamma \vdash \lambda x . t : \sigma T$ is a valid typing judgement.
			This proves that $\sigma$ is a solution for $(\Gamma, \lambda x.t, T)$.
		
		\item $t \, u$. Supposing that
			\begin{center}
			\begin{tabular}{l}
				$\Gamma |_t \vdash_{\mathcal{L}_t} t : \,!^0(!^n\alpha \multimap T)$ \\
				$\Gamma |_u \vdash_{\mathcal{L}_u} u : \,!^n\alpha$ \\
				$\Gamma \backslash_{t \oplus u} = \,!^I \Delta$
			\end{tabular}
			\end{center}
			$\mathcal{L} = \mathcal{L}_t \cup \mathcal{L}_u \cup \{ 1 \le I \}$. By hypothesis, $\sigma$ is a solution of $\mathcal{L}$,
			which means it is also a solution of $\mathcal{L}_t$, $\mathcal{L}_u$ and $\sigma I = 1 \dots 1$.
			Thus, by the induction hypothesis,
				\begin{center}
				\begin{tabular}{lc}
					$\sigma \Gamma|_t \vdash t : \,!^0(\sigma (!^n\alpha) \multimap \sigma T)$ & and \\
			  	$\sigma \Gamma|_u \vdash u : \sigma (!^n\alpha)$ & are both valid.
			  \end{tabular}
			  \end{center}
			By definition, the contexts $\Gamma|_t$ and $\Gamma|_u$ can be written as $\Gamma_1, \Delta|_{t, u}$ and $\Gamma_2, \Delta |_{t, u}$,
			with $\Gamma_1 \cap \Gamma_2 = \varnothing$. By property of $\sigma$ ($\sigma I = 1 \dots 1$),
			$\sigma \Delta|_{t, u} = \,!^1 \sigma \Delta|_{t, u}$. Hence by application of the rule $(app)$,
			the typing judgement $\sigma\Gamma_1, \sigma\Gamma_2, \,!^1\sigma\Delta|_{t, u} \vdash t \, u : \sigma T$ is valid.
			Consequently, using the lemma \ref{weakening} with the set $\sigma\Delta \backslash_{t, u}$, a derivation can be built of the judgement
			$\sigma\Gamma \vdash t \, u : \sigma T$.
			
		\item The same reasoning is used to prove the remaining cases.
		
 		\end{itemize}
	\end{proof}
\end{prop}

\begin{prop}{\bf (Completeness).} \\
	Suppose $\Gamma \vdash_\mathcal{L} t : T ~|_\chi$. If $\sigma$ is a solution for $(\Gamma, t, T)$ such that
	$dom(\sigma) \cap \chi = \varnothing$, then there 	is some solution $\sigma'$ of $\mathcal{L}$ such that
	$\sigma' \backslash \chi = \sigma$.
	
	\begin{proof}
		By induction on the constraint typing derivation. If the last case is
		\begin{itemize}
		\item $x$. The typing judgement $\sigma \Gamma \vdash x : \sigma T$ being valid, $\sigma \Gamma$ must be of the form
			$!^1 \Delta, x : \sigma U$, and $\sigma U <: \sigma T$. Thus, $\sigma$ is a solution of $\mathcal{L} = \{ U <: T, 1 \le I \}$.
			
		\item $\lambda x.t$. Suppose $\Gamma, x : \,!^n\alpha \vdash_\mathcal{L'} t : \,!^m\beta ~|_\chi$, $\mathcal{L} = \mathcal{L'} \cup
			\{ p \le I, \,!^p(!^n\alpha \multimap !^m\beta) \}$.
			Depending on the form taken by $\sigma T$:
			\begin{itemize}
			\item If $\sigma T = \,!^1 (T_1 \multimap T_2)$, then the typing derivation is
				\begin{prooftree}
					\AxiomC{$!^1 \sigma\Gamma, x : T_1 \vdash t : T_2$}
					\RightLabel{$(\lambda_1)$}
					\UnaryInfC{$!^1 \sigma\Gamma \vdash \lambda x.t : \,!^1(T_1 \multimap T_2)$}
				\end{prooftree}
				Since $dom(\sigma) \cap \{\alpha, \beta, m, n, p\} = \varnothing$, $\sigma$ can be extended as $\sigma'$ with
					$$\sigma' = \sigma, [p \mapsto 1, \,!^n\alpha \mapsto T_1, \,!^m\beta \mapsto T_2]$$
				By construction, $\sigma' (\Gamma, x : \,!^n\alpha) \equiv \sigma\Gamma, x : T_1$ and $\sigma' (!^m \beta) = T_2$. Hence
				$\sigma' \Gamma, x : \sigma'(!^n\alpha) \vdash t : \sigma'(!^m \beta)$ is valid. Finally,
				$dom(\sigma') \cap \chi = (dom(\sigma) \cup \{\alpha, \beta, n, m, p\}) \cap \chi = \varnothing$.
				This means $\sigma'$ is a solution for $(\Gamma,x : \,!^n\alpha, t, \,!^m\beta)$ with the right property
				$dom(\sigma') \cap \chi = \varnothing$. By the induction hypothesis, $\sigma^*$ solution of $\mathcal{L'}$ can be constructed
				with the property $\sigma^* \backslash_\chi = \sigma'$.
				$\sigma^*$ also satisfies $\sigma^*\backslash_{\chi \cup \{\alpha, \beta, n, m , p\}} =
				\sigma' \backslash_{\{\alpha, \beta, n, m, p \}} = \sigma$ and $\sigma^* \mathcal{L}$ solved. \\
					
			\item If $\sigma T = \,!^0 (T_1 \multimap T_2)$.
				The same reasoning applies, using $\sigma'$ solution for $(\Gamma,x : \,!^n\alpha, t, \,!^m\beta)$
					$$\sigma' = \sigma, [p \mapsto 0, \,!^n\alpha \mapsto T_1, !^m\beta \mapsto T_2]$$
				to construct $\sigma^*$ with the desired properties.
			\end{itemize}
		
		\item $t \, u$. Supposing that
			$$\left\{ \begin{array}{l}
				\Gamma|_t \vdash_{\mathcal{L}_1} t : \,!^0(!^n\alpha \multimap T) ~|_{\chi_1} \\
				\Gamma|_u \vdash_{\mathcal{L}_2} u : \,!^n\alpha ~|_{\chi_2} \\
				\Gamma\backslash_{t \oplus u} = \,!^I\Delta
			\end{array} \right.$$
			we get $\mathcal{L} = \mathcal{L}_1 \cup \mathcal{L}_2 \cup \{ 1 \le I \}$ and
			$\chi = \chi_1 \cup \chi_2 \cup \{ \alpha, n\}$. We are given $\sigma$ solution for $(\Gamma, t \, u, T)$ such that
			$dom(\alpha) \cap \chi = \varnothing$.
			The typing derivation for $\sigma \Gamma \vdash t \, u : \sigma T$ must be
				\begin{prooftree}
					\AxiomC{$\sigma\Gamma_1, !^1 \sigma\Delta \vdash t : \,!^0 (U \multimap \sigma T)$}
					\AxiomC{$\sigma\Gamma_2, !^1 \sigma\Delta \vdash u : U$}
					\RightLabel{$(app)$}
					\BinaryInfC{$\sigma\Gamma = \sigma\Gamma_1, \sigma\Gamma_2, !^1 \sigma\Delta \vdash t \, u : \sigma T$}
				\end{prooftree}
			The contexts $\Gamma|_t$ and $\Gamma|_u$ are related to $\Gamma_1$, $\Gamma_2$ and $\Delta$ by the relations
			$\Gamma|_t = \Gamma_1, \Delta|_t$ and $\Gamma|_u = \Gamma_2, \Delta|_u$. More importantly, the context
			$\Delta$ from the constraint typing rule is the same as the one from the regular typing rules. This means
			that by definition $\sigma I = 1 \dots 1$. \\
			By hypothesis, $dom(\sigma) \cap \{ \alpha, n \} = \varnothing$. Thus $\sigma$ can be extended as
			$\sigma' = \sigma, [!^n\alpha \mapsto U]$. \\
			By construction of $\sigma'$, the following properties are satisfied
				$$\left\{ \begin{array}{l}
									   \sigma' \Gamma = \sigma \Gamma \\
									   \sigma'(!^0(!^n \alpha \multimap T)) = \,!^0 (U \multimap \sigma T) \\
									   dom(\sigma') \cap \chi_1 = dom(\sigma) \cap \chi_1 = \varnothing
									 \end{array} \right.$$
		  Consequently, $\sigma'$ is a solution for $((\Gamma_1, \Delta), ~t, ~!^0(!^n \multimap T))$ satisfying $dom(\sigma') \cap \chi_1 = \varnothing$.
			But this statement has to be adjusted to match the problem $(\Gamma|_t, t, \,!^0 (!^n \alpha\multimap T), \mathcal{L}_1)$.
			This is done by applying the lemma \ref{QP'prop} which states that $\sigma'$ is also a solution for
			$(\Gamma_1, \Delta|_t, ~t, ~!^0(!^n\alpha \multimap T)) \equiv (\Gamma|_t, ~t, ~!^0(!^n\alpha \multimap T))$.
			By the induction hypothesis, $\sigma^*$ can be constructed that is solution of $\mathcal{L}_1$, with the property
			$\sigma^* \backslash_{\chi_1} = \sigma'$. We now need to modify $\sigma^*$ to solve $\mathcal{L}_2$ as well.
			By construction of $\sigma^*$, the properties are satisfied
				$$\left\{ \begin{array}{l}
										 \sigma^* \Gamma = \sigma' \Gamma = \sigma \Gamma \\
										 \sigma^* (!^n \alpha) = \sigma' (!^n \alpha) = U \\
										 dom(\sigma^*) \cap \chi_2 = dom(\sigma) \cap \chi_2 = \varnothing ~~(\text{because} ~\chi_1 \cap \chi_2 = \varnothing)
					  			 \end{array}\right.$$
			Thus $\sigma^*$ is a solution for $((\Gamma_2, \Delta), ~u, ~!^n\alpha)$, which means for $(\Gamma|_u, ~u, ~!^n\alpha)$ as well.
			It also satisfies $dom(\sigma^*) \cap \chi_2 = \varnothing$. By the induction hypothesis, there exists $\sigma^{**}$ solution of
			$\mathcal{L}_2$ such that $\sigma^{**} \backslash_{\chi_2} = \sigma^*$. \\
			Overall, $\sigma^{**}$ is a solution of $\mathcal{L}_1$, $\mathcal{L}_2$ and $\{ 1 \le I \}$, which means it solves $\mathcal{L}$.
			Moreover, $\sigma^{**} \backslash_{\chi_1 \cup \chi_2 \cup \{\alpha, n\}} = \sigma^* \backslash_{\chi_1 \cup \{\alpha, n\}} =
			\sigma' \backslash_{\{ \alpha, n\}} = \sigma$.
			
		\item The same approach is used to solve the remaining cases.
		\end{itemize}
	\end{proof}
\end{prop}

\subsection{Unification}

A constraint set can have infinitely many solutions, but some may be better than others. In particular, we will give a privileged status to solutions
that make fewer assumptions about the structure of types. For example, the set $\lset = \{!^n \alpha \prec: !^m \beta\}$ will have
$\sigma = [\alpha \mapsto \beta, n \mapsto 1, m \mapsto 0]$ and $\tau = [\alpha \mapsto {!}^1 A \otimes {!}^1 A, \beta \mapsto {!}^1 A \otimes {!}^1 A,
n \mapsto 1, m \mapsto 0]$ as solutions, but we want to say that $\sigma$ is ``more general" than $\tau$.

\begin{defn} Suppose two substitutions $\alpha$ and $\alpha'$. We say that $\alpha$ is \textit{more general} than $\alpha'$, written
	$\alpha \sqsubseteq \alpha'$ if there exists a substitution $\tau$ such that $\alpha' = \tau \circ \alpha$.
\end{defn}

With this definition, we can now say in the example above that $\sigma \sqsubseteq \tau$. Pierce {\cite{pierce02}} goes even further by stating the existence
of a most general unifier of a constraint set, a substitution solution of a set that is more general that any other solution. However,
while this is true in simply-typed lambda calculus, it is not true $\QP'$. A constraint set $\lset$ doesn't have  most general unifier, at least not according to the definition given
by Pierce. To prove this, it suffices to consider the set $\{ m \le n \}$. Three solutions of this constraint set exist:
	$$ \begin{array}{ccl}
		   \sigma_1 & = & [m \mapsto 0, n \mapsto 0] \\
			 \sigma_2 & = & [m \mapsto 0, n \mapsto 1] \\
			 \sigma_3 & = & [m \mapsto 1, n \mapsto 1]
			\end{array} $$
and it is easy to verify that none is more general than the others.
We adapt the definition of the most general unifier to be:

\begin{defn} A \textit{most general unifier} of a constraint set $\lset$ is a pair of a substitution $\sigma$ and an atomic
	constraint set $\lset'$ such that:
		\begin{itemize}
			\item For every solution $\tau$ of $\lset	'$, $\tau \circ \sigma$ is a solution of $\lset$.
			\item For every solution $\rho$ of $\lset$, there exists a substitution $\tau$ solution of $\lset'$ such that $\rho = \tau \circ \sigma$.
		\end{itemize}
\end{defn}

This definition is only useful if such most general unifiers actually exist. Thus we need to prove the theorem:
\begin{thm}
	\label{existence-MGU}
	Suppose a constraint $\lset$ that has at least one solution $\sigma$. Then $\lset$ has a most general unifier.
\end{thm}

This notion of most general unifier is analogous to the one defines for the purpose of unification in the simply-typed lambda calculus.
The unification algorithm proceeds in two steps:
	\begin{enumerate}
		\item Type unification. The algorithm reduces and solves the sub-typing constraints of the input set $\lset$.
			The output is a substitution $\sigma$, and a residual constraint set $\lset'$ which contains only atomic sub-typing constraints and
			flag constraints.
			
		\item Flag unification. This step reduces the flag constraints. It inputs the constraint set $\lset'$ left by the type unification,
			and returns a complementary substitution $\tau$, along with an atomic constraint set.
	\end{enumerate}

\subsubsection{Type Unification}

\begin{algorithm}{\bf (Reduction).}
  Given a constraint set $\lset$, we define its {\em reduction} $\lset^*$ by repeated application of the following rules.
  If any constraint $c$ in the set is:
  	\begin{itemize}
      \item $T \multimap U \prec: T' \multimap U'$, replace $c$ by $T' \prec: T$ and $U \prec: U'$;
      \item $T \otimes U \prec: T' \otimes U'$, replace $c$ by $T \prec: T'$ and $U \prec: U'$;
      \item $!^n A \prec: \, !^m B$, replace $c$ by $A \prec: B$ and $m \le n$;
      \item $1 \prec: 1$, remove the constraint;
      \item For any other composite type constraint, the algorithm fails.
    \end{itemize}
  When all the composite constraints have been reduced, there remains a set of constraints either atomic ($\alpha \prec: \beta$) or
  semi-composite ($T \prec: \alpha$ or $\alpha \prec: T$).
\end{algorithm}

\begin{lemma} \label{reductionQP'}
	Let $\lset$ be a constraint set, and $\lset^*$ the reduction of $\lset$. Then $\sigma$ is a solution of $\lset$ if and only if it is
	a solution of $\lset*$.
	
	\begin{proof}
		The right to left implication is proved by application of the sub-typing relation rules defined in Figure \ref{subtypingQP'}.
		The fact that all these rules are invertible gives the left to right implication.
	\end{proof}
\end{lemma}

Before we can define the type unification algorithm (Algorithm~\ref{alg-type-uni} below), we need an ordering of type variables that ensures termination and gives an upper
bound of its complexity (even though the limit itself is exponential).

\begin{defn} Given a constraint set $\lset$, let $\vset_\lset$ be the set of the free type variables of $\lset$. Every equivalence relation
	on the elements of $\vset_\lset$ defines a partition $\mathcal{C}$ of $\vset_\lset$ whose elements are the equivalence classes.
	Let $\equiv$ be a smallest equivalence relation on type variables such that for all $\alpha,\beta$:
		\begin{center}
			If $\exists \gamma_1 \dots \gamma_n$, s.t. $\alpha \prec: \gamma_1 \dots
				\gamma_n \prec: \beta$, then $\alpha \equiv \beta$.
		\end{center}
	For all variables $\alpha \in \vset_\lset$, we write $c_\alpha$ for the the equivalence class of $\alpha$.
	We also equip the set $\mathcal{C}$ with the partial relation $\ll$ defined as follows:
  	\begin{center}
  	  For all constraints $\alpha \prec: T$ or $T \prec: \alpha$, $\forall \beta \in FV(T), c_\alpha \ll c_\beta$
  	\end{center}
\end{defn}

Informally, if $\alpha\ll\beta$, we say that the variable $\alpha$ is
{\em younger} than $\beta$. The intuition behind this terminology is
that in a constraint $\alpha \prec: T$,
we see that $\alpha$ can only be defined after the introduction of all the variables $\beta$ of $T$. This is analogous to the topological sorting.

Note that the set $\mathcal{C}$ equipped with the partial relation
$\ll$ may be inconsistent. For example, let $\bullet$ be any binary type
operator (i.e., $\otimes$ or $\loli$). Any set with
a constraint like $\alpha \prec: \,!^*\beta ~\bullet~ \,!^*\alpha$ gives the relation $c_\alpha \ll c_\alpha$. More generally,
an inconsistent set is the symptom of an infinite type definition $\alpha = \alpha \bullet \alpha$. As infinite types are not included
in the definition of the type system, such cases will lead the algorithm to failure.
  
\begin{algorithm}{\bf (Variable ordering).}
  Here is a sketch of an algorithm for computing the relations $\equiv$ and $\ll$. 
  The first step is to sort out all the atomic constraints $\alpha \prec: \beta$, which we use to build the set of equivalence classes
  $\mathcal{C}$. The relation $\ll$ on classes is then defined using the remaining constraints and the above definition.
  Suppose this poset is implemented as a graph, whose vertices are the equivalence classes, and where each edge
  $c_\alpha \rightarrow c_\beta$ corresponds an relation $c_\beta \ll c_\alpha$. An in-depth exploration of this graph starting from any
  vertex will either:
  	\begin{itemize}
  		\item Reach a vertex $c_\alpha$ that has no outgoing edges. Then $c_\alpha$ is a minimum class of the poset.
  		\item Reach a vertex $c_\alpha$ that has already been explored. This means the poset is inconsistent, and the algorithm fails.
  	\end{itemize}
  Since each class is explored at most once, this algorithm is linear in the number of vertices (and by extension in the number of type
  variables).
\end{algorithm}

We are now ready to state the type unification algorithm. 
In the definition of the algorithm, the notation $\{ A_k \prec: B_k \}_k$ denotes the set $\bigcup_{k} \{ A_k \prec: B_k \}$.

\begin{algorithm}{\bf (Type Unification).}\label{alg-type-uni}
 \\
	The algorithm TypeUnification inputs a set of linear constraints $\mathcal{L}$, and returns
	a substitutions $\sigma$ associated with a set $\mathcal{L'}$ whose only sub-typing constraints are atomic.
	The algorithm proceeds as follows: \\
	\\
  Let $\alpha_1 \dots \alpha_n$ be the the youngest variables under the $\ll$ ordering. The constraint set $\mathcal{L}$ is formed of the constraints
  $\mathcal{L'} \cup \{ \alpha_{i_k} \prec: \alpha_{j_k} \}_k \cup \{ \alpha_{i_l} \prec: B_l \}_l
	\cup \{ A_m \prec: \alpha_{j_m} \}_m$.
  The set $\mathcal{L'}$ contains all the constraints unrelated to $\alpha_1 \dots \alpha_n$,
  and the linear types $A_m, B_l$ are all composite.
  		
  Depending on the types $\{A_m, B_l\}_{l, m}$:
	\begin{itemize}
		\item If this set is empty, then \\
			\begin{tabular}{l}
				let $\sigma, \mathcal{L''} = ~\text{TypeUnification}\,(\mathcal{L'})$ \\
				return $\sigma, \mathcal{L''} \cup \{ \alpha_{i_k} <: \alpha_{j_k} \}_k$
			\end{tabular}
	  
	  \item If not, it contains an element $C$, which can be either $1$ or $C^1 \,\bullet \, C^2$, with $\bullet$ one in $\{ \otimes, \multimap \}$.
		  The algorithm proceeds as follow:\\
	  	\begin{tabular}{l}
	  		for each $\alpha_i$ \\
	  		~~ if $C = 1$ then \\
	  		~~ ~~ map $\alpha_i$ to $1$ \\
	  		~~ else \\
	  		~~ ~~ let $n_i, m_i, \beta_i, \gamma_i$ be fresh \\
	  		~~ ~~ map $\alpha_i$ to $!^{n_i} \beta_i ~\bullet~ !^{m_i}\gamma_i$ in $\sigma$ \\
	  		substitute $\alpha_1 \dots \alpha_n$ in $\mathcal{L}$ \\
	  		reduce $\mathcal{L}$ \\
	  		let $\sigma', \mathcal{L''} = ~\text{TypeUnification}\,(\mathcal{L})$ \\
	  		return $\sigma' \circ \sigma, \mathcal{L''}$
	  	\end{tabular}
  \end{itemize}

  Of course, if the set of sub-typing constraints is empty, the algorithm stops.
  TypeUnification can fail in two different ways: either it attempts to construct an infinite
  type, and the variable ordering fails, or the reduction encounters an absurd composite constraint (such as $A\otimes B <: C\loli D$).
\end{algorithm}

\begin{defn} Let $\mathcal{L}$ be a constraint set. The following notations are introduced:
	\begin{itemize}
		\item[$\#\mathcal{L}$] for the complexity of $\mathcal{L}$, given by the number of connectives $\multimap$,
			$\otimes$ and symbols $1$ used in the types of the sub-typing constraints.
		\item[$|\mathcal{L}|$] for the number of constraints (sub-typing and flag) of $\mathcal{L}$.
	\end{itemize}
	Sets $\mathcal{L}$ are indexed by the pair $(\#\mathcal{L}, |\mathcal{L}|)$, equipped with the lexicographical order.
\end{defn}

\begin{thm}{\bf (Termination).} \\
	\textit{For all constraint sets $\mathcal{L}$, the unification algorithm applied to $\mathcal{L}$ terminates. }
	\begin{proof}
		By induction on indexed sets (see definition above). The induction hypothesis is this: \textit{if for all sets of index $I^{-} \lneq I$ the
		algorithm terminates, then for all sets of index $I$ TypeUnification terminates}. The initial case corresponds to sets of index $(0, N)$.
		If the index of the set is
		\begin{itemize}
			\item $(0, N_\lset)$, then $\mathcal{L}$ is only composed of atomic constraints, and TypeUnification returns
				immediately when applied to $\mathcal{L}$.
				
			\item $(C_\lset, N_\lset)$ and $\lset = \lset' \cup \{ \alpha_{i_k} \prec: \alpha_{j_k} \}_k$, where $\alpha_1 \dots \alpha_n$ are
				the youngest variables. Since $\mathcal{L'} \subsetneq \mathcal{L}$, it ensues that $N_\lset' = |\mathcal{L'}| < N_\lset$.
				Moreover, because all the constraints of $\mathcal{L} \,\backslash\, \mathcal{L'}$ are atomic, $C_\lset' = \#\mathcal{L'} = C_\lset$.
				Consequently, the index $(C_\lset', N_\lset')$ of $\mathcal{L'}$ is such that $(C_\lset', N_\lset') \lneq (C_\lset, N_\lset)$.
				Thus, the induction hypothesis applies, and thus we know that the function call $\text{TypeUnification}\,(\mathcal{L'})$ terminates, and the same
				for $\text{TypeUnification}\,(\mathcal{L})$.
			
			\item $(C_\lset, N_\lset)$ and $\lset = \lset' \cup \{ \alpha_{i_k} \prec: \alpha_{j_k} \}_k \cup \{ \alpha_{i_l} \prec: B_l \}_l
				\cup \{ A_m \prec: \alpha_{j_m} \}_m$, where $\alpha_1 \dots \alpha_n$ are the youngest variables.

				If $\{A_m, B_l\}_{l, m}$ contains $C = 1$. Each variable $\alpha_i$ is mapped in $\sigma$ to $1$.
				The result of the application of the new mappings to $\mathcal{L}$ is:
			  	$$\sigma \mathcal{L} = \mathcal{L'} \cup \{1 \prec: 1\}_k \cup \{ 1 \prec: B_l \}_l \cup \{ A_m \prec: 1 \}_m$$
       	The reduction of the constraints $\{ 1 \prec: B_l \}_l \cup \{ A_m \prec: 1 \}_m$ can only finish with either an error or the empty set.
       	Thence, what remains after reduction of the composite constraints is only $\mathcal{L'}$.
       	The complexity of the reduced constraint set $\sigma \mathcal{L}$ is:
			  	$$C_\lset' = \#\sigma\mathcal{L} = \#\mathcal{L'} $$
			  which is strictly less than $C_\lset$, meaning $(C_\lset',|\sigma\mathcal{L}|) \lneq (C_\lset, N_\lset)$.
			  The induction hypothesis holds, and the function call $\text{TypeUnification}\,(\sigma \mathcal{L})$ terminates,
			  giving the termination of the call $\text{TypeUnification}\,(\mathcal{L})$. \\
							
				If it contains $C = C_1 \,\bullet\, C_2$. Since the variable ordering did not fail, we know for sure that
						$$\alpha_1 \dots \alpha_n ~ \notin ~ \bigcup_{l} FV(B_l) \cup \bigcup_{m} FV(A_m)$$
				Each $\alpha_i$ is mapped in $\sigma$ to $!^{p_i} \beta_i ~ \bullet ~ !^{q_i}\gamma_i$.
				The result of the application of the new mappings $\sigma$ to $\mathcal{L}$ is:
			  	\begin{center}
			  	\begin{tabular}{lcl}
			  		$\sigma \mathcal{L}$ & $ = $ & $\mathcal{L'}$ \\
				  	& $ \cup $ & $\{!^{p_{i_k}} \beta_{i_k} ~ \bullet ~ !^{q_{i_k}}\gamma_{i_k} \prec: \,
				  	              \,!^{p_{j_k}} \beta_{j_k} ~ \bullet ~ !^{q_{j_k}}\gamma_{j_k} \}_k$ \\
				  	& $ \cup $ & $\{!^{p_{i_l}} \beta_{i_l} ~ \bullet ~ !^{q_{i_l}}\gamma_{i_l} \prec: B_l \}_l$ \\
				  	& $ \cup $ & $\{ A_m \prec: \,!^{p_{i_m}} \beta_{i_m} ~ \bullet ~ !^{q_{i_m}}\gamma_{i_m} \}_m$
				  \end{tabular}
				  \end{center}
       	What remains after reduction of the composite constraints is
					\begin{center}
			  	\begin{tabular}{lcl}
			  		$\sigma \mathcal{L}$ & $ = $ & $\mathcal{L'}$ \\
			  		& $ \cup $ & $\{q_{j_k} \le q_{i_k}, ~p_{j_k} \le p_{i_k}, ~\beta_{i_k} \prec: \beta_{j_k}, ~\gamma_{i_k} \prec: \gamma_{j_k} \}_k$ \\
			  		& $ \cup $ & $\{ \beta_{i_l} \prec: B^1_l, ~ \gamma_{i_l} \prec: B^2_l \}_l$ \\
			  		& $ \cup $ & $\{ A^1_m \prec: \beta_{j_m}, ~ A^2_m \prec: \gamma_{j_m} \}_m$ \\
			  		& $ \cup $ & $\{$ some flag constraints $\}$
			  	\end{tabular}
			  	\end{center}
				depending on the connective $\bullet$, the orientation of the constraints $\beta_{i_k} \prec: \beta_{j_k}$,
				$A^1_m \prec: \beta_{j_m}$ and $\beta_{i_l} \prec: B^1_l$ may be reversed, though it is of no consequence to the proof.
			  In the reduction of the constraints with $A_m$ and $B_l$, it is assumed that the types $A_m$ and $B_l$ are of the form
			  $!^*A^1_m ~\bullet~ !^*A^2_m$ and $!^*B^1_l ~\bullet~ !^*B^2_l$. The value of the flags doesn't really matter here. \\
			  Following, the complexity of the reduced constraint set $\sigma \mathcal{L}$ compared to that of the original set $\mathcal{L}$ is:
			  	$$C_\lset' = \#\sigma\mathcal{L} = \#\mathcal{L'} + \sum_{l} (\#B_l - 1) + \sum_{m} (\#A_m - 1) $$
			  which is strictly inferior to $C_\lset$, meaning $(C_\lset',|\sigma\mathcal{L}|) \lneq (C_\lset, N_\lset)$.
			  The induction hypothesis holds, and the function call $\text{TypeUnification}\,(\sigma \mathcal{L})$ terminates,
			  giving the termination of the call $\text{TypeUnification}\,(\mathcal{L})$.
			  
		\end{itemize}
	\end{proof}
\end{thm}

\begin{remark} Properties $(a)$ and $(b)$ in the following correctness
  theorem of type unification very much resemble the definition of the
  most general unifier.  However, the constraint set that is returned
  along with the substitution can still have {\em non-atomic flag
    constraints}.  Solving these remaining constraints is the purpose
  of the flag unification algorithm, which will be introduced in
  Section~\ref{sssec-flag}.
\end{remark}


\begin{thm}{\bf (Correctness).} \\
	Let $\mathcal{L}$ be a constraint set. The unification algorithm applied to $\mathcal{L}$ produces
	a substitution $\sigma$ and a set $\mathcal{L'}$ such that:
		\begin{center}
		\begin{tabular}{ll}
			(a) & for every solution $\tau$ of $\mathcal{L'}$, $\tau \circ \sigma$ is a solution of $\mathcal{L}$ \\
			(b) & for every solution $\rho$ of $\mathcal{L}$, there exists $\tau$ solution of $\mathcal{L'}$ such that $\rho = \tau \circ \sigma$ \\
			(c) & $\sigma \mathcal{L'} = \mathcal{L'}$
		\end{tabular}
		\end{center}

	\begin{proof}
		The third property (c) is easier to prove than the other two, and the proof will be omitted. The same case by case arguments are
		used to justify the use of the induction hypothesis as in the proof of termination, so this spares us the trouble of having to write
		them again. By induction on indexed sets. If the index is
		\begin{itemize}
			\item $(0, N_\lset)$. $\lset$ is composed only of atomic constraints, and TypeUnification returns $\varnothing, \lset$.
				The properties (a) and (b) are trivially satisfied.
				
			\item $(C_\lset, N_\lset)$, and $\lset = \lset' \cup \{ \alpha_{i_k} \prec: \alpha_{j_k} \}_k$. \\
				Let $\sigma, \mathcal{L''} = ~\text{TypeUnification}\,(\mathcal{L'})$. By application of the induction hypothesis, $\sigma$ and
				$\mathcal{L''}$ respect the properties (a) and (b).
					\begin{itemize}
						\item[(a)] Let $\tau$ be a solution of $\mathcal{L''} \cup \{ \alpha_{i_k} \prec: \alpha_{j_k} \}_k$.
							$\tau$ being a solution of $\mathcal{L''}$, the application of the property (a) with $\tau$, $\mathcal{L''}$ and
							$\sigma$ proves that $\tau \circ \sigma$ is a solution of $\mathcal{L'}$. Overall, $\tau \circ \sigma$ is a
							solution of $\mathcal{L}$.
							
						\item[(b)] Let $\rho$ be a solution of $\mathcal{L}$, and therefore of $\lset'$.
							The property (b) applied with $\mathcal{L''}$ and $\sigma$ ensures the existence
							of $\tau$ solution of $\mathcal{L''}$ such that $\tau \circ \sigma$ is equal to $\rho$. Moreover, it follows from (c) that
							$\sigma \{ \alpha_{i_k} \prec: \alpha_{j_k} \}_k = \{\alpha_{i_k} \prec: \alpha_{j_k} \}_k$. Thus,
								$$(\tau \circ \sigma) \{\alpha_{i_k} \prec: \alpha_{j_k} \}_k = \tau \{\alpha_{i_k} \prec: \alpha_{j_k} \}_k$$
							Since $\rho = \tau \circ \sigma$ is a solution of $\{\alpha_{i_k} \prec: \alpha_{j_k} \}_k$, it follows that $\tau$ is a solution of
							$\{\alpha_{i_k} \prec: \alpha_{j_k} \}_k$, and also of $\lset'' \cup \{\alpha_{i_k} \prec: \alpha_{j_k} \}_k$. This completes
							the proof of the property (b).
					\end{itemize}
			
			\item $(C_\lset, N_\lset)$, and $\mathcal{L} = \mathcal{L'} \cup \{ \alpha_{i_k} \prec: \alpha_{j_k} \}_k \cup
				\{ \alpha_{i_l} \prec: B_l \}_l \cup \{ A_m \prec: \alpha_{j_m} \}_m$.
				The case where the composite types $\{ A_m, B_l \}_{l, m}$ are all equal to $1$ is solved easily, as the mappings
				$[\alpha_i \mapsto 1]_i$ give the only solution for $\alpha_1 \dots \alpha_n$.
				By applying the induction hypothesis, the properties (a) and (b) are easily proved. \\
				
				Otherwise, let $\sigma$ be the substitution $\sigma = [\alpha_i \mapsto !^{p_i}\beta_i \,\bullet\, !^{q_i}\gamma_i]_i$, and
				$\mathcal{L}^*$ the reduced set $\sigma \mathcal{L}$. Finally, let $\sigma', \mathcal{L''} = ~\text{TypeUnification}\,(\mathcal{L}^*)$.
				The induction hypothesis applied to $\mathcal{L}^*$ proves that $\sigma', \mathcal{L''}$ respect the properties (a) and (b).
					\begin{itemize}
						\item[(a)] Let $\tau$ be a solution of $\mathcal{L''}$. By application of (a), $\tau \circ \sigma'$ is a solution of $\mathcal{L}^*$.
							Since the solutions of $\mathcal{L}^*$ and $\sigma \mathcal{L}$ are the same, $\tau \circ \sigma'$ is also a solution of
							$\sigma \mathcal{L}$. Thence, $\tau \circ \sigma' \circ \sigma$ is a solution of $\mathcal{L}$.
							
						\item[(b)] Let $\rho$ be a solution of $\mathcal{L}$. In particular, since $\rho$ is assumed to be idempotent,
							it contains the mappings:
								$$\rho = \rho', [\alpha_i \mapsto T_i \,\bullet\, U_i]_i$$
							Thus it can be modified to be:
								$$\rho = (\rho',[!^{p_i}\beta_i \mapsto T_i]_i, [!^{q_i}\gamma_i \mapsto U_i]_i) \circ
									[\alpha_i \mapsto \,!^{p_i}\beta_i \,\bullet\, !^{q_i}\gamma_i]_i$$
							Using the notation $\rho^* = \rho' ,[!^{p_i}\beta_i \mapsto T_i]_i,[!^{q_i}\gamma_i \mapsto U_i]_i$,
							$\rho = \rho^* \circ \sigma$. By application of the sub-typing relation rules on $\mathcal{L}^*$, it can be proved
							that $\rho^*$ is a solution of $\mathcal{L}^*$. By application of the induction hypothesis and (b), there exists
							$\tau$ solution of $\mathcal{L''}$, such that $\rho^* = \tau \circ \sigma'$. Hence $\rho = (\tau \circ \sigma') \circ \sigma =
							\tau \circ (\sigma' \circ \sigma)$.
					\end{itemize}
		\end{itemize}
	\end{proof}
\end{thm}

\subsubsection{Flag Unification}\label{sssec-flag}

\begin{algorithm}{\bf (Flag unification).} \\
	The algorithm inputs a set of constraints that contains only atomic type constraints, and returns a substitution
	$\sigma$ on flags, along with the remaining constraints (all atomic).
	
	The algorithm looks for constraints of the form
	\begin{itemize}
		\item $n \le 1$ or $0 \le n$, those constraints don't affect the set of solutions of the constraint set, and thus can be removed
			without any consequences.
		\item $1 \le n$, in which case it removes the constraint, and sets $n$ to $1$.
		\item $n \le 0$, in which case it removes the constraint, and sets $n$ to $0$.
	\end{itemize}	
	When setting a flag to a value, the algorithm also needs to replace the instances of this flag in the constraint set.
	If no such constraint remains, it returns. The algorithm fails if and only if it encounters an absurdity,
	as in the case of the set $\{ 1 \le 0 \}$.
	The termination of FlagUnification is obvious because the number of constraints strictly diminishes each step.
	The correctness is also straightforward.
\end{algorithm}

We now can connect TypeUnification and FlagUnification, to build the unification algorithm.

\begin{defn} {\bf (Unification).} \\
	The algorithm inputs a constraint set $\lset$ and successively applies TypeUnification and FlagUnification:
		\begin{center}
		\begin{tabular}{l}
			Unification($\lset$) = do \\
			~~ ~~ let $(\sigma, \lset') = \text{ TypeUnification} \,(\lset)$ \\
			~~ ~~ let $(\sigma', \lset'') = \text{ FlagUnification} \,(\lset')$ \\
			~~ ~~ return $(\sigma' \circ \sigma, \lset'')$
		\end{tabular}
		\end{center}
\end{defn}

\begin{lemma} Suppose $\lset$ a solvable constraint set. Then Unification$(\lset)$ returns the most general unifier of $\lset$.
	\begin{proof}
		Using the properties $(a)$ and $(b)$ of the correctness of TypeUnification, and the correctness of FlagUnification.
	\end{proof}
\end{lemma}

\subsubsection{Approximations}

Algorithm~\ref{alg-type-uni} gives an exact, and most general solution to the type unification algorithm. However, this result is obtained at the cost of a high complexity:
the application of the unification algorithm to the set
{ \small
	$$\lset = \{\alpha_0 \prec: (!^{n_1}\alpha_1 ~\bullet~ !^{n_1}\alpha_1), ~~ 
	\alpha_1 \prec: (!^{n_2}\alpha_2 ~\bullet~ !^{n_2}\alpha_2) ~~ \dots ~~ \alpha_{i-1} \prec: (!^{n_i}\alpha_i ~\bullet~ !^{n_i}\alpha_{i}) \}$$ }
potentially leads to the creation of an exponential number of variables, and the resulting set also contains an exponential number of
constraints.
This is why it may be interesting to make some concessions to lower the cost of the algorithm. Also, the goal of the algorithm is
often only to determine the {\em existence} of solutions, and not necessarily find all the possible ones.
We propose some approximations that keep the satisfiability of the constraint set, but restrict the set of inferred solutions:
	
\begin{itemize}
	\item In the case where $\mathcal{L} = \mathcal{L'} \cup \{\alpha_{i_k} \prec: \alpha_{j_k} \}_k$,
		make the approximation ${[\alpha_i \mapsto \alpha_1]_{i \in \{2 \dots n\}}}$.
		
	\item In the case where $\mathcal{L} = \mathcal{L'} \cup \{T \prec: \alpha_1 \prec: \dots \prec: \alpha_n \prec: U\}$, make the
		approximation $\sigma = [\alpha_i \mapsto T]_i$, and then apply TypeUnification to $\mathcal{L'} \cup \{ T \prec: U \}$.
		
	\item In the case where $\mathcal{L} = \mathcal{L'} \cup \{\alpha_{i_k} \prec: \alpha_{j_k}\}_k \cup
		\{T_{i, 1} \dots \{T_l \prec: \alpha_{j_l} \}_l$, make the substitution $\sigma = {[\alpha_i \mapsto \alpha_1]_{i \in \{2 \dots n\}}}$,
		and apply the algorithm to $\mathcal{L'} \cup \{T_l \prec: \alpha_1\}_l$.
		
	\item The same for the case $\mathcal{L} = \mathcal{L'} \cup \{\alpha_{i_k} \prec: \alpha_{j_k}\}_k \cup \{\alpha_{i_m} \prec: U_m\}_m$.
\end{itemize}

\subsection{Extending Type Inference to Core Proto-Quipper}
\label{ssec-extending-to-QP}

This section gives the additions that have to be made to the grammar of $\QP'$ for the type inference algorithm to perform in the syntax
of core Proto-Quipper.
We now consider that all the terms of Proto-Quipper have been added to $\QP'$. First, we need to include the types $\bool, \qubit, \Circ(T, U)$
in the rule of $\texttt{LinearType}$, along with the typing rules from Figure \ref{exttypeQP'}

\begin{figure}[!ht]
\begin{mdframed}
	\begin{prooftree}
		\AxiomC{}
		\RightLabel{$(\rev)$}
		\UnaryInfC{$!^1 \Gamma \vdash \rev : ~!^1 (!^1\Circ(T, U) \multimap~ !^1\Circ(U, T))$}
	\end{prooftree}
	\begin{prooftree}
		\AxiomC{}
		\RightLabel{$(\unbox)$}
		\UnaryInfC{$!^1\Gamma \vdash \unbox : ~!^1 (!^1\Circ(T, U) \multimap~ !^1(T \multimap U))$}
	\end{prooftree}
	\begin{prooftree}
		\AxiomC{}
		\RightLabel{$(\boxx)$}
		\UnaryInfC{$!^1\Gamma \vdash \boxx^T : ~!^1(!^1(T \multimap U) \multimap~ !^1\Circ (T, U))$}
	\end{prooftree}
	$$ \mbox{
		\AxiomC{}
		\RightLabel{$(\textit{true})$}
		\UnaryInfC{$!^1 \Gamma \vdash \texttt{True} : ~!^1 \bool$}
		\DisplayProof
		~~
		~~
		\AxiomC{}
		\RightLabel{$(\textit{false})$}
		\UnaryInfC{$!^1 \Gamma \vdash \texttt{False} : ~!^1 \bool$}
		\DisplayProof
	} $$
	\begin{prooftree}
		\AxiomC{$\Gamma_1, ~!^1\Delta \vdash t : ~!^0 \bool$}
		\AxiomC{$\Gamma_2, ~!^1\Delta \vdash u : T$}		\noLine
		\UnaryInfC{$\Gamma_2, ~!^1\Delta \vdash v : T$}
		\RightLabel{$(\textit{if})$}
		\BinaryInfC{$\Gamma_1, \Gamma_2, !^1\Delta \vdash \ifthenelse{t}{u}{v} : T$}
	\end{prooftree}
\end{mdframed}
\caption{Additional typing rules.}
\label{exttypeQP'}
\end{figure}
In the same way, the constraint typing relation has to be extended to cover the new terms. The additional typing rules are developed in Figure \ref{extctypeQP'}.

\begin{figure}[!ht]
\begin{mdframed}
	\begin{prooftree}
	  \AxiomC{$\alpha, \beta, n, m \notin \Gamma, T$} \noLine
		\UnaryInfC{$\mathcal{L} = \{1 \le I, ~ !^1 (!^1\Circ(!^n\alpha, \,!^m\beta) \multimap~ !^1 \Circ(!^n\beta, \,!^m\alpha)) \prec: T\}$}
		\RightLabel{$(\rev)$}
		\UnaryInfC{$!^I \Gamma \vdash_\mathcal{L} \rev : T ~|_{\{\alpha, \beta, m, n\}}$}
	\end{prooftree}
	\begin{prooftree}
		\AxiomC{$\alpha, \beta, n, m \notin \Gamma, T$} \noLine
		\UnaryInfC{$\mathcal{L} = \{1 \le I, ~ !^1 (!^1\Circ(!^n\alpha, \,!^m\beta) \multimap~ !^1 (!^n\alpha \multimap \,!^m\beta)) \prec: T\}$}
		\RightLabel{$(\unbox)$}
		\UnaryInfC{$!^I\Gamma \vdash_\mathcal{L} \unbox : T ~|_{\{\alpha, \beta, m, n\}}$}
	\end{prooftree}
	\begin{prooftree}
		\AxiomC{$\alpha, n \notin \Gamma, T, U$} \noLine
		\UnaryInfC{$\mathcal{L} = \{1 \le I, ~ !^1(!^1(T \multimap \,!^n\alpha) \multimap~ !^1 \Circ (T, !^n\alpha)) \prec: U\}$}
		\RightLabel{$(\boxx)$}
		\UnaryInfC{$!^I\Gamma \vdash \boxx^T : U ~|_{\{\alpha, n\}}$}
	\end{prooftree}
	$$ \mbox{
		\AxiomC{$\mathcal{L} = \{1 \le I, ~ !^1 \bool \prec: T\}$}
		\RightLabel{$(\textit{true})$}
		\UnaryInfC{$!^I \Gamma \vdash \texttt{True} : T ~|_\varnothing$}
		\DisplayProof
                \quad
		\AxiomC{$\mathcal{L} = \{1 \le I, ~ !^1 \bool \prec: T\}$}
		\RightLabel{$(\textit{false})$}
		\UnaryInfC{$!^I \Gamma \vdash \texttt{False} : T ~|_\varnothing$}
		\DisplayProof
	} $$
	\begin{prooftree}
		\AxiomC{$\Gamma|_t \vdash_{\mathcal{L}_1} t : ~!^0 \bool ~|_{\chi_1}$} \noLine
		\UnaryInfC{$\Gamma|_{u, v} \vdash_{\mathcal{L}_2} u : T ~|_{\chi_2}$}	 \noLine
		\UnaryInfC{$\Gamma|_{u, v} \vdash_{\mathcal{L}_3} v : T ~|_{\chi_3}$}
		\AxiomC{$\chi_1$, $\chi_2$, $\chi_3$ mutually disjoint} \noLine
		\UnaryInfC{$\Gamma \backslash_{t \oplus (u,v)} = \,!^I \Delta$} \noLine
		\UnaryInfC{$\mathcal{L} = \mathcal{L}_1 \cup \mathcal{L}_2 \cup \mathcal{L}_3 \cup \{ 1 \le I \}$}
		\RightLabel{$(\textit{if})$}
		\BinaryInfC{$\Gamma \vdash_\mathcal{L} \ifthenelse{t}{u}{v} : T ~|_{\chi_1 \cup \chi_2 \cup \chi_3}$}
	\end{prooftree}
\end{mdframed}
\caption{Additional constraint typing relation rules.}
\label{extctypeQP'}
\end{figure}

The reader may have noticed that, although the circuits $(t, C, a)$ have been included in the extended $\QP'$ syntax, no associated typing rule or
constraint typing relation rules were specified. The reason for this is that the type inference algorithm is designed to infer the
types of the terms coded by a programmer. However, the constructor $(t, C, a)$ has never been intended to be directly accessible.
More specifically, the programmer can access a built-in collection of basic gate circuits, whose types are already known at compile time.



% ----------------------------------------------------------------------
\clearpage
\section{Language extensions}
\label{sec-extensions}

The implementation of Proto-Quipper provides all of the features of
Core Proto-Quipper described in Sections~\ref{sec-core} and
{\ref{sec-type-safety}}, and also includes the type inference
algorithm of Section~\ref{sec-inference}. The implementation also
includes some additional features that are not part of Core
Proto-Quipper. These extensions are provided for the convenience of
the programmer, but are not yet formalized. This means that the proofs 
of type-safety for these extensions of the core language are 
currently under development. We briefly describe these features 
here. Further explanations and sample code can be found in the 
user guide of Section \ref{sec-user-guide}.

\subsection{Polymorphism} 

Our implementation of Proto-Quipper is equipped with a polymorphic
$\textit{let}$ binder. This means that expressions such as:
\[
\letin{x}{a}{b}
\]
are valid expressions of the syntax (note that $a$ is not required
to be a pair). Moreover, such $\textit{let}$ expressions are not
simply syntactic sugar for $(\lambda x.b)a$. In fact, the
expression will not be attributed a type but rather a \emph{type
  scheme} (also known as a {\em polymorphic type}), which
describes all possible types of this expression.  Type schemas
follow the pattern:
$$\forall \vec{\alpha}~ \vec{i}, \lset \Rightarrow T$$
This is to be interpreted as: 
\begin{quote} For all type variables
  $\vec{\alpha'}$ and flag variables $\vec{i'}$ satisfying the
  constraints $\lset[\vec{\alpha'} / \vec{\alpha}, \vec{i'}/
  \vec{i}]$, the expression ``$a$" can have the type
  $T[\vec{\alpha'} / \vec{\alpha}, \vec{i'}/ \vec{i}]$.
\end{quote}
This is illustrated by the following constraint typing rule:
\begin{prooftree}
  \AxiomC{$\vec{\alpha'}, \vec{i'} \notin \Delta, (\forall \vec{\alpha} ~\vec{i}, \lset_T \Rightarrow T)$} \noLine
  \UnaryInfC{$\lset = \lset_T[\vec{\alpha'}/\vec{\alpha}, \vec{i'}/\vec{i}] \cup
    \{ 1 \le I, T[\vec{\alpha'}/\vec{\alpha}, \vec{i'}/\vec{i}] \prec: U \}$}
  \RightLabel{$(\rul{ax}.S)$}
  \UnaryInfC{${!}^I \Delta, x: ~(\forall \vec{\alpha} ~\vec{i}, \lset_T \Rightarrow T) \vdash_\lset x : U ~|_{\{\vec{\alpha'}, \vec{i'}\}}$}
\end{prooftree}
For example, in the term 
\[ \letin{f}{\lambda x.x}{b},
\]
the expression $f$ will be given the type scheme 
\[ \forall \alpha, \beta,k,n,m,\s{\alpha \prec:\beta, m\leq n} \imp {!}^k({!}^n\alpha\loli{!}^m\beta),
\]
Some possible types for this expressions are $A\loli A$, ${!}A\loli
A$, ${!}A\loli{!}A$, ${!}(A\loli B)\loli ({!}A\loli B)$, and so on. 
Since $f$ can be typed differently at each calls site, a term such as
\[ \letin{f}{\lambda x.x}{f(f)}
\]
is typeable in polymorphic Proto-Quipper.

\subsection{Algebraic types, integers and lists}

Our implementation of Proto-Quipper provides built-in types of
integers and lists. These are accompanied by basic functions such as
equality testing, comparison, etc. Moreover, Proto-Quipper supports
user-definable recursive types (in fact, lists are a special case of
this, and are defined using this general mechanism).

\subsection{Recursion}

Our implementation of Proto-Quipper allows recursive function
definitions. To do this, we add the following term constructor to the
language:
\[
\letrecin{a\,x}{b}{c}
\]    
together with the associated typing rule:
\[
\infer[(\rul{rec})]{{!}\Delta;Q \entails \letrecin{a\,x}{b}{c}:C}{
  {!}\Delta,a:{!}(A\loli B), x:A;\emptyset \entails b:B
  &
  {!}\Delta, a:{!}(A\loli B) ;Q \entails c:C      
}
\]
and reduction rule:
\[
\infer[(\rul{rec}).]{[C,\letrecin{a\,x}{b}{c}]\to [C, c[\lambda x.(\letrecin{a\,x}{b}{b})/a]}{}
\]
Preliminary research shows that the core language can be extended 
with this recursion operator without losing type-safety. 

% ----------------------------------------------------------------------
\clearpage
\section{Future work}\label{sec-future}

In this section, we describe some work that is ongoing or planned for
the future. As already mentioned in Section~\ref{ssec-goals}, the
rationale behind the design of Proto-Quipper was to start with the
simplest language possible, establish type-safety, and then extend the
language in small steps with the goal of eventually adding most of
Quipper's features to Proto-Quipper in a type-safe way. 

\subsection{Extensions and proof mechanization}

On project that is currently ongoing is to incorporate the extensions
discussed in the previous section (polymorphism, algebraic data types,
and recursion) into the type safe fragment of the language. This
progressive augmentation process is inherent to Proto-Quipper since
the language was defined to be a type safe reconstruction of
Quipper. 

Currently, making even small extensions to Core Proto-Quipper is a
labor-intensive process, as for every change to the language, type
system, or operational semantics, the proofs of type safety have to be
rewritten. Because this process is time-consuming and error-prone, we
are investigating the use of proof assistant software (such as Coq,
Isabelle, Abella, Twelf, Beluga) to formalize the definition of
Proto-Quipper and the proofs of its meta-properties.

\subsection{Reversibility and measurements}

In Proto-Quipper 0.1, all circuits are reversible. This follows from
the definition of the $(\rev)$ and $(\rul{circ})$ typing rules and
will have to be modified to accommodate non-reversible gates such as
measurements. In such a setting, the type system should ensure that
circuits are reversed only if it is meaningful to do so. In
particular, if a circuit contains a measurement, then it should not be
possible to reverse it.

\subsection{Dependent types and parameterized families of circuits}
 
A circuit generating function that inputs a list of qubits does not
define just one circuit, but rather a family of circuits parameterized
by the length $n$ of the list.  To box such a function, a particular
value of $n$ has to be given. In Quipper, we refer to $n$ as the
``shape'' of the argument of the function. Operations such as boxing
and reversing often require shape information. An alternative solution
would be to equip Proto-Quipper with a {\em dependent type
  system}. This would allow shape information to the stored at the
type level.

\subsection{Imperative-style arguments}

In the current version of Proto-Quipper, all the wires manipulated in
the body of a function must be explicitly returned. It is not possible
to implicitly return the wires that have not been terminated. In
Quipper, we call this the ``functional style'' of programming. Unlike
Proto-Quipper, Quipper also permits an ``imperative style'' of
programming. We plan to extend the Proto-Quipper type system to handle
such imperative-style arguments in a type-safe way.

\subsection{Type-safe cleanup of garbage}

When programming classical oracles, usually some number of ancilla
bits are used, and returned in a ``dirty'' state at the end of the
circuit. The standard trick for making such an oracle reversible is to
uncompute the ancillas by running the computation in reverse. When
programming in a strictly linear language such as Proto-Quipper, all
the ancilla bits must be explicitly accounted for in the type of the
oracle function. A useful extension of the type system would be to
handle such ``garbage'' ancillas automatically.

\subsection{Non-linear controls}

Proto-Quipper handles qubits in a strictly linear fashion. This
ensures that qubits are never discarded or duplicated and is therefore
key in guaranteeing that the type system enforces physical properties
such as no-cloning. However, this discipline is too strict since
certain operations, like controlling, can safely use a qubit in
non-linear way. For example, if a quantum-not gate on qubit $q$ is
controlled by qubits $r$ and $s$, then we must have $q\neq r$ and
$q\neq s$, but it is not necessary that $r\neq s$. This situation
frequently arises in the context of automatically generated quantum
oracles, such as in the Quantum Linear Systems algorithm. In fact,
this is the reason that strict linearity checking is currently
disabled in Quipper. It is an interesting problem how to extend the
Proto-Quipper type system the non-linear use of qubits as controls.

\subsection{Dynamic lifting}

In contrast to the quantum lambda calculus, the reduction in
Proto-Quipper is non-probabilistic. Of course, the hypothetical
quantum device running the circuit produced by Proto-Quipper would
have to perform probabilistic operations, but the circuit generation
itself does not have to. This is justified by the ``principle of
deferred measurement" which states that any quantum circuit is
equivalent to one where all measurements are performed as the very
last operations (see, e.g., \cite{NC02} p.186). We therefore do not
need to rely on the result of a measurement to construct circuits and,
in theory, no computational power is lost by making this
assumption. In practice, however, this delaying of measurement may
significantly increase the size of the circuit. Thus in terms of
computational resources it is sometimes advantageous to permit circuit
generating functions that access previous measurement results. In
fact, several existing quantum algorithms, including the Unique
Shortest Vectors algorithm, rely on such interactive circuit
building. In the Unique Shortest Vectors algorithm, this takes the
form of a sieving step. 

In Quipper, this capability is captured by the notion of \emph{dynamic
  lifting} wherein a value of type $\bit$ (a circuit-execution time
value) is lifted to a value of type $\bool$ (a circuit-generation time
value). Adding such a feature to Proto-Quipper would make the
reduction relation probabilistic. It is a research problem how such an
extension can be carried out in a type-safe way. 

% ----------------------------------------------------------------------
\clearpage
\section{Proto-Quipper user guide}
\label{sec-user-guide}

In this section, we provide a tutorial-style introduction to 
programming in Proto-Quipper. 

\subsection{Installing and invoking Proto-Quipper}

Install instructions can be found in the README file accompanying the code. 
To build the code type \verb#make# in the main Proto-Quipper directory. 

The \verb#qlib# directory contains some basic modules and examples that can 
be used to test the install. To invoke Proto-Quipper on a file, call 
\verb#./ProtoQuipper qlib/qft.qp -fv#. This will run Proto-Quipper on the sample code \verb#qft.qp#. The \verb#-fv# option selects the visual format. 
This should produce the following output, representing the circuit for the 
Quanutm Fourier Transform on three qubits (QFT3) together with its inferred 
type:
\begin{verbatim}
$ ./ProtoQuipper qlib/qft.qp -fv

----------------------*-----*----[H]---
                      |     |          
----------*----[H]----|----[R]---------
          |           |                
---[H]---[R]---------[R]---------------

 : !circ(qbit * qbit * qbit, qbit * qbit * qbit)
\end{verbatim}
The default format option is \verb#ir# which outputs the intermediate representation of quantum circuits specified in the context of the QCS program. The \verb#-h# option displays the list of available command line 
options. If multiple files are passed as arguments to Proto-Quipper, they 
are sequentially executed. If no argument file is given, Proto-Quipper 
enters its interactive mode:
\begin{verbatim}
$ ./ProtoQuipper
### Proto-Quipper -- Interactive Mode ###
# 
\end{verbatim}
In interactive mode, all commands must end with \verb#;;#. This mode accepts:
\begin{itemize}
  \item import statements, e.g., \verb#import Gates ;;#,
  \item type definitions, e.g., 
  \verb#type list a = Nil | Cons of a * list a ;;# and
  \item top-level declarations, e.g., \verb#let x = 1 ;;# or \verb#1+1 ;;#.
\end{itemize}
Commands in interactive mode always start with a \verb#:#. For example, 
\verb#:help# will display a list of available commands. These are (other 
than \verb#:help# itself): 
\begin{itemize}
  \item \verb#:ctx# which lists the variables of the current context, 
  \item \verb#:exit# which quits the interactive mode and
  \item \verb#:display# which displays the top-level circuit.
\end{itemize}
To invoke a command any one of the corresponding prefix will suffice.

\subsection{Primitives}

Our implementation of Proto-Quipper was developed in Haskell. However, the 
syntax might be more familiar to programmers having experience with OCaml.

Code in Proto-Quipper is organized in modules. Module names always start 
with an upper-case character. The implementation must be placed in a file 
of the same name but written in lower-case and ending with the {\tt.qp} 
extension. For example, the module {\tt List} will be implemented in 
{\tt list.qp}. Modules are organized as follows.
\begin{enumerate}
  \item The module starts with a list of import declarations, written 
  {\tt import Module}.
  \item All the type definitions of the module are placed after the list 
  of imports.
  \item The rest of the file forms the body of the module, organized as a 
  list of variable declarations and top-level expressions.
\end{enumerate}

The Proto-Quipper primitives include (most of) the formal syntax presented 
in the preceeding sections together with a list of built-in quantum gates. 
Figures \hyperref[types]{\ref*{types}} and \hyperref[terms]{\ref*{terms}} 
describe the correspondance between the formal and implemented versions of 
the core syntax.

\begin{figure}[!ht]
\begin{center}
\renewcommand{\arraystretch}{1.4}
\begin{tabular}{|c|c|}
  \hline
  \textbf{Formalization}    & \textbf{Implementation} \\\hline
  $qbit$, $bool, 1$         & \verb#qbit#, \verb#bool#, \verb#()# \\\hline
  $A \otimes B$             & \verb# A * B# \\\hline
  $A \multimap B$           & \verb# A -> B # \\\hline
  $Circ(T, U)$              & \verb# circ(T,U)# \\\hline
  ${!} A$                   & \verb# !A# \\\hline
\end{tabular}
\end{center}
\caption{Proto-Quipper Types.}
\label{types}
\end{figure}

\begin{figure}[!ht]
\begin{center}
\renewcommand{\arraystretch}{1.4}
\begin{tabular}{|c|c|}
  \hline
  \textbf{Formalization}          & \textbf{Implementation} \\\hline
  $x$                             & \verb# x# \\\hline
  $\texttt{True}, \texttt{False}$ & \verb#true#, \verb#false# \\\hline
  $\pair{a}{b}$                   & \verb# (a,b) # \\\hline
  $*$                             & \verb# ()# \\\hline
  $\lambda x.a$                   & \verb# fun x -> a# \\\hline
  $a \, b$                        & \verb#a b# \\\hline
  $box^T$                         & \verb# box[T]# \\\hline
  $unbox, rev$                    & \verb#unbox#, \verb#rev# \\\hline
  $if ~a~ then ~b~ else ~c$       & \verb# if a then b else c# \\\hline
  $let \pair{x}{y} = a ~in~ b$    & \verb# let (x,y) = a in b# \\\hline
\end{tabular}
\end{center}
\caption{Proto-Quipper Terms.}
\label{terms}
\end{figure}

Note that no term of the form $(t,C,u)$ appears in table 
\hyperref[terms]{\ref*{terms}}. This is because the $(t,C,u)$ construct is 
not accessible to the programmer. In its place, we provide a set of 
primitive quantum gates. The module \verb#Gates#, that can be found in the 
\verb#qlib# directory, contains the definition of a set of basic gates. 
These include: \verb#init0#, \verb#init1#, \verb#term0#, \verb#term1#, 
\verb#not#, \verb#hadamard#, \verb#T#, \verb#S#, \verb#Y#, \verb#Z#, 
\verb#phase#, \verb#W#, \verb#E#, \verb#toffoli#, \ldots Please refer to 
the documentation for a complete list.

\subsection{Bell states}
\label{ssec-quipper-by-e}

Now that we have introduced the primitive functions that can be used in Quipper programs, we can combine them into a larger program. We start with 
a basic example: a circuit producing two qubits in the Bell state. Below is
the code of the module \verb#Bell# contained in a file \verb#bell.qp#.

%\begin{figure}[!ht]
\begin{verbatim}
import Gates

box[] (fun x ->
  let q0 = hadamard (init0 ()) in
  let q1 = init0 () in
  let (q1, q0) = cnot (q1, q0) in
  (q0, q1)
) ;;
\end{verbatim}
%\caption{Contents of the module Bell.}
%\end{figure}

We give a line by line interpretation of the code. The circuit is the standard one found in the litterature and uses the gates \verb#hadamard#, 
\verb#init0# and \verb#cnot# that are provided by the \verb#Gates# module. 
To access this module, we must import it explicitly. This is the role of 
the line:
\begin{center}
  \verb#import Gates#
\end{center}
The \verb#box[]# constructor starts a new circuit of type \verb#circ((), T)#. 
Note that no type annotation was provided. The type is then assumed to be 
the unit type \verb#()#, so that \verb#box[]# is shorthand for 
\verb#box[()]#. The first step is to create a new qubit, initialized in the 
state $\ket{0}$. This is done by a call to \verb#init0#, which is defined as 
\verb#unbox INIT0#, where \verb#INIT0# corresponds to the following circuit 

$$\Qcircuit @C=1em @R=.7em {
  & \lstick{\ket{0}} & \qw
}$$

In fact, \verb#unbox# has been applied to all the primitive gates provided 
in the module \verb#Gates# so that they can be directly used as functions. 
In this manner, the \verb#hadamard# gate is applied to the newly created 
\verb#qbit# and the result is stored in the variable \verb#q0#. Similarly, 
another \verb#qbit# is initialized in the state $\ket{0}$ and stored in the 
variable \verb#q1#. The rest of the code is straightforward. A \verb#cnot# 
gate is applied to the pair \verb#(q1,q0)# and the pair is then returned. 
Remark that the pair \verb#(q0,q1)# \emph{must} be returned as the end of 
the box definition to respect the strict linearity of Proto-Quipper. The 
hypothetical code where \verb#q0, q1# are not returned would generate a 
typing error. Finally, every top-level expression in the body of the 
module must be suffixed by \verb#;;#. 

\subsection{Shorthand notations}

Remark that in the code given above for the Bell-state circuit, 
the first two occurences of the \verb#let# binder are not 
instances of the formal binder defined in subsection 
\hyperref[ssec-types-and-terms]{\ref*{ssec-types-and-terms}}. 
Indeed, the let binding used there not applied to a pair. 

%This is a syntactic sugar introduced in the implementation 
%to make the programming easier. A list of the available 
%shorthand notations is provided in the next subsection.

As is common with a lot of functional programming languages 
(Haskell, OCaml \ldots) some notations are introduced to increase
the readability of the code. Note that the let-binding is \emph{not} a syntactic sugar (it may be) as it is used to introduce
the let-polymorphism (an integral part of all the ML-based languages).
The figure \ref{sugar} gives the list of the notations introduced in Proto-Quipper. Some of those use patterns, objects that are
defined by the grammar
	\begin{center}
	\begin{tabular}{rcl}
		Pattern $p_1 \dots p_n$ & ::= & $x$ \\
		            & $|$ & $(p_1, \dots, p_n)$ \\
		            & $|$ & $()$
	\end{tabular}
	\end{center}

\begin{figure}[!ht]
\begin{tabular}{ll}
  \verb#let p = a in b#       & Binds the variables of the pattern 
                                \verb#p# \\
  \verb#p <- a; b#            & Same as \verb#let p = a in b# \\
  \verb#p <-* a; b#           & Used to compute \verb#let p = a p in b#
                                (\verb#a# must be a function)\\
  \verb#fun p -> a#           & Equivalent of 
                                \verb#fun x -> let p = x in a# \\
  \verb#fun x y z -> a#       & Equivalent of 
                                \verb#fun x -> fun y -> fun z -> a# \\
  \verb#let f x y z = a in b# & Equivalent of 
                                \verb#let f = (fun x y z -> a) in b#
\end{tabular}
\label{sugar}
\caption{Syntactic sugars.}
\end{figure}

\subsection{The Quantum Frourier Transform on three qubits}

With the help of the shorthand notations introduced above, we can write 
larger programs. For example, an implementation of the Quantum Fourier 
Transform of three qubits is given below:
\begin{verbatim}
box[qbit * qbit * qbit] (
  fun (q0, q1, q2) ->
    q2 <-* hadamard;
    (q2, q1) <-* cphase 2;
    q1 <-* hadamard;
    (q2, q0) <-* cphase 4;
    (q1, q0) <-* cphase 2;
    q0 <-* hadamard;
    (q0, q1, q2)
) ;;
\end{verbatim}
This code illustrate the usefulness of the \verb#p <-* a# notation, since 
the result of the application of quantum gates to qubits is usually stored 
in the same variables. Note that the type annotation in \verb#box# is 
\verb#qbit *qbit * qbit#, rather than \verb#qbit * (qbit * qbit)# or 
\verb#(qbit * qbit) * qbit#. This is because our implementation of 
Proto-Quipper allows the use unparenthesized tuples of arbitrary size. As 
with most syntactic sugars, this function producing same output circuit 
could have been coded without using this extension. However, this version 
is more compact. 

\subsection{Extensions}

The extensions discussed here were introduced in section 
\hyperref[sec-extensions]{\ref*{sec-extensions}}. 

\subsubsection{Polymorphism}

The let-bindings are always assumed to be polymorphic. For example, the 
function:
\begin{verbatim}
let double f q = f (f q) ;;
\end{verbatim}
can be used as follows:
\begin{verbatim}
box[qbit * qbit] (fun (q, r) ->
  q <- double not q;
  (q, r) <- double cnot (q, r);
  (q, r)
) ;;
\end{verbatim}
The first occurrence of \verb#double# has the type 
\verb#!(qbit -> qbit) -> qbit -> qbit#, while the second occurrence has type 
\verb#!(qbit * qbit -> qbit * qbit) -> qbit * qbit -> qbit * qbit#.

\subsubsection{Algebraic types, integers and lists}

The formal definition of sum types, with $left$ and $right$ constructors, 
is not particularly user-friendly. We therefore authorize algebraic types, 
which can be seen as a generalization of sum types. The definitions of such 
types have to be placed at the beginning of the source file, immediately 
following the \verb#import# statements. These definitions follow the pattern: 
\begin{verbatim}
type typename [a b ..] =
    Datacon1 [of T1]
  | Datacon2 [of T2]
  ..
  | DataconN [of TN]
\end{verbatim}
where \verb#a#, \verb#b#, \ldots are type variables and \verb#T1#, \ldots, 
\verb#TN# are type expressions possibly involving the variables \verb#a#, 
\verb#b#, \ldots In particluar:
\begin{itemize}
  \item The name of the type must start with a lower case character
  \item The type is followed by a (eventually empty) list of type arguments
  \item The name of the data constructors starts with an upper case  
  character. Each data constructor can take one argument type, introduced 
  by \verb#of#.
\end{itemize}

All the types defined in the same module are assumed to be co-inductive. For 
example, the type of lists included in the module \verb#List# has the 
definition
\begin{verbatim}
type list a =
    Nil
  | Cons of a * list a
\end{verbatim}
Since lists are part of the basic objects, some notations have been introduced in the syntax to simplify the writing and manipulation of lists. 
These are
\begin{center}
\begin{tabular}{ll}
  \verb#[]#               & for ~ \verb#Nil# \\
  \verb#x:xs#             & for ~ \verb#Cons (x, xs)# \\
  \verb#[x1, x2, .., xn]# & for ~ 
                            \verb#Cons (x1, Cons (x2, .. Cons (xn, Nil) ..))#
\end{tabular}
\end{center}
and can be used in either expressions or patterns.

\subsubsection{Recursion}

Recursive functions can be written using the syntax:
\begin{verbatim}
let rec f x y z = a in b
\end{verbatim}
Note that only functions may be declared recursive in Proto-Quipper, whereas 
Haskell authorized recursive values, and for example lists of infinite size.

\subsection{The Quantum Fourier Transform}

With the extensions given in the previous subsection, we can write an 
algorithm for the Quantum Fourier Transform for quantum registers encoded 
as lists of qubits. The QFT is now defined as a recursive function that 
inputs a quantum register. The output is the list of resulting qubits. 

We need to import the \verb#List#, \verb#Gates# and \verb#Core# modules. 
We assume that the \verb#List# module contains the following function, 
with the given type:
\begin{verbatim}
val sf_last <: !(list a -> (a, list a))
\end{verbatim}
This function returns the last element of a list, and for linearity reasons 
also the rest of the list. The module \verb#Core# provides some basic integer 
operations like \verb#+ ^ - * ..#. The \verb#Gates# module you already known 
about.

The implementation of the algorithm is as follow:

\begin{verbatim}
import Core
import Gates
import List

-- Apply the nth phase gate to each qbit of the list, controlled by a parameter
let rec appphase n q qn =
  match qn with
   [] -> (q, [])
 | q0:qr ->
     (q, qr) <- appphase (n+1) q qr;
     (q0, q) <-* cphase (2 ^ n);
     (q, q0:qr)
;;

-- Quantum Fourier Transform
let rec qft qreg =
  match qreg with
    [] -> []
  | q:reg ->
      (qn, qn1) <- sf_last (q:reg);
      qn1 <- qft qn1;
      (qn, qn1) <- appphase 1 qn qn1;
      qn <-* hadamard;
      qn:qn1
;;
\end{verbatim}


% ----------------------------------------------------------------------
\clearpage
\section*{Glossary of acronyms}
\addcontentsline{toc}{section}{Glossary of Acronyms}

\begin{itemize}
  \item $\FQ$ = Free quantum variables.
  \item $\FV$ = Free quantum variables.
  \item GFI = Government Furnished Information
  \item IARPA = Intelligence Advanced Research Projects Activity
  \item $IH$ = Induction hypothesis.
  \item ML = A family of functional programming languages, including
    Standard ML of New Jersey and OCaml.
  \item OCaml = The Objective Caml programming language.
  \item QCS = The IARPA Quantum Computer Science program.
  \item QFT = Quantum Fourier Transform.
  \item $\QP$ = Core Proto-Quipper.
  \item $\QP'$ = Type-inference version of Core Proto-Quipper.
\end{itemize}

% ----------------------------------------------------------------------
\clearpage

\bibliography{biblio}{}
\bibliographystyle{plain}
\addcontentsline{toc}{section}{Bibliography}

\end{document}




Our implementation of Proto-Quipper was developed in Haskell. 
Nonetheless, the syntax of Proto-Quipper might be more familiar 
to programmers having experience with OCaml.

Code in Proto-Quipper is organized as modules. Modules names 
always start with an upper-case character and the implementation 
must be placed in a file of the same name but written in lower-case, 
ending with the {\tt.qp} extension. For example, the module {\tt List} 
will be implemented in {\tt list.qp}.
The organization of a module is as follow :
	\begin{enumerate}
		\item The module starts with a list of import declarations, written {\tt import Module}.
		\item All the type definitions of the module have to be place after the list of imports.
		\item The rest of the file forms the body of the module, organized as a list of variable declarations and
			top-level expressions.
	\end{enumerate}

\subsection{Proto-Quipper primitives}

The figure \ref{expressions} contains the terms of Proto-Quipper and their equivalent in the implementation.

\begin{figure}[!ht]
\begin{center}
\begin{tabular}{|c|c|}
	\hline
	Formalization & Implementation \\\hline
	$qbit$, $bool, 1$, & \verb# qbit bool ()# \\\hline
	$A \otimes B$ & \verb# A * B# \\\hline
	$A \multimap B$ & \verb# A -> B # \\\hline
	$Circ(T, U)$ & \verb# circ(T, U)# \\\hline
	${!} A$ & \verb# !A# \\\hline
\end{tabular}
\end{center}
\label{types}
\caption{Proto-Quipper types}
\end{figure}

\begin{figure}[!ht]
\begin{center}
\begin{tabular}{|c|c|}
	\hline
	Formalization & Implementation \\\hline
	$x$ & \verb# x# \\\hline
	$\texttt{True}, \texttt{False}$ & \verb# true false# \\\hline
	$\pair{a}{b}$ & \verb# (a,b) # \\\hline
	$*$ & \verb# ()# \\\hline
	$\lambda x.ab$ & \verb# fun x -> a# \\\hline
	$box^T$ & \verb# box[T]# \\\hline
	$unbox, rev$ & \verb# unbox rev# \\\hline
	$if ~a~ then ~b~ else ~c$ & \verb# if a then b else c# \\\hline
	$let \pair{x}{y} = a ~in~ b$ & \verb# let (x,y) = a in b# \\\hline
\end{tabular}
\end{center}
\label{terms}
\caption{Proto-Quipper terms}
\end{figure}

A \verb#qbit# is essentially a reference to a logical qubit within a
computation. It can be thought of as a reference to an actual quantum
bit on some physical device, or simply as a reference to a quantum
wire within a circuit that is being generated. \verb#bool# is the type of boolean parameters
(\verb#true#, \verb#false#).

The module \verb#Gates# that can be found in the library qlib contains the definition of a set of basic gates,
which are, amongst others: \verb#init0#, \verb#init1#, \verb#term0#, \verb#term1#, \verb#not#, \verb#hadamard#, \verb#T#,
\verb#S#, \verb#Y#, \verb#Z#, \verb#phase#, \verb#W#, \verb#E#, \verb#toffoli# \ldots
For a complete list of the gate circuits, please refer to the documentation.

\subsection{A simple example}\label{ssec-quipper-by-e}
Having introduced the primitive functions that can be used in Quipper
programs, we can now move on to combining these primitives into a
larger program. The best way to introduce this notation is to start with a basic example: the circuit of the
bell state. Here is the code of the module \verb#Bell#:
\begin{figure}[!ht]
\begin{verbatim}
import Gates

box[] (fun x ->
  let q0 = hadamard (init0 ()) in
  let q1 = init0 () in
  let (q1, q0) = cnot (q1, q0) in
  (q0, q1)
) ;;
\end{verbatim}
\caption{Contents of the module Bell.}
\end{figure}

Here is a line by line interpretation of the code:
The circuit uses the gates \verb#hadamard#, \verb#init0# and \verb#cnot# that are provided by the \verb#Gates# module.
Since we use this module, we must import it explicitly:
  \begin{center}
  	\verb#import Gates#
  \end{center}
The \verb#box[]# constructor starts a new circuit of type \verb#circ((), T)# (if the type annotation is omitted under the brackets, it is
interpreted as the unit type \verb#()#). The first step is to create a new qubit, initialized in the state $|0\rangle$. This is done by
a call to $init0$, which is in fact defined as \verb#unbox INIT0#, where \verb#INIT0# is the circuit
	$$\Qcircuit @C=1em @R=.7em {
	  & \lstick{\ket{0}} & \qw
	 }$$
All the gates defined in the \verb#Gates# module have already been applied to \verb#unbox#, and thus can be used as functions.
The newly created \verb#qbit# is then applied the \verb#hadamard# gate, and stored in the variable \verb#q0#.
As you may have noticed, the let binding used here differs from the formalization in that it is used with something not a pair.
This is a syntactic sugar introduced in the implementation to make the programming easier. We make a list of the shorthand notations
in a later section.
The rest of the code is straightforward. The pair \verb#(q0,q1)# \emph{must} be returned as the end of the box definition, as is ensured by the
strict linearity. The hypothetical code where \verb#q0, q1# are not returned would generate a typing error. 
Finally, every top-level expression in the body of the module must be suffixed by \verb#;;#.


\subsection{Short hand notations}

As is common with a lot of functional programming languages (Haskell, OCaml \ldots) some notations are introduced to increase
the readability of the code. Note that the let-binding is \emph{not} a syntactic sugar (it may be) as it is used to introduce
the let-polymorphism (an integral part of all the ML-based languages).
The figure \ref{sugar} gives the list of the notations introduced in Proto-Quipper. Some of those use patterns, objects that are
defined by the grammar
	\begin{center}
	\begin{tabular}{rcl}
		Pattern $p_1 \dots p_n$ & ::= & $x$ \\
		            & $|$ & $(p_1, \dots, p_n)$ \\
		            & $|$ & $()$
	\end{tabular}
	\end{center}

\begin{figure}[!ht]
\begin{tabular}{ll}
	\verb#let p = a in b# & Binds the variables of the pattern \verb#p# \\
	\verb#p <- a; b#      & Same as \verb#let p = a in b# \\
	\verb#p <-* a; b#     & Used to compute \verb#let p = a p in b# (\verb#a# must be a function)\\
	\verb#fun p -> a#     & Equivalent of \verb#fun x -> let p = x in a# \\
	\verb#fun x y z -> a# & Equivalent of \verb#fun x -> fun y -> fun z -> a# \\
	\verb#let f x y z = a in b# & Equivalent of \verb#let f = (fun x y z -> a) in b#
\end{tabular}
\label{sugar}
\caption{Syntactic sugars.}
\end{figure}

\subsection{Another example}

Now that we introduced short hand notations, we can write larger programs. Let us implement the Quantum Fourier Transform 
of three qubits ($QFT_3$). The content of the file \verb#qft3.qp# is given in the figure \ref{qft3}.

\begin{figure}[!ht]
\begin{verbatim}
box[qbit * qbit * qbit] (
  fun (q0, q1, q2) ->
    q2 <-* hadamard;
    (q2, q1) <-* cphase 2;
    q1 <-* hadamard;
    (q2, q0) <-* cphase 4;
    (q1, q0) <-* cphase 2;
    q0 <-* hadamard;
    (q0, q1, q2)
) ;;
\end{verbatim}
\label{qft3}
\caption{Module Qft3.}
\end{figure}

You can see that the notation \verb#p <-* a# really comes in handy when it comes to applying quantum gates to
qubits, since the result is usually put in the same variables. Not that the type annotation in \verb#box# is
\verb#qbit *qbit * qbit#, rather than \verb#qbit * (qbit * qbit)# or \verb#(qbit * qbit) * qbit#. As the syntactic sugars,
this code could have been written without this extension, with exactly the same circuit output. However, this version is more compact.
Proto-Quipper authorizes the use of tuples of any size.
The next section introduces further extensions, that are not part of the Core language.

\subsection{Extensions}

The extensions defined here were already mentioned in the report on Proto-Quipper. For more information, please refer to these notes.

\subsubsection{Let-polymorphism}

The let-bindings are always assumed to be polymorphic. For example, we could define the function
\begin{verbatim}
let double f q = f (f q) ;;
\end{verbatim}
and use it in the following circuit
\begin{verbatim}
box[qbit * qbit] (fun (q, r) ->
  q <- double not q;
  (q, r) <- double cnot (q, r);
  (q, r)
) ;;
\end{verbatim}
The first occurrence of \verb#double# has the type \verb#!(qbit -> qbit) -> qbit -> qbit#, when the second occurrence is typed with
\verb#!(qbit * qbit -> qbit * qbit) -> qbit * qbit -> qbit * qbit#.

\subsubsection{Algebraic types, integers and lists}

The formal definition of sum types, with the constructors $left$ and $right$, is not particularly user friendly. We therefore authorize
the algebraic types, which can be seen as a generalization of sum types.
The definition of these types has to be placed at the beginning of the source files, just after the \verb#import# statements, and follows the
pattern:
\begin{verbatim}
type typename [a b ..] =
    Datacon1 [of T1]
  | Datacon2 [of T2]
  ..
  | DataconN [of TN]
\end{verbatim}
In particluar:
\begin{itemize}
	\item The name of the type must start with a lower case character
	\item The type is followed by a (eventually empty) list of type arguments
	\item The name of the data constructors starts with an upper case character. Each data constructor can take one argument type, introduced
		by \verb#of#.
\end{itemize}

All the types defined in the same module are assumed to be co-inductive. 
For example, the type of lists included in the module \verb#List# has the definition
\begin{verbatim}
type list a =
    Nil
  | Cons of a * list a
\end{verbatim}
Since lists are part of the basic objects, some notations have been introduced in the syntax to simplify the writing and manipulation
of lists. These are
	\begin{center}
	\begin{tabular}{ll}
		\verb#[]# & for ~ \verb#Nil# \\
		\verb#x:xs# & for ~ \verb#Cons (x, xs)# \\
		\verb#[x1, x2, .., xn]# & for ~ \verb#Cons (x1, Cons (x2, .. Cons (xn, Nil) ..))#
	\end{tabular}
	\end{center}
and can be used in either expressions or patterns.

\subsubsection{Recursion}

Recursive functions can be written using the syntax:
\begin{verbatim}
let rec f x y z = a in b
\end{verbatim}
Note that only functions may be declared recursive in Proto-Quipper, whereas Haskell authorized recursive values, and for example lists of infinite
size.

\subsection{More examples}

Equipped with the extensions given in the previous section, let us write the Quantum Fourier Algorithm for quantum registers encoded as lists
of qubits. The QFT algorithm is now defined in a recursive function that inputs a quantum register, and applies the QFT algorithm. The output
is the list of resulting qubits.

We need to import the \verb#List#, \verb#Gates# and \verb#Core# modules. We assume that the \verb#List# module contains the following function, 
with the given type:
\begin{verbatim}
val sf_last <: !(list a -> (a, list a))
\end{verbatim}
This function returns the last element of a list, and for linearity reasons also the rest of the list.
The module \verb#Core# provides some basic integer operations like \verb#+ ^ - * ..#. The \verb#Gates# module you already known about.

The implementation of the algorithm is as follow:

\begin{verbatim}
import Core
import Gates
import List

-- Apply the nth phase gate to each qbit of the list, controlled by a parameter
let rec appphase n q qn =
  match qn with
   [] -> (q, [])
 | q0:qr ->
     (q, qr) <- appphase (n+1) q qr;
     (q0, q) <-* cphase (2 ^ n);
     (q, q0:qr)
;;

-- Quantum Fourier Transform
let rec qft qreg =
  match qreg with
    [] -> []
  | q:reg ->
      (qn, qn1) <- sf_last (q:reg);
      qn1 <- qft qn1;
      (qn, qn1) <- appphase 1 qn qn1;
      qn <-* hadamard;
      qn:qn1
;;
\end{verbatim}

\end{document}


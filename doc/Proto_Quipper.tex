\documentclass{article}

% ----------------------------------------------------------------------
% Packages:

\usepackage{amssymb}          % Usual AMS packages.
\usepackage{amsmath}
\usepackage{amsthm}
\usepackage{proof}            % For proof trees.
\usepackage{mdframed}         % To box figures.
\usepackage[]{nohyperref}     % Like hyperref (allows referencing 
                              % of sub-lemmas) but without the hyperlinks.
%    Q-circuit version 2
%    Copyright (C) 2004  Steve Flammia & Bryan Eastin
%    Last modified on: 9/16/2011
%
%    This program is free software; you can redistribute it and/or modify
%    it under the terms of the GNU General Public License as published by
%    the Free Software Foundation; either version 2 of the License, or
%    (at your option) any later version.
%
%    This program is distributed in the hope that it will be useful,
%    but WITHOUT ANY WARRANTY; without even the implied warranty of
%    MERCHANTABILITY or FITNESS FOR A PARTICULAR PURPOSE.  See the
%    GNU General Public License for more details.
%
%    You should have received a copy of the GNU General Public License
%    along with this program; if not, write to the Free Software
%    Foundation, Inc., 59 Temple Place, Suite 330, Boston, MA  02111-1307  USA

% Thanks to the Xy-pic guys, Kristoffer H Rose, Ross Moore, and Daniel Müllner,
% for their help in making Qcircuit work with Xy-pic version 3.8.  
% Thanks also to Dave Clader, Andrew Childs, Rafael Possignolo, Tyson Williams,
% Sergio Boixo, Cris Moore, Jonas Anderson, and Stephan Mertens for helping us test 
% and/or develop the new version.

\usepackage{xy}
\xyoption{matrix}
\xyoption{frame}
\xyoption{arrow}
\xyoption{arc}

\usepackage{ifpdf}
\ifpdf
\else
\PackageWarningNoLine{Qcircuit}{Qcircuit is loading in Postscript mode.  The Xy-pic options ps and dvips will be loaded.  If you wish to use other Postscript drivers for Xy-pic, you must modify the code in Qcircuit.tex}
%    The following options load the drivers most commonly required to
%    get proper Postscript output from Xy-pic.  Should these fail to work,
%    try replacing the following two lines with some of the other options
%    given in the Xy-pic reference manual.
\xyoption{ps}
\xyoption{dvips}
\fi

% The following resets Xy-pic matrix alignment to the pre-3.8 default, as
% required by Qcircuit.
\entrymodifiers={!C\entrybox}

\newcommand{\bra}[1]{{\left\langle{#1}\right\vert}}
\newcommand{\ket}[1]{{\left\vert{#1}\right\rangle}}
    % Defines Dirac notation. %7/5/07 added extra braces so that the commands will work in subscripts.
\newcommand{\qw}[1][-1]{\ar @{-} [0,#1]}
    % Defines a wire that connects horizontally.  By default it connects to the object on the left of the current object.
    % WARNING: Wire commands must appear after the gate in any given entry.
\newcommand{\qwx}[1][-1]{\ar @{-} [#1,0]}
    % Defines a wire that connects vertically.  By default it connects to the object above the current object.
    % WARNING: Wire commands must appear after the gate in any given entry.
\newcommand{\cw}[1][-1]{\ar @{=} [0,#1]}
    % Defines a classical wire that connects horizontally.  By default it connects to the object on the left of the current object.
    % WARNING: Wire commands must appear after the gate in any given entry.
\newcommand{\cwx}[1][-1]{\ar @{=} [#1,0]}
    % Defines a classical wire that connects vertically.  By default it connects to the object above the current object.
    % WARNING: Wire commands must appear after the gate in any given entry.
\newcommand{\gate}[1]{*+<.6em>{#1} \POS ="i","i"+UR;"i"+UL **\dir{-};"i"+DL **\dir{-};"i"+DR **\dir{-};"i"+UR **\dir{-},"i" \qw}
    % Boxes the argument, making a gate.
\newcommand{\meter}{*=<1.8em,1.4em>{\xy ="j","j"-<.778em,.322em>;{"j"+<.778em,-.322em> \ellipse ur,_{}},"j"-<0em,.4em>;p+<.5em,.9em> **\dir{-},"j"+<2.2em,2.2em>*{},"j"-<2.2em,2.2em>*{} \endxy} \POS ="i","i"+UR;"i"+UL **\dir{-};"i"+DL **\dir{-};"i"+DR **\dir{-};"i"+UR **\dir{-},"i" \qw}
    % Inserts a measurement meter.
    % In case you're wondering, the constants .778em and .322em specify
    % one quarter of a circle with radius 1.1em.
    % The points added at + and - <2.2em,2.2em> are there to strech the
    % canvas, ensuring that the size is unaffected by erratic spacing issues
    % with the arc.
\newcommand{\measure}[1]{*+[F-:<.9em>]{#1} \qw}
    % Inserts a measurement bubble with user defined text.
\newcommand{\measuretab}[1]{*{\xy*+<.6em>{#1}="e";"e"+UL;"e"+UR **\dir{-};"e"+DR **\dir{-};"e"+DL **\dir{-};"e"+LC-<.5em,0em> **\dir{-};"e"+UL **\dir{-} \endxy} \qw}
    % Inserts a measurement tab with user defined text.
\newcommand{\measureD}[1]{*{\xy*+=<0em,.1em>{#1}="e";"e"+UR+<0em,.25em>;"e"+UL+<-.5em,.25em> **\dir{-};"e"+DL+<-.5em,-.25em> **\dir{-};"e"+DR+<0em,-.25em> **\dir{-};{"e"+UR+<0em,.25em>\ellipse^{}};"e"+C:,+(0,1)*{} \endxy} \qw}
    % Inserts a D-shaped measurement gate with user defined text.
\newcommand{\multimeasure}[2]{*+<1em,.9em>{\hphantom{#2}} \qw \POS[0,0].[#1,0];p !C *{#2},p \drop\frm<.9em>{-}}
    % Draws a multiple qubit measurement bubble starting at the current position and spanning #1 additional gates below.
    % #2 gives the label for the gate.
    % You must use an argument of the same width as #2 in \ghost for the wires to connect properly on the lower lines.
\newcommand{\multimeasureD}[2]{*+<1em,.9em>{\hphantom{#2}} \POS [0,0]="i",[0,0].[#1,0]="e",!C *{#2},"e"+UR-<.8em,0em>;"e"+UL **\dir{-};"e"+DL **\dir{-};"e"+DR+<-.8em,0em> **\dir{-};{"e"+DR+<0em,.8em>\ellipse^{}};"e"+UR+<0em,-.8em> **\dir{-};{"e"+UR-<.8em,0em>\ellipse^{}},"i" \qw}
    % Draws a multiple qubit D-shaped measurement gate starting at the current position and spanning #1 additional gates below.
    % #2 gives the label for the gate.
    % You must use an argument of the same width as #2 in \ghost for the wires to connect properly on the lower lines.
\newcommand{\control}{*!<0em,.025em>-=-<.2em>{\bullet}}
    % Inserts an unconnected control.
\newcommand{\controlo}{*+<.01em>{\xy -<.095em>*\xycircle<.19em>{} \endxy}}
    % Inserts a unconnected control-on-0.
\newcommand{\ctrl}[1]{\control \qwx[#1] \qw}
    % Inserts a control and connects it to the object #1 wires below.
\newcommand{\ctrlo}[1]{\controlo \qwx[#1] \qw}
    % Inserts a control-on-0 and connects it to the object #1 wires below.
\newcommand{\targ}{*+<.02em,.02em>{\xy ="i","i"-<.39em,0em>;"i"+<.39em,0em> **\dir{-}, "i"-<0em,.39em>;"i"+<0em,.39em> **\dir{-},"i"*\xycircle<.4em>{} \endxy} \qw}
    % Inserts a CNOT target.
\newcommand{\qswap}{*=<0em>{\times} \qw}
    % Inserts half a swap gate.
    % Must be connected to the other swap with \qwx.
\newcommand{\multigate}[2]{*+<1em,.9em>{\hphantom{#2}} \POS [0,0]="i",[0,0].[#1,0]="e",!C *{#2},"e"+UR;"e"+UL **\dir{-};"e"+DL **\dir{-};"e"+DR **\dir{-};"e"+UR **\dir{-},"i" \qw}
    % Draws a multiple qubit gate starting at the current position and spanning #1 additional gates below.
    % #2 gives the label for the gate.
    % You must use an argument of the same width as #2 in \ghost for the wires to connect properly on the lower lines.
\newcommand{\ghost}[1]{*+<1em,.9em>{\hphantom{#1}} \qw}
    % Leaves space for \multigate on wires other than the one on which \multigate appears.  Without this command wires will cross your gate.
    % #1 should match the second argument in the corresponding \multigate.
\newcommand{\push}[1]{*{#1}}
    % Inserts #1, overriding the default that causes entries to have zero size.  This command takes the place of a gate.
    % Like a gate, it must precede any wire commands.
    % \push is useful for forcing columns apart.
    % NOTE: It might be useful to know that a gate is about 1.3 times the height of its contents.  I.e. \gate{M} is 1.3em tall.
    % WARNING: \push must appear before any wire commands and may not appear in an entry with a gate or label.
\newcommand{\gategroup}[6]{\POS"#1,#2"."#3,#2"."#1,#4"."#3,#4"!C*+<#5>\frm{#6}}
    % Constructs a box or bracket enclosing the square block spanning rows #1-#3 and columns=#2-#4.
    % The block is given a margin #5/2, so #5 should be a valid length.
    % #6 can take the following arguments -- or . or _\} or ^\} or \{ or \} or _) or ^) or ( or ) where the first two options yield dashed and
    % dotted boxes respectively, and the last eight options yield bottom, top, left, and right braces of the curly or normal variety.  See the Xy-pic reference manual for more options.
    % \gategroup can appear at the end of any gate entry, but it's good form to pick either the last entry or one of the corner gates.
    % BUG: \gategroup uses the four corner gates to determine the size of the bounding box.  Other gates may stick out of that box.  See \prop.

\newcommand{\rstick}[1]{*!L!<-.5em,0em>=<0em>{#1}}
    % Centers the left side of #1 in the cell.  Intended for lining up wire labels.  Note that non-gates have default size zero.
\newcommand{\lstick}[1]{*!R!<.5em,0em>=<0em>{#1}}
    % Centers the right side of #1 in the cell.  Intended for lining up wire labels.  Note that non-gates have default size zero.
\newcommand{\ustick}[1]{*!D!<0em,-.5em>=<0em>{#1}}
    % Centers the bottom of #1 in the cell.  Intended for lining up wire labels.  Note that non-gates have default size zero.
\newcommand{\dstick}[1]{*!U!<0em,.5em>=<0em>{#1}}
    % Centers the top of #1 in the cell.  Intended for lining up wire labels.  Note that non-gates have default size zero.
\newcommand{\Qcircuit}{\xymatrix @*=<0em>}
    % Defines \Qcircuit as an \xymatrix with entries of default size 0em.
\newcommand{\link}[2]{\ar @{-} [#1,#2]}
    % Draws a wire or connecting line to the element #1 rows down and #2 columns forward.
\newcommand{\pureghost}[1]{*+<1em,.9em>{\hphantom{#1}}}
    % Same as \ghost except it omits the wire leading to the left. 
              % To draw quantum circuits.

% ----------------------------------------------------------------------
% Margins: An environment to locally change the margins.

\newenvironment{changemargin}[2]{%
\begin{list}{}{%
\setlength{\topsep}{0pt}%
\setlength{\leftmargin}{#1}%
\setlength{\rightmargin}{#2}%
\setlength{\listparindent}{\parindent}%
\setlength{\itemindent}{\parindent}%
\setlength{\parsep}{\parskip}%
}%
\item[]}{\end{list}}

% ----------------------------------------------------------------------
% Theorem definitions:
%
% theorem, lemma, proposition, corollary, definition, example, 
% examples, openproblem, principle, remark, remarks, convention, 
% notation
%
% All are numbered by default. Get unnumbered versions by un-lemma etc.

\newcounter{n-lemma}
\newtheorem{n-lema}[n-lemma]{Lemma}
\newtheorem{n-proposition}[n-lemma]{Proposition}
\newtheorem{n-theorem}[n-lemma]{Theorem}
\newtheorem{n-principle}[n-lemma]{Principle}
\newtheorem{n-corollary}{Corollary} 
\newtheorem*{un-lemma}{Lemma}
\newtheorem*{un-theorem}{Theorem}
\newtheorem*{un-proposition}{Proposition}
\newtheorem*{un-corollary}{Corollary} 
\newtheorem*{un-principle}{Principle}

\theoremstyle{definition}
\newtheorem{n-definition}[n-lemma]{Definition}
\newtheorem{n-assumption}[n-lemma]{Assumption}
\newtheorem{n-openproblem}[n-lemma]{Open Problem}
\newtheorem*{un-definition}{Definition}
\newtheorem*{un-assumption}{Assumption}
\newtheorem*{un-openproblem}{Open Problem}

\theoremstyle{remark}
\newtheorem{n-example}[n-lemma]{Example}
\newtheorem{n-examples}[n-lemma]{Example}
\newtheorem{n-remark}[n-lemma]{Remark}
\newtheorem{n-notation}[n-lemma]{Notation}
\newtheorem{n-remarks}[n-lemma]{Remarks}
\newtheorem{n-convention}[n-lemma]{Convention}
\newtheorem*{un-example}{Example}
\newtheorem*{un-examples}{Examples}
\newtheorem*{un-remark}{Remark}
\newtheorem*{un-remarks}{Remarks}
\newtheorem*{un-convention}{Convention}

\newenvironment{theorem}{\begin{n-theorem}}{\end{n-theorem}}
\newenvironment{lemma}{\begin{n-lema}}{\end{n-lema}}
\newenvironment{proposition}{\begin{n-proposition}}{\end{n-proposition}}
\newenvironment{corollary}{\begin{n-corollary}}{\end{n-corollary}}
\newenvironment{definition}{\begin{n-definition}}{\end{n-definition}}
\newenvironment{example}{\begin{n-example}}{\end{n-example}}
\newenvironment{examples}{\begin{n-examples}}{\end{n-examples}}
\newenvironment{openproblem}{\begin{n-openproblem}}{\end{n-openproblem}}
\newenvironment{principle}{\begin{n-principle}}{\end{n-principle}}
\newenvironment{remark}{\begin{n-remark}}{\end{n-remark}}
\newenvironment{remarks}{\begin{n-remarks}}{\end{n-remarks}}
\newenvironment{convention}{\begin{n-convention}}{\end{n-convention}}
\newenvironment{notation}{\begin{n-notation}}{\end{n-notation}}

% ----------------------------------------------------------------------
% Proofs.
\newenvironment{proofof}[1]                     % loose proofs
        {\begin{trivlist}\item[]{\it Proof of #1:\hspace{.5em}}\rm}
        {\end{trivlist}}
\newenvironment{proofsketch}                    % proof sketch
        {\begin{trivlist}\item[]{\it Proof sketch:\hspace{.5em}}\rm}
        {\end{trivlist}}

% ----------------------------------------------------------------------
% Set and category notation.
\newcommand{\Cc}{{\bf C}}
\newcommand{\Dd}{{\bf D}}
\newcommand{\abs}[1]{|{#1}|}                  % |A|: underlying set.
\newcommand{\homof}[2]{\textrm{hom}_{#1}(#2)} % hom set \homof{C}{A,B} = C(A,B).
\renewcommand{\hom}[1]{\homof{}{#1}}          % hom set \hom{A,B}      =  (A,B).
\newcommand{\obj}[1]{|#1|}                    % |C|: objects of a cat.
\newcommand{\seq}{\subseteq}    
\newcommand{\such}{\,\,|\,\,}                 % in sets {x|...}.
\newcommand{\cp}{\circ}                       % composition.
\newcommand{\id}{{\textrm{\rm id}}}           % identity.
\newcommand{\from}{\colon}                    % F\from A\ii B = F:A->B.
\newcommand{\ii}{\rightarrow}                 % functions f:a->b (alternative: \to).
\newcommand{\iii}{\longrightarrow}
\newcommand{\pii}{\rightharpoonup}            % partial function f:a-`b.
\newcommand{\adjoint}{\dashv}                 % F\adjoint G: F is left adjoint.
\newcommand{\iso}{\cong}                      % isomorph ~=.
\newcommand{\family}[2]{(#1)_{#2}}            % (X_n)_J or {X_n|J}..
\newcommand{\rest}[1]{|_{#1}}                 % restriction S|A.
\newcommand{\atuple}[1]{\langle#1\rangle}     % <a,b,c...>.
\newcommand{\rtuple}[1]{(#1)}                 % round tuple    (a,b,c...).
\newcommand{\tuple}{\rtuple}
\newcommand{\p}{\atuple}                      % <a,b>
\newcommand{\copair}[1]{[#1]}                 % [a,b]
\newcommand{\s}[1]{\{#1\}}                    % {a,b}
\newcommand{\ms}[1]{\{| #1 |\}}               % {|a,b|}
\newcommand{\es}{\emptyset}                   % empty set
\newcommand{\defeq}{:=}                       % := or =_{def}
\renewcommand{\iff}{\Leftarrow\!\!\!\Rightarrow}
\newcommand{\defiff}{\mathrel{{:}{\iff}}}     % :<=> or <=>_{def}
\newcommand{\pow}{{\mathscr{P}}}
\newcommand{\catarrow}{\xrightarrow}          % extensible arrow for cat notation
\newcommand{\da}{^\dagger}
\newcommand{\dada}{^{\dagger\dagger}}
\newcommand{\x}{\tensor}
\newcommand{\+}{\oplus}

% ----------------------------------------------------------------------
% particular categories and sets
\usepackage{mathrsfs}                      % special script style \mathscr{S} for \Set
\newcommand{\Set}{\mathscr{S}}             % Cat of Sets
\newcommand{\N}{\mathbb{N}}                % Nat'l numbers AMS fonts
\newcommand{\Z}{\mathbb{Z}}                % Integers AMS fonts
\newcommand{\R}{{\mathbb{R}}}              % Reals
\newcommand{\C}{{\mathbb{C}}}              % Complex
{\makeatletter                             % Cat of finite-dim Hilbert spaces
\gdef\alphalabels{\def\theenumi{\@alph\c@enumi}\def\labelenumi{(\theenumi)}}}
\newcommand{\FinHilb}{\textrm{\bf FinHilb}}

% ----------------------------------------------------------------------
% Spacing and notes.
\newcommand{\sep}{\hspace{2em}} 
\newcommand{\ssep}{\hspace{1em}} 
\newcommand{\todo}[1]{[#1]\marginpar{\mbox{\Huge !}}}
\newcommand{\toask}[1]{[#1]\marginpar{\mbox{\Huge ?}}}

% ----------------------------------------------------------------------
% Special symbols.
\newcommand{\entails}{\vdash}                     % turnstile
\newcommand{\semm}[1]{[\![#1]\!]}                 % semantic brackets
\newcommand{\sem}[2]{\semm{#1}_{#2}}              % semantic brackets with subscript
\newcommand{\sems}[2]{\semm{#1}^{#2}}             % semantic brackets w/ superscript
\newcommand{\semp}{\sem{\,\,\,}{}}                % \sem prototype
\newcommand{\eqbyrule}[1]{\stackrel{#1}{=}}       % annotated =
\newcommand{\FV}{{\rm FV}}                        % free variables
\newcommand{\downdeal}{{\downarrow}}              % downdeal
\newcommand{\updeal}{{\uparrow}}                  % updeal
\newcommand{\chk}{\raisebox{.2em}{\tiny$\surd$}}  % check in case distinctions
\newcommand{\ih}{{\rm (IH)}}                      % ind. hyp.
\newcommand{\tensor}{\otimes}
\newcommand{\bk}{\discretionary{}{}{}}            % optional break
\newcommand{\loli}{\multimap}                     % linear implication

% ----------------------------------------------------------------------
% BNF.
\newcommand{\bnf}{::=}                                   % ::= in bnf
\newcommand{\bor}{\ \ \rule[-.75ex]{.01in}{2.75ex}\ \ }  % | in bnf

% ----------------------------------------------------------------------
% Prettier symbols.
\renewcommand{\leq}{\leqslant}          
\renewcommand{\geq}{\geqslant}  

% ----------------------------------------------------------------------
% Quipper specifics. 
\newcommand{\spec}{\mathtt{Spec}}                % Specification of a type
\newcommand{\true}{\mathtt{True}}                % Boolean: True
\newcommand{\false}{\mathtt{False}}              % Boolean: False
\newcommand{\binding}{\mathfrak{b}}              % Arbitrary binding

% ----------------------------------------------------------------------
% General Information.
\title{\textbf{Proto-Quipper 0.2}}
\date{}
\author{}


% ----------------------------------------------------------------------
\begin{document} 

\maketitle

\begin{abstract}
These notes contain an extension of Proto-Quipper 0.1 with a new 
modality $R$ accounting for reversibility. As a result, it is no longer 
the case that all circuits are reversible. The proofs of type safety 
were carried through without any problems.
\end{abstract}

\noindent\hrulefill
\tableofcontents

\

\noindent\hrulefill


\section{Introduction}

\textbf{Quipper} is an \emph{embedded language}. This means, roughly, that Quipper 
consists of definitions made within a \emph{host language}: \textbf{Haskell}. 
One consequence of this situation, is that Quipper inherits Haskell's type system. 
This is not ideal because Haskell's type system lacks features that are 
desirable for quantum computing. One important such property is linearity. 
However, we can hope that this situation is only temporary. In the long run, 
Quipper could move from an embedded to a \emph{stand alone} language, with a 
dedicated linear type system.

\textbf{Proto-Quipper} is a first step towards this goal. The idea behind the 
development of Proto-Quipper is to reconstruct Quipper piece by piece, starting 
with a core language and extending it progressively. These notes represent a first 
attempt at defining the core language. Consequently, we have focused on defining 
a linearly typed language with the ability to generate, and act on, quantum circuits.  

Proto-Quipper is based on the \textbf{Quantum Lambda Calculus} (QLC). In the QLC, 
the reduction relation is defined on \emph{closures}. A closure is essentially a 
pair $[C,t]$ consisting of a term $t$ and a \emph{quantum state} $C$. The state is 
a unit vector in a complex Hilbert space and $t$ is a term whose free variables are 
linked to the qubits composing $C$. The quantum state is held in a 
\emph{quantum device} which is capable of performing certain operations (applying 
unitaries, measuring qubits,\ldots). The reduction in the QLC is then defined as a 
probabilistic rewrite procedure on these closures. Typically, the reduction will be 
classical until a redex involving a quantum constant is reached. At this point, 
the quantum device will be instructed to perform the appropriate quantum operation. 
For example: ``Apply a Hadamard gate to qubit number 3". 

The reduction in Proto-Quipper will similarly be defined on closures. However, a 
closure $[C,t]$ will now consist of a term $t$ and a \emph{circuit state} $C$. This 
state represents the circuit being currently built. Instead of having a quantum device 
capable of performing quantum operations, we will therefore assume that we have a 
\emph{circuit constructor} capable or performing certain circuit building operations 
(appending gates,\ldots). The reduction will then be defined as a non-probabilistic rewrite 
procedure on closures. As in the QLC, some redexes will affect the state by sending 
instructions to the circuit constructor. For example: ``Append a Hadamard gate to 
wire number 3".


\section{The Language}

\begin{definition} 
The \emph{types} of Proto-Quipper are defined by:
\begin{center}
\begin{tabular}{rl}
$\mathtt{Type}~A,B~\bnf$ & $ qubit \bor 1 \bor Bool \bor A\tensor B \bor A\loli B 
\bor !A \bor Circ (T,U).$\\
\end{tabular}
\end{center}
Among the types, we single out the subset of \emph{QData types}:
\begin{center}
\begin{tabular}{rcl}
$\mathtt{QDataType}~T,U$ & $\bnf$ & $qubit \bor 1 \bor T \tensor U.$
\end{tabular}
\end{center}
\end{definition}

The types $1,Bool, A\x B, A\loli B$ and $!A$ are intended to have their usual 
meaning, i.e., as in linear logic. In particular, $!A$ is the set of 
\emph{duplicable} or \emph{reusable} elements of type $A$. We will 
sometimes write $!^nA$, with $n\in\N$, to mean: 
\[
\underbrace{!\ldots !}_{n} A.
\]

The elements of $qubit$ are (quantum) wire identifiers. The QData types are 
circuit endpoints, which consist of tuples of wire identifiers. The type 
$Circ(T,U)$ is the set of all circuits having $T$-type input and $U$-type 
output. 

\begin{definition}
Assume three countable sets: a set $\mathcal{V}$ of \emph{variables}, a set 
$\mathcal{Q}$ of \emph{quantum addresses} and a set $\mathcal{C}$ of 
\emph{circuit constants}. The \emph{terms} of Proto-Quipper are defined by:
\begin{center}
\begin{tabular}{rcl}
$\mathtt{Term}~a,b,c$ & $\bnf$ & $x \bor q \bor (t,C,a) \bor \true \bor \false 
  \bor \p{a,b} \bor * \bor$ \\[0.05in]
& & $ab \bor \lambda x.a \bor box^T \bor unbox \bor rev \bor $\\[0.05in]
& & $if ~a~ then ~b~ else ~c \bor let ~\atuple{x,y}=a~ in ~b.$
\end{tabular}
\end{center}
Among the terms, we single out the subset of \emph{QData terms}:
\begin{center}
\begin{tabular}{rcl}
$\mathtt{QDataTerm}~t,u$ & $\bnf$ & $q \bor * \bor \atuple{t,u}.$
\end{tabular}
\end{center}
Moreover, we assume that $\mathcal{C}$ is equipped with two functions 
$In,Out\from \mathcal{C}\to\mathcal{P}_f(\mathcal{Q})$ and that $\mathcal{Q}$ 
is well-ordered.
\end{definition}

Again, the meaning of most terms is intended to be the usual one, i.e., as in any typed 
lambda calculi. For example, $\lambda x.a$ is the function which assigns $a$ to $x$. 
For the remaining terms we have:
\begin{itemize}
  \item $q$ is a wire identifier,
  \item $(t,C,a)$ is a circuit with $t$ as an input and $a$ as an output,
  \item $box^T$ is a constant used to turn a circuit-producing function into a circuit,
  \item $unbox$ is a constant used to turn a circuit into a circuit-producing function and
  \item $rev$ is a constant used to reverse circuits.
\end{itemize}

Note that the term $box^T$ depends on the type $T$. This \emph{Church-style} typing of the 
language is the reason why types were introduced before terms. Note also that in the above 
syntax, $box^T$, $unbox$ and $rev$ are terms in their own right. In earlier versions of 
these notes, they were term constructors. The current presentation  was chosen because it 
allows for a more concise presentation of the operational semantics. Finally, remark that 
in a term like $(t,C,a)$, $t$ is a QData term but $a$ isn't. The type system to 
be introduced later will guarantee that even though $a$ is not yet a QData term 
it will eventually reduce to one.

The \emph{values} are the terms whose execution is finished.

\begin{definition}
The \emph{Values} of Proto-Quipper are defined by:
\begin{center}
\begin{tabular}{rcl}
$\mathtt{Value}$ $v,w$ & $\bnf$ & $x \bor q \bor (t,C,u) \bor \true \bor \false 
\bor \atuple{v,w} \bor$ \\
& & $* \bor \lambda x.a  \bor box^T \bor unbox \bor rev.$
\end{tabular}
\end{center}
\end{definition}

The notational conventions used throughout these notes are:
\begin{itemize}
  \item $x,y,z,\ldots$ for variables,
  \item $q_1,q_2,q_3,\ldots$ for quantum addresses,
  \item $C,D,\ldots$ for circuit constants,
  \item $A,B,\ldots$ for types,
  \item $a,b,c,\ldots$ for terms,
  \item $S,T,U,\ldots$ for QData types,
  \item $s,t,u,\ldots$ for QData terms and
  \item $v,w,\ldots$ for values.
\end{itemize}
Additions to these conventions will be provided throughout.


\section{Operations on Types and Terms}

We now introduce some useful operations on types and terms. 

\begin{definition}
The set of \emph{free variables} of a term $a$, written $FV(a)$, 
is defined inductively as follows:
\begin{itemize}
  \item $FV(x)=\s{x}$,
  \item $FV(\atuple{a,b})=FV(a)\cup FV(b)$,
  \item $FV(ab)=FV(a)\cup FV(b)$,
  \item $FV(\lambda x.a)=FV(a)\setminus\s{x}$,
  \item $FV(if ~a~ then ~b~ else ~c) = FV(a) \cup FV(b) \cup FV(c)$,
  \item $FV(let ~\atuple{x,y}=a ~in~ b)= FV(a)\cup (FV(b)\setminus \s{x,y})$,
  \item $FV((t,C,a))= FV(a)$ and
  \item $FV(a)=\emptyset$ in all remaining cases.
\end{itemize}
\end{definition}

The above definition of free variable extends the usual one. Note that the 
free variables of a term of the form $(t,C,a)$ are the free variables of $a$. This
is justified since no variables ever appear in the QData term $t$. The notions 
of $\alpha$-equivalence, capture-avoiding substitution, etc. are defined in 
a straightforward manner. By analogy with the free variables of a term, we 
introduce a notion of \emph{quantum address of term}.

\begin{definition}
The set of \emph{quantum addresses} of a term $a$, written $FQ(a)$, is defined 
inductively as follows:
\begin{itemize}
  \item $FQ(q)=\s{q}$,
  \item $FQ(\atuple{a,b})=FQ(a)\cup FQ(b)$,
  \item $FQ(ab)=FQ(a)\cup FQ(b)$,
  \item $FQ(\lambda x.a)=FQ(a)$,
  \item $FQ(if ~a~ then ~b~ else ~c) = FQ(a) \cup FQ(b) \cup FQ(c)$,
  \item $FQ(let ~\atuple{x,y}=a ~in~ b)= FQ(a)\cup FQ(b)$ and
  \item $FQ(a)=\emptyset$ in all remaining cases.
\end{itemize}
\end{definition}

Remark that $FQ((t,C,a))=\emptyset$. This reflects the idea that the quantum 
addresses appearing in $t$ and $a$ are ``bound" in $(t,C,a)$. 

To append circuits, we will need to be able to express the way in which wires should 
be connected. For this, we use the notion of a \emph{binding}.

\begin{definition}
A \emph{binding} is a bijection $\mathcal{Q}\to \mathcal{Q}$.
\end{definition}

% For now it is simply more convenient to have binding be bijections.
% In fact they could be functions and additional conditions could
% be added elsewhere.

Adding to the list of notational conventions above, we will write $\binding$ for an 
arbitrary binding. 

\begin{definition}
If $a$ is a term with $FQ(a)=\s{q_1,\ldots,q_n}$ and $\mathfrak{b}$ is a binding, 
then $\mathfrak{b}(a)$ is the following term:
\[
\mathfrak{b}(a)= a[\mathfrak{b}(q_1)/q_1,\ldots ,\mathfrak{b}(q_n)/q_n].
\]
\end{definition}

\begin{definition}
The partial function $bind: \mathtt{QDataTerm}^2\to \pow (\mathcal{Q}^2)$ is 
defined recursively as follows:
\[
bind (t,u)= \left\{
  \begin{array}{ll}
    \emptyset & \mbox{if}~~ t=u=*, \\
    \s{(q_1,q_2)} & \mbox{if}~~ t=q_1 \mbox{ and } u=q_2, \\        
    bind (t_1,u_1) \cup bind (t_2,u_2) & \mbox{if}~~ t=\atuple{t_1,t_2} \mbox{ and } 
                                                     u=\atuple{u_1,u_2}, \\
    \mbox{undefined} & \mbox{in all remaining cases.}
  \end{array}
\right.
\]
\end{definition}

\begin{remark}
\label{bind_extension}
If $t$ and $u$ are QData terms for which $bind$ is defined, we can extend $bind(t,u)$ to a 
binding $\binding$ by setting $\binding (q_1) = q_2$ if $\tuple{q_1,q_2}\in bind (t,u)$ 
and $\binding (q_1) = q_1$ otherwise.
\end{remark}

\begin{definition}
Let $T$ be a QData type and $X$ be a proper subset of $\mathcal{Q}$. An \emph{$X$-specimen} 
for $T$ is QData term written $\spec_X(T)$ defined by induction as follows:
\begin{itemize}
  \item $\spec_X(1)=*$,
  \item $\spec_X(qubit)=q$ where $q$ is the smallest quantum index of $\mathcal{Q}\setminus X$,
  \item $\spec_X(T\tensor U)=\atuple{t,u}$ where $t=\spec_X(T)$ and $u=\spec_{X\cup FQ(t)}(U)$.  
\end{itemize}
\end{definition}

% The definition of specimen uses the fact that $\mathcal{Q}$ is well-ordered.

Informally, an $X$-specimen for $T$ is a QData term $t$ that is 
``fresh" with respect to the quantum addresses appearing in $X$.
If $X$ is clear, we simply write $\spec (T)$.


\section{The Type System}

\begin{definition}
The \emph{subtyping relation} $<:$ is the smallest relation on types satisfying 
the rules given in figure~\hyperref[subtyping_congruences]{\ref*{subtyping_congruences}}.
\end{definition}

\begin{figure}[!ht]
\begin{mdframed}
\[
  \infer[]{nqubit <: qubit}{}
~~~~
  \infer[]{1 <: 1}{}
~~~~
  \infer[]{Bool <: Bool}{}
\]
\[
  \infer[]{ (A_1\tensor A_2) <:  (B_1 \tensor B_2)}{A_1<:B_1~~A_2<:B_2}
~~~~
  \infer[]{ (A'\loli B) <: (A\loli B')}{A<:A'~~B<:B'}
\]
\[
  \infer[]{ Circ(A', B) <:  Circ(A, B')}{A<:A'~~B<:B'}
\]
\[
  \infer[]{ !^nA <: !^mB}{
    A<:B
    &
    (m=0)\vee (n\geq 1)
  }
\]
% A not as nice version of the subtping rules:
%
%\[
%  \infer[]{!^nqubit <: !^mqubit}{}
%~~~~
%  \infer[]{!^n1 <: !^m1}{}
%~~~~
%  \infer[]{!^nBool <: !^mBool}{}
%\]
%\[
%  \infer[]{!^n (A_1\tensor A_2) <: !^m (B_1 \tensor B_2)}{A_1<:B_1~~A_2<:B_2}
%~~~~
%  \infer[]{!^n (A'\loli B) <: !^m (A\loli B')}{A<:A'~~B<:B'}
%\]
%\[
%  \infer[.]{!^n Circ(A', B) <: !^m Circ(A, B')}{A<:A'~~B<:B'}
%\]
\end{mdframed}
\caption{Subtyping rules.}
\label{subtyping_congruences}
\end{figure}


\begin{remark}
\label{equiv_subtype_arithm}
The following are equivalent:
\begin{itemize}
  \item $(m=0)\wedge (n\geq 1)$,
  \item $(m\geq 1) \implies (n\geq 1)$.
\end{itemize}
\end{remark}

\begin{remark}
\label{subtyping_shape}
If $A<:B$ then:
\begin{enumerate}
  \item if $A\in\s{qubit, 1, Bool}$, then $A=B$;
  \item if $A=A_1\x A_2$, then $B=B_1\x B_2$, 
  $A_1<:B_1$ and $A_2<:B_2$;
  \item if $A=A_1\loli A_2$, then $B=B_1\loli B_2$, 
  $B_1<:A_1$ and $A_2<:B_2$;
  \item if $A=Circ(A_1, A_2)$, then $B=Circ(B_1,B_2)$, 
  $B_1<:A_1$ and $A_2<:B_2$;
  \item if $B=!B'$, then $A=!A'$ and $A'<:B'$; \label{subtype_bang}
  \item if $A$ is not of the form $!A'$, then $B$ is not 
  of the form $!B'$.
\end{enumerate}
\end{remark}

\begin{proposition}
The subtyping relation is reflexive and transitive.
\end{proposition}

\begin{proof}~
\begin{description}
  \item[Reflexivity] We show that for every type $A$ it is the case 
  that $A<:A$ by induction on $A$. If $A$ is one of the base types, 
  then the result holds by application of the corresponding subtyping
  rule. If $A$ is composite type, the result holds by induction.
  \item[Transitivity] We show that for all types $A,B,C$, if $A<:B$ 
  and $B<:C$, then $A<:C$. Let $\pi$ and $\pi'$ be the derivations of
  $A<:B$ and $B<:C$ respectively. We prove the result by induction on 
  $\pi$. If $\pi$ is of height 1, then $A$ is one of the base types 
  and $A=B$. By remark, this implies that $A=B=C$ and we can 
  conclude by applying the appropriate base rule. If $\pi$ is of 
  height $n>1$ and the last rule of $\pi$ is $\x$, then $A=A_1\x A_2$,
  $B=B_1\x B_2$, $A_1<: B_1$ and $A_2<: B_2$. By remark, this implies 
  that $C=C_1\x C_2$. The last rule of $\pi'$ must therefore have been 
  $\x$ and we have $B_1<: C_1$ and $B_2 <: C_2$. By the induction 
  hypothesis, we get $A_1<: C_1$ and $A_2<: C_2$ and we can conclude 
  by applying $\x$. The $\loli$ and $Circ$ cases are treated similarly. 
  If the last rule of $\pi$ is $!$
\end{description}
\end{proof}

\begin{definition}
A \emph{typing context} is a finite set $\s{x_1:A_1,\ldots,x_n:A_n}$ of 
pairs of a variable and a type, such that no variable occurs more than 
once. A \emph{quantum context} is a finite set of quantum variables.
The expressions of the form $x:A$ in a typing context are called 
\emph{type declarations}.	
\end{definition}

We write $\Gamma$ or $\Delta$ for a typing context and $Q$ for a quantum 
context. If $\Gamma =\s{x_1:A_1,\ldots,x_n:A_n}$ is a typing context, then 
$|\Gamma|=\s{x_1,\ldots,x_n}$, $\Gamma (x_i)=A_i$ and we write $!\Gamma$ 
if $\Gamma(x_i)=!A_i'$ for every $i$. The union of two contexts 
$|\Gamma|\cap  |\Gamma'|=\emptyset$ is denoted by $\Gamma,\Gamma'$, and 
similarly for quantum contexts. Finally, we extend the subtyping relation 
to typing contexts as follows: $\Gamma <: \Gamma'$ if, and only if, 
$|\Gamma | = |\Gamma'|$ and $\Gamma (x_i)<: \Gamma' (x_i)$ for every 
$i$.

\begin{definition}
Let $T,U$ be QData types and $m,n\in\N$. For each of the constants, $box^T$, 
$unbox$ and $rev$ we introduce a type as follows:
\begin{itemize}
  \item $A_{box^T}(T,U,n)=!(T\loli U)\loli !^nCirc(T,U)$,
  \item $A_{unbox}(T,U,n)=Circ(T,U)\loli !^n(T\loli U)$ and
  \item $A_{rev}(T,U,n)=Circ(T,U) \loli !^nCirc(U,T)$.
\end{itemize}
\end{definition}

\begin{definition}
A \emph{typing judgment} is an expression of the form:
\[
\Gamma ; Q \entails a:A
\] 
where $\Gamma$ is a typing context, $Q$ is a quantum context, 
$a$ is a term and $A$ is a type. A typing judgment is \emph{valid} if it can 
be inferred from the rules given in figure~\hyperref[typing_rules]{\ref*{typing_rules}}. 
In the rule $(cst_c)$, $c$ ranges over the set $\s{box^T, unbox, rev}$.
\end{definition}

Note that in the typing judgements of Proto-Quipper, quantum addresses and variables are kept 
separate. As a result, we don't have to specify that $q:qubit$ for every quantum address 
$q$ since the typing rules implicitly enforce this. However, when Proto-Quipper will be 
equipped with the ability to manipulate quantum \emph{and} classical wires, the type of a 
wire might have to be explicitly stated.

% Another reason for this is because in the reduction rule for
% (t,c,u) we need to be able to construct, e.g., t in a context
% that is empty but for the quantum addresses appearing in t.

\begin{figure}[!ht]
\begin{mdframed}
\[
\infer[(ax_c)]{!\Delta, x:A;\emptyset\entails x:B}{
  A<:B
}
~~~~
\infer[(ax_q)]{!\Delta;\s{q}\entails q:qubit}{
} 
\]
\[
\infer[(cst_c)]{!\Delta;\emptyset \entails c:B}{
  A_{c}(T,U,n)<:B
} 
~~~~
\infer[(unit)]{!\Delta;\emptyset\entails *:!^n 1}{
}
\]
\[
\infer[(\lambda_1)]{\Gamma;Q\entails \lambda x.b:A\loli B}{
  \Gamma,x:A;Q \entails b:B
}
~~~~
\infer[(\lambda_2)]{!\Delta;\emptyset \entails \lambda x.b:~!^{n+1}(A\loli B)}{
  !\Delta, x:A;\emptyset \entails b:B
}
\]
\[
\infer[(app)]{\Gamma_1,\Gamma_2, !\Delta;Q_1,Q_2\entails ca:B}{
  \Gamma_1, !\Delta;Q_1\entails c:A\loli B 
  &
  \Gamma_2, !\Delta ;Q_2\entails a:A 
}
\]
\[
\infer[(\x\mbox{-i})]{\Gamma_1,\Gamma_2, !\Delta;Q_1,Q_2\entails \atuple{a,b}:!^n(A\x B)}{
  \Gamma_1, !\Delta;Q_1\entails a:!^nA 
  &
  \Gamma_2, !\Delta ;Q_2\entails b:!^nB
}
\]
% Note that in fact the \x intro rule could have equivalently been written with
% a:!^nA b:!^mB and (a,b):!^o(A\x B) where o=min {n,m}
% Indeed the rules are equivalent via the type isomorphism !!A~!A and the 
% subtyping relation !A<:A. The moral here is that a pair is as 
% reusable as its least reusable component.
\[
\infer[(\x\mbox{-e})]{\Gamma_1,\Gamma_2, !\Delta;Q_1,Q_2\entails let ~\atuple{x,y}=b~in~a:A}{
  \Gamma_1, !\Delta;Q_1\entails b:!^n(B_1\x B_2) 
  &
  \Gamma_2, !\Delta, x:!^nB_1, y:!^nB_2 ;Q_2\entails a:A
}
\]
\[
\infer[(\top)]{!\Delta;\emptyset\entails \true:!^n Bool}{
} 
~~~~
\infer[(\bot)]{!\Delta;\emptyset\entails \false:!^n Bool}{
}
\]
\[
\infer[(if)]{\Gamma_1,\Gamma_2, !\Delta;Q_1,Q_2\entails if ~b~ then ~a_1~ else ~a_2:A}{
  \Gamma_1, !\Delta;Q_1\entails b:Bool 
  &
  \Gamma_2, !\Delta;Q_2 \entails a_1:A ~~~ \Gamma_2, !\Delta;Q_2 \entails a_2:A
}
\]
\[
\infer[(circ)]{!\Delta;\emptyset \entails (t,C,a):!^nCirc(T,U)}{
  \emptyset ; Q_1\entails t:T 
  &
  !\Delta ; Q_2\entails a:U 
  &
  In(C)=Q_1 
  &
  Out(C)=Q_2
}
\]
\end{mdframed}
\caption{Typing rules.}
\label{typing_rules}
\end{figure}

In the current type system, any circuit can be reversed. 
This is because the present version of Proto-Quipper does not 
make a distinction between reversible and irreversible circuits. 
Of course, this will have to be modified to accommodate non-reversible 
gates such as measurements.


\section{Properties of the Type System}

We record some properties of the type system, including 
the important \emph{Substitution Lemma}. Note that the typing 
rules enforce a \emph{strict} linearity on variables and quantum addresses. 
In particular, if a quantum address appears in the quantum context of a 
valid typing judgement for a term $a$, then it must belong to the 
quantum addresses of $a$.

\begin{lemma}~
\label{prop_type_syst}
\begin{enumerate}
  \item If $\Gamma; Q \entails a:A$ is valid, 
  then $Q=FQ(a)$.\label{q_context}
  \item If $\Gamma,x:B;Q \entails a:A$ is valid, 
  and $x\notin FV(a)$, then $\Gamma;Q \entails a:A$ is valid.\label{unused_var}
  \item If $\Gamma; Q \entails a:A$ is valid, 
  then $\Gamma, !\Delta ; Q \entails a:A$ is valid.\label{weakening}
  \item If $\Gamma ; Q \entails a:A$ is valid, $\Delta <: \Gamma$
  and $A<:B$, then $\Delta ; Q \entails a:B$ is valid.\label{subtype}
\end{enumerate}
\end{lemma}

\begin{proof}
By induction on the corresponding typing derivation.
\end{proof}

\begin{lemma}
\label{specimen}
If $T$ is a QData type and $X\subset \mathcal{Q}$, then 
$FQ(t)\entails \spec_X(T):T$ is valid.
\end{lemma}

\begin{proof}
We prove the Lemma by induction on $T$.
  \begin{itemize}
    \item If $T=1$, then $\spec_X(T)=*$ and we can use the $(unit)$ rule.
    \item If $T=qubit$, then $\spec_X(T)=q$ for some quantum address $q$ and we can 
          use the $(ax_q)$ rule.
    \item If $T=T_1\x T_2$, then $\spec_X(T)=\p{t_1,t_2}$ where $t_1=\spec_X(T_1)$ 
          and $u=\spec_{X\cup FQ(t_1)}(T_2)$. By the induction hypothesis, both 
          $FQ(t_1)\entails t_1:T_1$ and $FQ(t_2)\entails t_2:T_2$ are valid typing 
          judgements. We can therefore conclude by applying the $(\tensor\mbox{-i})$ rule.
  \end{itemize}
\end{proof}

\begin{lemma}
\label{binding_judgement}
If $\Gamma;Q \entails a:A$ is valid and $\mathfrak{b}$ is a 
binding, then $\Gamma;\binding (Q) \entails \binding(a):A$ is valid.
\end{lemma}

\begin{proof}
By induction on the typing derivation of $\Gamma;Q \entails a:A$.
\end{proof}

\begin{lemma}
\label{context_value}
If $v\in\mathtt{Val}$ and $\Gamma,!\Delta;Q \entails v:!A$ is valid, 
then $\Gamma=Q=\emptyset$.
\end{lemma}

\begin{proof}
By induction on the typing derivation of $\Gamma,!\Delta;Q \entails v:!A$. For 
the $(ax_c)$ case, use Lemma~\hyperref[subtype_bang]{\ref*{subtyping_shape}.\ref*{subtype_bang}}.
\end{proof}

\begin{lemma}
\emph{(Substitution)}
\label{substitution}
If $v\in\mathtt{Val}$ and both $\Gamma',!\Delta;Q' \entails v:B$ and 
$\Gamma,!\Delta,x:B;Q \entails a:A$ are valid typing judgements, 
then $\Gamma,\Gamma',!\Delta;Q,Q' \entails a[v/x]:A$ is also valid.
\end{lemma}

\begin{proof}
Let $\pi_1$ and $\pi_2$ be the typing derivations  of 
$\Gamma,!\Delta,x:B;Q \entails a:A$ and  $\Gamma',!\Delta;Q' \entails v:B$ 
respectively. We prove the Lemma by induction on $\pi_1$.
\begin{itemize}
 \item If the last rule of $\pi_1$ is $(ax_c)$ and $a=x$, then $\pi_1$ is
 \[
   \infer[(ax_c)]{!\Delta, x:B;\emptyset\entails x:A}{
     B<:A
   }
 \]
 with $\Gamma=Q=\emptyset$. Then $a[v/x]=v$ and can conclude by applying 
 Lemma~\hyperref[subtype]{\ref*{prop_type_syst}.\ref*{subtype}} to $\pi_2$.
 \item If the last rule of $\pi_1$ is $(ax_c)$ and $a=y\neq x$, then 
 $\pi_1$ is
 \[
   \infer[(ax_c)]{!\Delta, x:!B',y:A';\emptyset\entails y:A}{
     A'<:A
   }
 \]
 with $B=!B'$, $Q=\emptyset$ and $\Gamma =\s{y:A'}$ or $\Gamma=\emptyset$ 
 depending on whether or not $A'$ is duplicable. Therefore $v$ is a value of 
 type $!B'$ and by Lemma~\hyperref[context_value]{\ref*{context_value}}, 
 we know that $\Gamma'=Q'=\emptyset$. Since $a[v/x]=y$ and $x\notin FV(y)$ 
 we can conclude by applying
 Lemma~\hyperref[unused_var]{\ref*{prop_type_syst}.\ref*{unused_var}} to $\pi_1$.
 \item If the last rule of $\pi_1$ is one of $(ax_q)$, $(cst_c)$, 
 $(unit)$, $(\top)$ and $(\bot)$, and $a$ is the corresponding constant, 
 then $x\notin FV(a)$ and  $x$ must be declared of some type $!B'$.
 We can therefore reason as in the previous case. 
 \item If the last rule of $\pi_1$ is $(\lambda_1)$ and $a=\lambda y.b$, then $\pi_1$ is
  \[
   \infer[(\lambda_1)]{\Gamma,!\Delta,x:B;Q\entails \lambda y.b:A_1\loli A_2}{
     \deduce[]{\Gamma, !\Delta,x:B,y:A_1;Q \entails b:A_2}{
       \vdots
     }
   }
 \]
 with $A=A_1\loli A_2$. By the induction hypothesis, 
 $\Gamma, \Gamma',!\Delta,y:A_1;Q,Q' \entails b[v/x]:A_2$ is valid and we can conclude
 by applying $(\lambda_1)$.
 \item If the last rule of $\pi_1$ is $(\lambda_2)$ and $a=\lambda y.b$, 
 then $\pi_1$ is
 \[
  \infer[(\lambda_2)]{!\Delta, x:!B';\emptyset \entails \lambda y.b:~!^{n+1}(A_1\loli A_2)}{
    \deduce[]{!\Delta, x:!B',y:A_1;\emptyset \entails b:A_2}{
      \vdots
    }
  }
 \]
 with $A=!^{n+1}(A_1\loli A_2)$ and $B=!B'$. Hence $v$ is a value of type $!B'$ and  
 by Lemma~\hyperref[context_value]{\ref*{context_value}}, we know that
 $\Gamma'=Q'=\emptyset$. The induction hypothesis therefore implies that
 $!\Delta,y:A_1;\emptyset \entails b[v/x]:A_2$ is valid and we can conclude
 by applying $(\lambda_2)$.
 \item If the last rule of $\pi_1$ is $(app)$, and $a=ca'$, then $\pi_1$ can 
 be of one of three forms depending on $B$. If $B$ is duplicable, then $\pi_1$ is
 \[
 \infer[(app)]{\Gamma_1,\Gamma_2,!\Delta,x:!B';Q_1,Q_2\entails ca':A}{
    \deduce[]{\Gamma_1, x:!B',!\Delta;Q_1\entails c:A'\loli A}{
      \vdots
    }
    &
    \deduce[]{\Gamma_2, x:!B',!\Delta ;Q_2\entails a':A' }{
      \vdots
    }
 }
 \]
 with $B=!B'$. Using 
 Lemma~\hyperref[context_value]{\ref*{context_value}} again, we 
 know that $\Gamma'=Q'=\emptyset$. The induction hypothesis therefore 
 implies that $\Gamma_1,!\Delta;Q_1\entails c[v/x]:A'\loli A$ and 
 $\Gamma_2,!\Delta ;Q_2\entails a'[v/x]:A'$ are valid and we can conclude 
 by applying $(app)$. If, instead, $B$ is non-duplicable, then the declaration 
 $x:B$ can only appear in one branch of the derivation. This means that $\pi_1$ 
 is either 
  \[
 \infer[(app)]{\Gamma_1,\Gamma_2,!\Delta,x:B;Q_1,Q_2\entails ca':A}{
    \deduce[]{\Gamma_1, x:B,!\Delta;Q_1\entails c:A'\loli A}{
      \vdots
    }
    &
    \deduce[]{\Gamma_2, !\Delta ;Q_2\entails a':A'}{
      \vdots
    }     
 }
 \]
 or
  \[
 \infer[(app).]{\Gamma_1,\Gamma_2,!\Delta,x:B;Q_1,Q_2\entails ca':A}{
    \deduce[]{\Gamma_1, !\Delta;Q_1\entails c:A'\loli A}{ 
      \vdots
    }
    &
    \deduce[]{\Gamma_2, x:B,!\Delta ;Q_2\entails a':A'}{
      \vdots
    }     
 }
 \]
 In the first case, the induction hypothesis implies that 
 $\Gamma_1,\Gamma' !\Delta;Q_1,Q'\entails c[v/x]:A'\loli A$ is valid and 
 we can conclude by $(app)$. The second case is treated analogously.
 \item If the last rule of $\pi_1$ is on of $(\x\mbox{-i})$, $(\x\mbox{-e})$ 
 and $(if)$, and $a$ is the corresponding term, then we can reason as above
 by considering in turn the case where $B$ is duplicable and the case where
 $B$ is non-duplicable.
 \item If the last rule of $\pi_1$ is $(circ)$, and $a=(t,C,a')$, then 
 $\pi_1$ is
 \[
 \infer[(circ)]{!\Delta,x:!B';\emptyset \entails (t,\mathbf{c},a'):!^nCirc(T,U)}{
    \deduce[]{\emptyset ; Q_1\entails t:T}{
      \vdots
    }    
    &
    \deduce[]{!\Delta,x:!B' ; Q_2\entails a':U}{
      \vdots
    }     
    &
    In(C)=Q_1 
    &
    Out(C)=Q_2
 }
 \]
 with $A=!^nCirc(T,U)$ and $B=!B'$ for some types $T$, $U$ and $B'$. Using  
 Lemma~\hyperref[context_value]{\ref*{context_value}} again, we know 
 that $\Gamma'=Q'=\emptyset$. The induction hypothesis therefore implies that
 $!\Gamma; Q_2\entails a'[v/x]:U$ is valid and we can conclude
 by applying $(circ)$.
\end{itemize}
\end{proof}

\section{Circuit Constructors}

As mentioned in the introduction, the reduction relation for 
Proto-Quipper is defined in the presence of a \emph{circuit
constructor}. This is a device capable of 
performing certain basic circuit building operations. It is 
not necessary to have a detailed description of the inner 
workings of this device. In fact, all that is required for 
the definition of Quipper's operational semantics is the 
existence of some primitive operations. These are wrapped 
in a structure that we take to axiomatize circuit constructors. 

\begin{definition}
\label{circuit_constructor}
A \emph{circuit constructor} consists of a pair of countable sets $\atuple{Q,S}$ 
together with the following maps:
\begin{itemize}
  \item $\mathtt{New}\from \pow_f(Q) \to S$,
  \item $\mathtt{In}\from S\to \pow_f(Q)$,
  \item $\mathtt{Out}\from S\to \pow_f(Q)$,
  \item $\mathtt{Rev}\from S \to S$,
  \item $\mathtt{Encap}\from S \to  S$,
  \item $\mathtt{Unencap}\from S\times S\times Q^Q \to S \times Q^Q$ and
\end{itemize}
verifying the following conditions:
\begin{enumerate}
  \item $\mathtt{Rev}\circ\mathtt{Rev}=1_S$,
  \item $\mathtt{In}\circ\mathtt{Rev}= \mathtt{Out}$ and 
        $\mathtt{Out}\circ\mathtt{Rev}= \mathtt{In}$\label{in_out_rev},
  \item $\mathtt{In}\circ\mathtt{New} =1_{\mathcal{P}_f(\mathcal{Q})}$,
  \item $\mathtt{In}\circ \mathtt{Encap} =\mathtt{In}$ and $\mathtt{Out}\circ \mathtt{Encap} =\mathtt{Out}$,
  \item $\mathtt{In}\circ\pi_1\circ\mathtt{Unencap}=\mathtt{In}\circ\pi_1$,\label{Unencap_In}
  \item if $\mathtt{Unencap} (C,D,\binding)=(C',\binding')$ then $b$ and $b'$ are bijective and\label{Unencap_cond}
    \begin{enumerate}
      \item $\binding[\mathtt{In}(D)]\subseteq \mathtt{Out} (C)$,\label{Unencap_cond_1}
      \item $\binding'[\mathtt{Out}(D)]\subseteq\mathtt{Out}(C')$ and\label{Unencap_cond_2}
      \item $\mathtt{Out}(C')=\binding'(\mathtt{Out}(D)),(\mathtt{Out}(C)\setminus\binding^{-1}(\mathtt{In}(D)))$.\label{Unencap_cond_3}
    \end{enumerate}
\end{enumerate}
\end{definition}

If $\p{Q,S}$ is a circuit constructor, we call \emph{circuit states} 
the elements of $S$ and \emph{wire identifiers} the elements of $Q$. 

The functions that equip a constructor represent operations on 
circuits. The definition of circuit used by a particular 
constructor needn't be specified here. However, it can be helpful 
to keep in mind the usual representation of quantum circuits when 
thinking about constructors. For example, one can think of 
$\mathtt{New}(q_1,q_2,q_3)$ as the circuit given in figure~\hyperref[rep_new]{\ref*{rep_new}}.

\begin{figure}[!ht]
\[
\mbox{
\Qcircuit @C=1em @R=2.7em {
& \qw & \ustick{q_1} \qw & \qw & \qw \\
& \qw & \ustick{q_2} \qw & \qw & \qw \\
& \qw & \ustick{q_3} \qw & \qw & \qw 
}
}
\]
\caption{A representation of $\mathtt{New}\s{q_1,q_2,q_3}$.}
\label{rep_new}
\end{figure}

The $\mathtt{Unencap}$ function can similarly be explained in terms of 
the usual graphical language for circuits. It inputs two circuit states, 
say $C$ and $D$, together with a map $\binding\from Q\to Q$ and returns 
a new circuit $C'$ together with a new map $\binding'$. $C'$ can be 
seen as the composition of $C$ and $D$ along $\binding$. An illustration of 
this process is given in figure~\hyperref[rep_unencap]{\ref*{rep_unencap}}. 
The function $\binding$ is used to specify along which wires to compose $C$ 
and $D$ while the function $\binding'$ updates the wire names post-composition.

\begin{figure}[!ht]
\[
\mbox{
\Qcircuit @C=.5em @R=2.7em {
& & & & & 
     & & & & & & \push{\binding} & & &
     & & & 
     & & & & & & \push{\binding'} & & & \\
&\qw &\ustick{q_1}\qw &\qw &\qw &\multigate{4}{~~~C~~~} 
     &\qw &\qw &\ustick{q_1}\qw &\qw &\qw &\qw &\qw &\qw &\ustick{q_1'}\qw
     &\qw &\qw &\multigate{2}{~~~D~~~} 
     &\qw &\qw &\ustick{p_1}\qw &\qw &\qw &\qw &\qw &\qw &\ustick{p_1'}\qw \\
&\qw &\ustick{q_2}\qw &\qw &\qw &\ghost{~~~C~~~}       
     &\qw &\qw &\ustick{q_2}\qw &\qw &\qw &\qw &\qw &\qw &\ustick{q_2'}\qw
     &\qw &\qw &\ghost{~~~D~~~} 
     &\qw &\qw &\ustick{p_2}\qw &\qw &\qw &\qw &\qw &\qw &\ustick{p_2'}\qw \\     
&\qw &\ustick{q_3}\qw &\qw &\qw &\ghost{~~~C~~~}        
     &\qw &\qw &\ustick{q_3}\qw &\qw &\qw &\qw &\qw &\qw &\ustick{q_3'}\qw
     &\qw &\qw &\ghost{~~~D~~~} 
     &\qw &\qw &\ustick{p_3}\qw &\qw &\qw &\qw &\qw &\qw &\ustick{p_3'}\qw \\     
&\qw &\ustick{q_4}\qw &\qw &\qw &\ghost{~~~C~~~}       
     &\qw &\qw &\ustick{q_4}\qw &\qw &\qw &\qw &\qw &\qw &\qw
     &\qw &\qw &\qw 
     &\qw &\qw &\qw &\qw &\qw &\qw &\qw &\qw &\ustick{q_4}\qw \\     
&\qw &\ustick{q_5}\qw &\qw &\qw &\ghost{~~~C~~~}     
     &\qw &\qw &\ustick{q_5}\qw &\qw &\qw &\qw &\qw &\qw &\qw
     &\qw &\qw &\qw 
     &\qw &\qw &\qw &\qw &\qw &\qw &\qw &\qw &\ustick{q_5}\qw     
     \gategroup{2}{3}{6}{8}{3.3em}{--}
     \gategroup{2}{16}{4}{20}{3.3em}{--}     
     \gategroup{2}{23}{5}{24}{4.5em}{^\}}     
     \gategroup{2}{11}{5}{13}{4.5em}{^\}}     
}}
\]
\caption{A representation of $\mathtt{Unencap} (C,D,\binding)$.}
\label{rep_unencap}
\end{figure}

Note that a gate, is a (very basic) circuit.
The operation of appending a gate to a circuit is therefore subsumed 
by the more general operation of composing circuits. This is why no 
primitive gate-appending operations are specified. Of course, a 
``usable" circuit constructor would come equipped with circuit 
states for each element of some chosen universal gate set.

The function $\mathtt{Encap}$ stores a circuit for later use.
We also say that it \emph{boxes} that circuit. The remaining functions 
are relatively straightforward. The function $\mathtt{In}$ 
(resp. $\mathtt{Out}$) returns the identifiers of the input 
(resp. output) wires of a circuit, while $\mathtt{Rev}$ reverses 
a circuit. 

Proto-Quipper's quantum addresses and circuit constants are supposed 
to be the syntactic representatives of a circuit constructor's wire 
identifiers. This idea is formalized in the following definition.

\begin{definition}
A circuit constructor $\atuple{Q,S}$ is \emph{adequate} if it can 
be equipped with bijections $\mathtt{Address}\from \mathcal{Q} \to Q$ 
and $\mathtt{Name}\from \mathcal{C} \to S$ such that:
\[In=\mathtt{Address}' \circ\mathtt{In}\circ \mathtt{Name} 
~\mbox{ and }~ 
Out=\mathtt{Address}'\circ\mathtt{Out}\circ \mathtt{Name}
\]
where $\mathtt{Address}'$ denotes the lifting of $\mathtt{Address}$ 
from $\mathcal{Q}$ to $\mathcal{P}(\mathcal{Q})$.
\end{definition}

\begin{remark}
\label{structure-transfer}
The existence of the bijections $\mathtt{Address}$ and $\mathtt{Name}$ has 
the following consequences:
\begin{itemize}
  \item $\mathcal{C}$ can be equipped with an involution:
\[
(.)^{-1} = \mathtt{Name}\circ \mathtt{rev} \circ \mathtt{Name}^{-1}
           \from \mathcal{C}\to\mathcal{C}
\]
such that $In(C^{-1})=Out(C)$ and 
$Out(C^{-1})=In(C)$.
  \item If $\binding$ is a binding obtained from QData terms $t,u$ in the 
  way described in remark~\hyperref{bind_extension}, we can use it to define the 
  following map (which we call by the same name):
\[
\binding = \mathtt{Address}\circ \binding\circ \mathtt{Address}^{-1}\from Q\to Q
\]
\end{itemize}

In what follows, we work under the simplifying assumptions that $\mathcal{Q}=Q$, 
$\mathtt{Address}=1_Q$ and $\mathtt{Encap}=1_Q$.
\end{remark}


\section{Operational Semantics}

We are now in a position to define Proto-Quipper's operational 
semantics.

\begin{definition}
Let $\atuple{Q,S}$ be an adequate circuit constructor. A 
\emph{closure} is a pair $[C,a]$ where $C\in S$, $a$ is a 
term and $FQ(t)\subseteq\mathtt{Out}(C)$. 
\end{definition}

% We require $FQ(t)\subseteq\mathtt{Out}(C)$ and not $FQ(t)=\mathtt{Out}(C)$
% because the reduction is defined by induction on the terms. For
% example, to reduce a pair, one must first reduce the left component
% in the context of the current circuit. If we required an equality 
% we couldn't simply do this.

\begin{definition}
The \emph{one-step reduction relation}, written $\to$, is defined 
on closures by the rules given in tables~\hyperref[cong_rules]{\ref*{cong_rules}}, 
\hyperref[classical_rules]{\ref*{classical_rules}} 
and \hyperref[circ_gen_rules]{\ref*{circ_gen_rules}}. 
In the $(box)$ rule, the $C,v$ in $\spec_{C,v}$ stands for $In(C),Out(C),FQ(v)$.
The \emph{reduction relation}, 
written $\to^*$, is defined to be the reflexive and transitive closure of $\to$.
\end{definition}

\begin{figure}[!ht]
\begin{mdframed}
\[
  \infer[(fun)]{[C,ab]\to[C',a'b]}{
    [C,a]\to[C',a']
  }
~~~~~~
  \infer[(arg)]{[C,vb]\to [C',vb']}{
    [C,b]\to [C',b']
  }
\]
\[
  \infer[(right)]{[C,\langle a,b\rangle]\to [C',\langle a,b'\rangle]}{
    [C,b]\to [C',b']
  }
~~~~~~
  \infer[(left)]{[C,\langle a,v\rangle]\to [C',\langle a',v\rangle]}{
    [C,a]\to [C',a']
  }
\]
\[
  \infer[(let)]{[C,let ~ \langle x,y \rangle = a ~in~ b]\to 
                [C', let ~ \langle x,y \rangle = a' ~in~ b]}{
    [C,a]\to [C',a']
  }
\]
\[
  \infer[(cond)]{[C, if ~a~ then ~b~ else ~c~]\to [C', if ~a'~ then ~b~ else ~c~]}{
    [C,a]\to [C',a']
  }
\]
\[
  \infer[(circ)]{[C, (t,D,a)]\to [C, (t,D',a')]}{
    [D,a]\to [D',a']
  }
\]
\end{mdframed}
\caption{The congruence rules.}
\label{cong_rules}
\end{figure}

\begin{figure}[!ht]
\begin{mdframed}
\[
  \infer[(\beta)]{[C,(\lambda x.a)v]\to [C, a[v/x]]}{}
\]
\[
  \infer[(pair)]{[C,let ~\atuple{x,y}=\atuple{v,w}~ in ~a]\to[C,a[v/x,w/y]]}{}
\]
\[
  \infer[(if\mbox{-}\mathtt{F})]{[C,if ~\mathtt{False}~ then ~a~ else ~b] \to [C, b]}{}
\]
\[
  \infer[(if\mbox{-}\mathtt{T})]{[C,if ~\mathtt{True}~ then ~a~ else ~b] \to [C, a]}{}
\]
\end{mdframed}
\caption{The classical rules.}
\label{classical_rules}
\end{figure}

\begin{figure}[!ht]
\begin{mdframed}
\[
  \infer[(box)]{[C,box^T(v)]\to [C,(t,D,a)]}{
    \spec_{C,v}(T)=t
    &
    [\mathtt{new}(FQ(t)), vt] \to [D,a]
  }
\]
\[
  \infer[(unbox)]{[C,(unbox~(u,D,u'))v]\to [C',\binding'(u')]}{
    bind(v,u)=\binding 
    &
    \mathtt{Unencap}(C,D,\binding) = (C',\binding') 
  }
\]
\[
  \infer[(rev)]{[C,rev (t,C,t')]\to [C,(t',C^{-1},t)]}{}
\]
\end{mdframed}
\caption{The circuit generating rules.}
\label{circ_gen_rules}
\end{figure}

The classical reduction rules and the congruence rules (minus $(circ)$) are standard. They 
describe the usual call-by-value reduction strategy. The $(rev)$ rule is 
straightforward. We briefly discuss the remaining rules. The $(box)$ rule is 
to be understood as follows. Start by generating a specimen of type $T$. Then 
apply the function $v$ on the input $t$ in the context of an empty circuit of 
the appropriate arity. Note that while this computation is taking place, the 
state $C$ is not accessible. By the $(circ)$ congruence rule, this computation will 
continue until a value is reached, i.e., a term of the form $(t,D,t')$. At this point, 
the construction of $C$ can resume. Remark that it was necessary to know the type 
$T$ in order to generate the appropriate specimen. This explains the choice of a 
Church-style typing of the language. The $(unbox)$ rule will first generate a binding 
from $v$ and the input $u$ of $D$. Then, it will compose $C$ and $D$ along that 
binding and update the names of the wire identifiers appearing in $u'$ 
according to $\binding'$.

The inductive nature of the reduction rule explains why closures are not
required to satisfy $FQ(a)=\mathtt{Out}(C)$. The requirement that 
$FQ(a)\subseteq \mathtt{Out}(C)$ is justified by the idea that a term should 
not affect a wire outside of $C$. But if we also asked for the opposite 
inclusion, it wouldn't be possible to define an inductive reduction in a 
straightforward way. For example, the reduction of a pair is done 
component-wise: to reduce $\p{a,b}$ one first reduces $b$. The simplest way 
to express this in terms of closures is to carry the whole circuit state along. 
This implies that if both $a$ and $b$ contain wire identifiers, then the 
equality $FQ(a)=\mathtt{Out}(C)$ cannot be satisfied.

Proto-Quipper's reduction is non-probabilistic, in the sense that the right member of 
any reduction rule is a unique closure. This property might be modified 
in later versions of the language if a notion of \emph{dynamic lifting} is added. The 
following proposition establishes that Proto-Quipper's reduction is  
\emph{deterministic}.

>From now on, we always assume an adequate circuit constructor.

\begin{proposition}
\label{determinicity}
If $[C,a]$ is a closure, then at most one reduction rule applies
to it.
\end{proposition}

\begin{proof}
By case distinction on $a$.
\end{proof}


\section{Type Safety}

To establish that Proto-Quipper is a \emph{type safe} language, we must prove that 
it enjoys two properties: \emph{Subject Reduction} and \emph{Progress}. The first 
property guarantees that the type of a term is stable under reduction. As a 
corollary, it also proves that if a term is well-typed, then it never reduces to 
an ill-typed term. The second property shows that a well-typed term never reaches 
an ``error state", which is a closure $[C,t]$ to which no reduction rule applies 
but such that $t\notin \mathtt{Val}$. 

The reduction relation is defined on closures but the typing rules apply to terms. 
We therefore start by extending the notions of typing judgement and validity to 
closures.

\begin{definition}
A \emph{typed closure} is an expression of the form:
\[
\Gamma;Q\entails [C,a]:A,(Q'|Q'')
\]
where $\mathtt{In}(C)=Q'$ and $\mathtt{Out}(C)=Q,Q''$. It is 
\emph{valid} if $\Gamma;Q\entails a:A$ is a valid 
typing judgement.
\end{definition}

\subsection{Subject Reduction}

\begin{lemma}
\label{Inwires}
If $[C,a]\to[C',a']$ then $\mathtt{In}(C)=\mathtt{In}(C')$.
\end{lemma}

\begin{proof}
By case distinction on the rule used in the reduction $[C,a]\to[C',a']$. 
In all but the $(unbox)$ case, the result follows either from the induction 
hypothesis or from the fact that $C=C'$. In the $(unbox)$ case, use 
Definition~\hyperref[Unencap_In]{\ref*{circuit_constructor}.\ref*{Unencap_In}}.
\end{proof}

\begin{proposition}
\label{subject_red_one_step}
If $\Gamma;FQ(a)\entails [C,a]:A,(Q'|Q'')$ is valid typed closure 
and $[C,a]\to [C',a']$, then $\Gamma;FQ(a')\entails [C',a']:A,(Q'|Q'')$ is 
a valid typed closure.
\end{proposition}

\begin{proof}
We prove the proposition by induction on the derivation of the reduction 
 $[C,a]\to[C',a']$. In each case, we start by reconstructing 
the unique typing derivation $\pi$ of $\Gamma;FQ(a)\entails a:A$ and we use 
it to prove that $\Gamma;FQ(a')\entails [C',a']:A,(Q'|Q'')$ is valid. 
By Lemma~\hyperref[Inwires]{\ref*{Inwires}} we never need to 
verify that $\mathtt{In}(C')=Q'$ so that in each case we only 
need to show:
\begin{itemize}
  \item $\mathtt{Out}(C')=FQ(a'),Q''$ and
  \item $\Gamma;FQ(a')\entails a':A$ is valid.
\end{itemize}
Throughout the proof, we write $IH(\pi)$ to denote the proof obtained by applying 
the induction hypothesis to $\pi$.

\begin{description}
\item[Congruence rules:] These rules are treated uniformly. We illustrate the 
$(fun)$ and $(circ)$ cases.
\begin{itemize}
  \item $(fun)$: the reduction rule is
  \[
    \infer[]{[C,cb]\to[C',c'b]}{
      [C,c]\to[C',c']
    }
  \]
  with $a=cb$ and $a'=c'b$. The typing derivation $\pi$ is therefore
  \[
    \infer[]{\Gamma_1,\Gamma_2, !\Delta;FQ(c),FQ(b)\entails cb:A}{
      \deduce[]{\Gamma_1, !\Delta;FQ(c)\entails c:B\loli A}{
        \vdots~\pi_1
      } 
      &
      \deduce[]{\Gamma_2, !\Delta ;FQ(b)\entails b:B}{
        \vdots~\pi_2
      } 
    }
  \]
  and $\Gamma_1,\Gamma_2;FQ(c),FQ(b)\entails [C, cb],(Q'|Q'')$ is valid.
  It follows that $\Gamma_1, !\Delta;FQ(c)\entails [C,c]:B\loli A,(Q'|FQ(b),Q'')$ 
  is valid and, by the induction hypothesis, this implies that 
  $\Gamma_1, !\Delta;FQ(c')\entails [C',c']:B\loli A,(Q'|FQ(b),Q'')$ is also valid.
  In particular, it follows that $\mathtt{Out}(C')=FQ(c'),FQ(b),Q''$. This, 
  together with the following typing derivation,
  \[
    \infer[]{\Gamma_1,\Gamma_2, !\Delta;FQ(c'),FQ(b)\entails c'b:A}{
      \deduce[]{\Gamma_1, !\Delta;FQ(c')\entails c':B\loli A}{
        \vdots~IH(\pi_1)
      } 
      &
      \deduce[]{\Gamma_2, !\Delta ;FQ(b)\entails b:B}{
        \vdots~\pi_2
      } 
    }
  \]  
  shows that $\Gamma_1,\Gamma_2, !\Delta;FQ(c'),FQ(b)\entails c'b:A,(Q',Q'')$ 
  is valid.
  \item $(circ)$: the reduction rule is
  \[
    \infer[(circ)]{[C, (t,D,b)]\to [C, (t,D',b')]}{
      [D,b]\to [D',b']
    }
  \]  
  with $a=(t,D,b)$ and $a'=(t,D',b')$. The typing derivation $\pi$ is therefore
  \[
  \infer[]{!\Delta;\emptyset\entails (t,D,b):!^nCirc(T,U)}{
    \deduce[]{\emptyset ; FQ(t)\entails t:T}{
      \vdots~\pi_1
    } 
    &
    \deduce[]{!\Delta ; FQ(b)\entails b:U}{
      \vdots~\pi_2
    }
    &
    \deduce[]{In(D)=FQ(t)}{
      Out(D)=FQ(b)
    }
  }
  \]  
  and $!\Delta ; \emptyset \entails [C,(t,D,b)]:!^nCirc(T,U),(Q'|Q'')$ is valid. 
  Disregarding $\pi_1$, it follows from the assumptions in the above rule 
  that $!\Delta ; FQ(b)\entails [D,b]:U, (FQ(t)|\emptyset)$ is valid and, 
  by the induction hypothesis, this implies that 
  $!\Delta,FQ(b')\entails [D',b']:U,(FQ(t)|\emptyset)$ is also valid.
  This, together with the following typing derivation,
  \[
  \infer[.]{!\Delta;\emptyset\entails (t,D,b'):!^nCirc(T,U)}{
    \deduce[]{\emptyset ; FQ(t)\entails t:T}{
      \vdots~\pi_1
    } 
    &
    \deduce[]{!\Delta ; FQ(b')\entails b':U}{
      \vdots~IH(\pi_2)
    }
    &
    \deduce[]{In(D')=FQ(t)}{
      Out(D')=FQ(b')
    }
  }  
  \]
  shows that $!\Delta;\emptyset\entails [C,(t,D',b')] :!^nCirc(T,U),(Q'|Q'')$ 
  is valid.
\end{itemize}
\item[Classical rules:] These rules are also treated uniformly, we illustrate 
the $(\beta)$ case.
\begin{itemize}
  \item $(\beta)$: the reduction rule is
  \[
    \infer[(\beta)]{[C,(\lambda x.b)v]\to [C, b[v/x]]}{}
  \]  
  with $a=(\lambda x.b)v$ and $a'=b[v/x]$. The typing derivation $\pi$ is 
  therefore
  \[
    \infer[]{\Gamma_1,\Gamma_2,!\Delta;FQ(b),FQ(v)\entails (\lambda x.b)v:A}{
      \infer[]{\Gamma_1,!\Delta;FQ(b)\entails \lambda x.b:B\loli A}{
        \deduce[]{\Gamma_1,!\Delta,x:B;FQ(b) \entails b:A}{
          \vdots~\pi_1
        }
      }
      &
      \deduce[]{\Gamma_2,!\Delta;FQ(v)\entails v:B}{
        \vdots~\pi_2
      }      
    }
  \]  
  and $\Gamma_1,\Gamma_2,!\Delta;FQ(b),FQ(v)\entails (\lambda x.b)v:A,(Q'|Q'')$ 
  is valid. By Lemma~\hyperref[substitution]{\ref*{substitution}}, 
  $\Gamma_1,\Gamma_2,!\Delta;FQ(b),FQ(v)\entails b[v/x]:A$ is a valid typing 
  judgement which implies that 
  $\Gamma_1,\Gamma_2,!\Delta;FQ(b),FQ(v)\entails [C,b[v/x]]:A,(Q'|Q'')$ is a valid
  typed closure.
\end{itemize}
\item[Circuit generating rules:] These rules represent the most interesting cases. 
We treat them individually.
\begin{itemize}
  \item $(box)$: the reduction rule is
  \[
  \infer[]{[C,box^T(v)]\to [C,(t,D,b)]}{
    \spec(T)=t
    &
    [\mathtt{new}(FQ(t)), vt] \to [D,b]
  }
  \]
  with $a=box^T(v)$ and $a'=(t,D,b)$. Since $v$ is a value, we know by  
  Lemma~\hyperref[context_value]{\ref*{context_value}} that the typing 
  derivation $\pi$ is
  \[
  \infer[]{!\Delta;\emptyset\entails box^T(v):!^nCirc(T,U)}{
    \infer[]{!\Delta;\emptyset \entails box^T:!(T\loli U)\loli !^nCirc(T,U)}{
    }   
    &
    \deduce[]{!\Delta ;\emptyset\entails v:!(T\loli U)}{
     \vdots ~\pi_1
    }
  }
  \]
  and $!\Delta;\emptyset\entails box^T(v):!^nCirc(T,U),(Q'|Q'')$ is valid.
  By Lemma~\hyperref[specimen]{\ref*{specimen}}, there exists 
  a typing derivation $\pi_2$ of $FQ(t)\entails t:T$ and applying 
  Lemma~\hyperref[weakening]{\ref*{prop_type_syst}.\ref*{weakening}} yields a typing 
  derivation $\pi_2'$ of $!\Delta;FQ(t)\entails t:T$. 
  Applying Lemma~\hyperref[subtype]{\ref*{prop_type_syst}.\ref*{subtype}}
  to $\pi_1$ we get a derivation $\pi_1'$ of $!\Delta ;\emptyset\entails v:T\loli U$.
  We can therefore construct the following derivation:
  \[
  \infer[]{!\Delta;FQ(t)\entails vt : U}{
    \deduce[]{!\Delta ;\emptyset\entails v:T\loli U}{
      \vdots ~\pi_1'
    }
    &
    \deduce[]{!\Delta;FQ(t)\entails t:T}{
      \vdots ~\pi_2'
    }
  }
  \]
  Moreover, $FQ(t)=\mathtt{Out}(\mathtt{New}(FQ(t)))=\mathtt{In}(\mathtt{New}(FQ(t)))$. 
  Therefore $!\Delta;FQ(t)\entails [\mathtt{New}(FQ(t)),vt]:U,(FQ(t)|\emptyset)$ is a 
  valid typed closure. By the induction hypothesis, this implies that 
  $!\Delta;FQ(b)\entails [D,b]:U,(FQ(t)|\emptyset)$ is also valid.
  In particular, this means that $\mathtt{In}(D)=FQ(t)$, $\mathtt{Out}(D)=FQ(b)$ 
  and there exists a typing derivation $\pi_3$ of $!\Delta;FQ(b)\entails b:U$.
  We can therefore construct the following typing derivation:
  \[
  \infer[.]{!\Delta;\emptyset \entails (t,D,b):!^nCirc(T,U)}{
    \deduce[]{\emptyset ; FQ(t)\entails t:T}{
      \vdots~\pi_2
    }
    &
    \deduce[]{!\Delta ; FQ(b)\entails b:U}{
      \vdots~\pi_3
    } 
    &
    \deduce[]{In(D)=FQ(t) }{
      Out(D)=FQ(b)
    }
  }
  \]  
  Hence $!\Delta ;\emptyset \entails (t,D,b):!^nCirc(T,U) (Q'|Q'')$ is valid.
  \item $(unbox)$ the reduction rule is
  \[
    \infer[(unbox)]{[C,(unbox~(t,D,t'))v]\to [C',\binding'(t')]}{
      bind(v,t)=\binding 
      &
      \mathtt{Unencap}(C,D,\binding) = (C',\binding') 
    }
  \]
  with $a=(unbox~(t,D,t'))v$ and $a'=\binding'(t')$. Since $v$ is a value of 
  QData type, we know by 
  Lemma~\hyperref[context_value]{\ref*{context_value}}
  that the typing derivation $\pi$ is 
  \begin{changemargin}{-4.8cm}{-1cm}
  \[
  \infer[]{!\Delta; FQ(v)\entails (unbox~(t,D,t'))v :U}{
    \infer[]{!\Delta;\emptyset\entails unbox(t,D,t'):T\loli U}{
      \infer[]{!\Delta;\emptyset \entails unbox:Circ(T,U)\loli (T\loli U)}{
      }   
      &
      \infer[]{!\Delta ;\emptyset\entails (t,D,t'):Circ(T,U)}{
        \deduce[]{FQ(t)\entails t:T}{
          \vdots ~\pi_1
        }
        &
        \deduce[]{!\Delta;FQ(t')\entails t':U}{
          \vdots ~\pi_2     
        }
        &
        \deduce[]{In(D)=FQ(t)}{
          Out(D)=FQ(t')
        }
      }
    }
    &
    \infer[]{!\Delta ; FQ(v)\entails v:T}{
      \vdots~\pi_3
    }
  }
  \]
  \end{changemargin}
  and $!\Delta; FQ(v)\entails (unbox~(t,D,t'))v :U,(Q'|Q'')$ 
  is valid. By applying
  Lemma~\hyperref[binding_judgement]{\ref*{binding_judgement}} 
  to $\pi_2$ we get a typing derivation $\pi_2'$ of 
  $!\Delta;FQ(\binding(t'))\entails \binding(t'):U$.
  Now note that by 
  Definition~\hyperref[Unencap_cond_3]{\ref*{circuit_constructor}.\ref*{Unencap_cond_3}} 
  we have:
  \[
  \begin{array}{rcl}
  \mathtt{Out}(C') & = & \binding(\mathtt{Out}(D)), (\mathtt{Out}(C)\setminus\binding^{-1}(\mathtt{In}(D))) \\
                   & = & \binding(FQ(t')) , ((Q'',FQ(v))\setminus \binding^{-1}(FQ(t))) \\
                   & = & FQ(\binding(t')) , ((Q'',FQ(v))\setminus FQ(v)) \\
                   & = & FQ(\binding(t')),Q''.                   
  \end{array}
  \]
  Hence $!\Delta; FQ(\binding(t'))\entails \binding(t') :U,(Q'|Q'')$  is valid.
  \item $(rev)$ the reduction rule is
  \[
    \infer[(rev)]{[C,rev (t,D,t')]\to [C,(t',D^{-1},t)]}{}
  \]
  with $a=rev(t,D,t')$ and $a'=(t',D^{-1},t)$. The typing derivation $\pi$ 
  is therefore
  \begin{changemargin}{-2.7cm}{-1cm}
  \[
  \infer[]{!\Delta;\emptyset\entails rev(t,D,t'):!^nCirc(U,T)}{
    \infer[]{!\Delta;\emptyset \entails rev:Circ(T,U)\loli !^nCirc(U,T)}{
    }   
    &
    \infer[]{!\Delta ;\emptyset\entails (t,D,t'):Circ(T,U)}{
      \deduce[]{FQ(t)\entails t:T}{
        \vdots ~\pi_1
      }
      &
      \deduce[]{!\Delta;FQ(t')\entails t':U}{
        \vdots ~\pi_2     
      }
      &
      \deduce[]{In(D)=FQ(t)}{
        Out(D)=FQ(t')
      }
    }
  }
  \]
  \end{changemargin}
  and $!\Delta;\emptyset\entails rev(t,D,t'):!^nCirc(T,U),(Q'|Q'')$ is 
  valid. Now note that since $t'$ is a QData term, it contains no variables. 
  Applying Lemma~\hyperref[unused_var]{\ref*{prop_type_syst}.\ref*{unused_var}} 
  to $\pi_2$ repeatedly we therefore get a derivation $\pi_2'$ of 
  $FQ(t')\entails t':U$. Moreover, by applying
  Lemma~\hyperref[weakening]{\ref*{prop_type_syst}.\ref*{weakening}} to $\pi_1$ 
  we get a typing derivation $\pi_1'$ of $!\Delta ,FQ(t)\entails t:T$.
  Since, by 
  Remark~\hyperref[structure-transfer]{\ref*{structure-transfer}}, 
  we have $Out(D^{-1})=In(D)=t$ and $In(D^{-1})=Out(D)=t'$, we can construct 
  the following typing derivation:
  \[
  \infer[]{!\Delta ;\emptyset\entails (t',D^{-1},t):Circ(U,T)}{
    \deduce[]{FQ(t')\entails t':U}{
      \vdots ~\pi_2'
    }
    &
    \deduce[]{!\Delta;FQ(t)\entails t:T}{
      \vdots ~\pi_1'     
    }
    &
    \deduce[]{In(D^{-1})=FQ(t')}{
      Out(D^{-1})=FQ(t')
    }
  }  
  \]
  Hence $!\Delta;\emptyset\entails (t',D,t):!^nCirc(T,U),(Q'|Q'')$ 
  is valid.    
\end{itemize}
\end{description}
\end{proof}

\begin{corollary}
\emph{(Subject Reduction)}
If $\Gamma;FQ(a)\entails [C,a]:A,(Q'|Q'')$ is a valid typed closure 
and $[C,a]\to^* [C',a']$, then $\Gamma;FQ(a')\entails [C',a']:A,(Q'|Q'')$ 
is also a valid typed closure.
\end{corollary}

\begin{proof}
By induction on the length of the reduction sequence. The base case is 
provided by Lemma~~\hyperref[subject_red_one_step]{\ref*{subject_red_one_step}}
\end{proof}

This formulation of Subject Reduction explains why a typed closure contains 
information about the input and output wires of the circuit state. Indeed, 
Subject Reduction now guarantees (1) that the input wires of a circuit 
remain unchanged through reduction and (2) that a term can only affect 
wires whose identifiers are among its quantum addresses.

\subsection{Progress}

\begin{lemma}
\label{non_values}
If a term $a$ is not a value then it is of one 
of the following forms: $(t,C,a')$ with $a'\notin \mathtt{Val}$, 
$\p{a_1,a_2}$ with $a_1\notin \mathtt{Val}$ or $a_2\notin \mathtt{Val}$, 
$if ~a_1~ then ~a_2~ else ~a_3$, $let ~\atuple{x,y}=a_1~ in ~a_2$, $a_1a_2$.
\end{lemma}

\begin{lemma}
\label{form_values}
A well-typed value $v$ is either a variable, a quantum address, a constant or one 
of the following case occurs: if it is of type $!^nCirc(T,U)$, it is of the 
form $(t,C,u)$ with $t$ and $u$ values; if it is of type $!^nBool$ it is either 
$\true$ or $\false$; if it is of type $!^n (A\x B)$, it is of the
form $\p{w,w'}$, with $w$ and $w'$ values; if it is of type $!^n1$, it is 
precisely the term $*$; if it is of type $!^n (A\loli B)$, it is a 
lambda-abstraction.
\end{lemma}

\begin{proof}
By induction on the typing derivation of $v$.
\end{proof}

\begin{proposition}
\emph{(Progress)}
If $FQ(a)\entails [C,a]:A,(Q'|Q'')$ is a valid typed closure then either 
$a\in\mathtt{Val}$ or there exists a closure $[C',a']$ such that 
$[C,a]\to [C',a']$.
\end{proposition}

First, note that Progress is stated for a typed closure whose typing 
context is empty. This is because the Lemma no longer holds if we allow 
for a non-empty typing context. Indeed, consider the following counterexample:
\[
\infer[.]{x:!(!A\loli B), y:!A;\emptyset\entails xy:B}{
 \infer[]{x:!(!A\loli B);\emptyset\entails x:!A\loli B}{ 
 }
 &
 \infer[]{y:!A;\emptyset\entails y:A}{ 
 }
}
\]
We now prove the Proposition.

\begin{proof}
We prove the proposition by induction on the typing derivation $\pi$ of 
$FQ(a)\entails a:A$. If $a$ is a value then there is nothing 
to prove. If $a$ is not a value, then by 
Lemma~\hyperref[non_values]{\ref*{non_values}} there are 5 cases to consider. 
In each case we show that $[C,a]$ is reducible in the sense that there exists a 
closure  $[C,b]$ such that $[C,a]\to[C,b]$
\begin{enumerate}
  \item If $a=(t,D,a')$ with $a'\notin \mathtt{Val}$, then the typing derivation 
  $\pi$ is:
  \[
  \infer[.]{\emptyset\entails (t,D,a'):Circ(T,U)}{
    \deduce[]{FQ(t)\entails t:T}{
      \vdots ~\pi_1
    }
    &
    \deduce[]{FQ(a')\entails a':U}{
      \vdots ~\pi_2     
    }
    &
    \deduce[]{In(D)=FQ(t)}{
      Out(D)=FQ(a')
    }
  }   
  \]
  The typed closure 
  \[
  \begin{array}{rcl}
  FQ(a') & \entails & [D,a']:U,(FQ(t)|\emptyset)
  \end{array}
  \]
  is therefore valid. Since $a'$ 
  is not a value, the induction hypothesis implies that there exists $a''$ such 
  that $[D,a']\to [D',a'']$ and $[C,(t,D,a')]$ therefore reduces to 
  $[C,(t,D',a'')]$ by the $(circ)$ reduction rule.
  \item If $a=\p{a_1,a_2}$ with $a_1\notin \mathtt{Val}$ or $a_2\notin \mathtt{Val}$,
  then the typing derivation $\pi$ is:
  \[
  \infer[.]{FQ(a_1),FQ(a_2)\entails \p{a_1,a_2}:!^n(A_1\x A_2)}{
    \deduce[]{FQ(a_1)\entails a_1:!^nA_1}{
      \vdots~\pi_1
    }
    & 
    \deduce[]{FQ(a_2)\entails a_2:!^nA_2}{
      \vdots~\pi_2
    }
  }
  \] 
  The typed closures 
  \[
  \begin{array}{rcl}
  FQ(a_1) & \entails & [C,a_1]:!^nA_1,(Q'|FQ(a_2),Q'')\\
  FQ(a_2) & \entails & [C,a_2]:!^nA_1, (Q'|FQ(a_1),Q'')
  \end{array}
  \]
  are therefore both valid. 
  Now if $a_2\notin\mathtt{Val}$, then by the induction hypothesis 
  $[C,a_2]\to[C',a_2']$. Hence $[C,\p{a_1,a_2}]$ reduces to $[C',\p{a_1,a_2'}]$ 
  by the $(right)$ reduction rule.
  If on the other hand $a_2\in\mathtt{Val}$, then it must be the case that $a_1\notin\mathtt{Val}$ 
  and we can conclude by reasoning analogously that $[C,\p{a_1,a_2}]$ reduces to 
  some $[C',\p{a_1',a_2}]$ by the $(left)$ reduction rule.. 
  \item If $a=if ~a_1~ then ~a_2~ else ~a_3$, then the typing derivation $\pi$ is:
  \[
  \infer[.]{FQ(a_1),Q\entails if ~a_1~ then ~a_2~ else ~a_3:A}{
    \deduce[]{FQ(a_1)\entails a_1:Bool }{
      \vdots~\pi_1
    }
    &
    \deduce[]{Q \entails a_2:A}{
      \vdots~\pi_2
    }    
    &
    \deduce[]{Q \entails a_3:A}{
      \vdots~\pi_3
    }    
  }
  \]  
  The typed closure
  \[
  \begin{array}{rcl}
  FQ(a_1) & \entails & [C,a_1]:Bool,(Q'|FQ(a_2),FQ(a_3),Q'')
  \end{array}
  \]
  is therefore valid. Now if $a_1\notin\mathtt{Val}$, then by the induction hypothesis 
  $[C,a_1]\to[C',a_1']$ and thus $[C,if ~a_1~ then ~a_2~ else ~a_3]$ reduces 
  to $[C',if ~a_1'~ then ~a_2~ else ~a_3]$ by the $(cond)$ reduction rule. 
  If on the other hand $a_1\in\mathtt{Val}$, then by 
  Lemma~\hyperref[form_values]{\ref*{form_values}} either $a_1=\true$ or $a_1=\false$. 
  Thus $[C,if ~a_1~ then ~a_2~ else ~a_3]$ reduces either to $[C,a_2]$ by the 
  $(if\mbox{-}\mathtt{T})$ reduction rule or 
  to $[C,a_3]$ by the $(if\mbox{-}\mathtt{F})$ reduction rule.
  \item If $a=let ~\atuple{x,y}=a_1~ in ~a_2$, then we can reason as above to show that 
  if $a_1$ is not a value, then the $(let)$ congruence rule applies and that if 
  $a$ is a value then Lemma~\hyperref[form_values]{\ref*{form_values}} 
  guarantees that the $(pair)$ rule applies.
  \item If $a=a_1a_2$ then the typing derivation $\pi$ is: 
  \[
    \infer[.]{FQ(a_1),FQ(a_2)\entails a_1a_2:A}{
      \deduce[]{FQ(a_1)\entails a_1:B\loli A}{
        \vdots~\pi_1
      }
      &
      \deduce[]{FQ(a_2)\entails a_2:B}{
        \vdots~\pi_2
      }      
    }
  \]  
  The typed closures
  \[
  \begin{array}{rcl}
  FQ(a_1) & \entails & [C,a_1]:B\loli A,(Q'|FQ(a_2),Q'')\\
  FQ(a_2) & \entails & [C,a_2]:B, (Q'|FQ(a_1),Q'')
  \end{array}
  \]
  are therefore valid. There are three cases to treat.
  \begin{itemize}
    \item If $a\notin\mathtt{Val}$, then by the induction hypothesis 
    $[C,ab]\to[C',a'b]$.
    \item If $a\in\mathtt{Val}$ and $b\notin\mathtt{Val}$, then by 
    the induction hypothesis $[C,ab]\to[C',ab']$.
    \item $a,b\in\mathtt{Val}$ then by 
    Lemma~\hyperref[form_values]{\ref*{form_values}}, $a$ is either 
    an abstraction or a constant. In each case the appropriate rule 
    among $(\beta)$, $(box)$, $(unbox)$ and $(rev)$ applies. 
  \end{itemize}
\end{enumerate}
\end{proof}

\end{document}

